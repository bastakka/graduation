\documentclass[11pt]{article}
\usepackage[a4paper, total={7in, 10in}]{geometry} % Border sizes
\usepackage[T1]{fontenc} % LaTeX default font encoding 
\usepackage{multirow} % LaTeX multirow package for table
\usepackage{titlesec} % LaTeX titlesec package for section title changes
\usepackage{sectsty} % For section styles
\usepackage{multicol} % For multi-column tables
\usepackage{graphicx} % For images
\usepackage{setspace}

\sectionfont{\centering} % Center section title

\begin{document}
    \pagenumbering{gobble} % Remove page numbers

    \section*{Maturitní otázky AUT}
        \begin{center}
            \Large Karel Bašta - V4D
        \end{center}
        \setstretch{0.9}
        \begin{enumerate}
            \item DSP - co je to, oblasti využití (průzkum, zdravotnictví, zvuk, video apod.)
            \item Automatizace ve výrobě (popis technol. procesu, TQM a efektivnost automatizace, sociální aspekty)
            \item DSP - využití statistiky (střední hodnota, standardní odchylka, vztah k SNR a CV)
            \item Kybernetika - řízení a toky signálů (základní blok. schéma, základní relace- sériová, paralelní, zpětná (antiparalelní), systémy determinované, stochastické a fuzzy, základní řízená soustava)
            \item DSP - lineární systémy (základní vlastnosti - homogenita, aditivita a shift invariance)
            \item Regulace - základní pojmy (schéma a popis regulátoru, regulovaný obvod, co je ovládání, regulace a kybernetické řízení, statické vlastnosti, lineární x kvazilineární, druhy nelinearit)
            \item Arduino - práce s I/O periférií (ošetření tlačítka proti zákmitům) dle přiložených propozic k úloze
            \item Arduino - práce s I/O periférií (multifunkční shield a práce se 7segment zobrazovačem) dle přiložených propozic k úloze
            \item DSP - využití pravděpodobnosti (procesy na pozadí, histogram, pdf (pmf), cdf, příklady)
            \item DSP - digitální šum (princip generování náhodných čísel, využití v praxi, přesnost a pojmy)
            \item DSP - ADC a DAC, princip fce (funkční bloky AD a DA, specifika a význam pro DSP vzorkovací teorém)
            \item Dělení regulátorů (kriteriální podle energií, napájení, průběhu přenosu signálů, přesnosti, spolehlivosti a linearity, srovnání diskrétní x spojité řízení, principy diskrétního řízení)
            \item Arduino - práce s I/O periférií (zpracování dat ze senzoru a zpracování výstupu) dle přiložených propozic k úloze
            \item DSP - konvoluce, princip a její vlastnosti (algoritmy ze strany vstupu / výstupu)
            \item DSP - filtry pro ADC a DAC (vlastnosti, požadavky, praktická řešení)
            \item MCU - práce s interními perifériemi (I/O a Timer) v asynchronním režimu (režim přerušení) dle přiložených propozic k úloze
            \item Arduino - základní vlastnosti MCU Atmel 328p (paměti a periferie), board Arduino UNO (pinout, způsob programování)
            \item PLC - popis LOGO! (princip funkce, technické řešení, blokové schéma řízeného procesu, popis jeho jednotlivých částí, vstupy, výstupy, příklady procesů)
            \item MCU - blokové schéma PIC16F84A (vnitřní architektura CPU, interní periférie, instrukční sada)
            \item MCU - rtOS (základní princip ET (event trigering) a TT (time trigening) aplikací, popis jádra rtOS, postup návrhu multitaskových aplikací)
            \item MCU - funkčnost vnitřních periférií PIC16F84A (Timer0, EEPROM, I/O porty PA a PB)
            \item Praktická řešení stavových vstupních signálů pro regulační obvody (čidla teploty, tlaku, osvětlení, hladiny kapalin, principy, vlastnosti a požadavky na ně kladené)
            \item Arduino - programování s využitím knihoven (měření, zobrazení, periferie) dle přiložených propozic k úloze
            \item Sběrnice I2C a 1-Wire - popis, protokol, rychlost, zapojení a využití
            \item Arduino - práce s EEPROM dle přiložených propozic k úloze
        \end{enumerate}
        \setstretch{1}
\end{document}