\documentclass[a4paper,11pt]{article}
\usepackage[czech]{babel}
\usepackage[a4paper, total={7in, 10in}]{geometry} % Border sizes
\usepackage[T1]{fontenc} % LaTeX default font encoding 
\usepackage{multirow} % LaTeX multirow package for table
\usepackage{titlesec} % LaTeX titlesec package for section title changes
\usepackage{sectsty} % For section styles
\usepackage{multicol} % For multi-column tables
\usepackage{graphicx} % For images
\usepackage{gensymb} % For degree /degree
\usepackage{charter} % use the charter font
\usepackage{enumitem} % Enumerated item list
\usepackage{listings} % For source codes
\usepackage{xcolor} % For coloring
\usepackage{setspace} 
\usepackage{csquotes}
\DeclareQuoteAlias{german}{czech}
\MakeOuterQuote{"}

\sectionfont{\centering} % Center section title
\def\svgwidth{\columnwidth} % Set width of svg images to column width

\begin{document}
\pagenumbering{gobble} % Remove page numbers

\section*{Maturitní otázky AUT}
\begin{center}
    \Large Karel Bašta - V4D
\end{center}

\setstretch{0.9}
\begin{enumerate}
    \item DSP - co je to, oblasti využití (průzkum, zdravotnictví, zvuk, video apod.)
    \item Automatizace ve výrobě (popis technol. procesu, TQM a efektivnost automatizace, sociální aspekty)
    \item DSP - využití statistiky (střední hodnota, standardní odchylka, vztah k SNR a CV)
    \item Kybernetika - řízení a toky signálů (základní blok. schéma, základní relace- sériová, paralelní, zpětná (antiparalelní), systémy determinované, stochastické a fuzzy, základní řízená soustava)
    \item DSP - lineární systémy (základní vlastnosti - homogenita, aditivita a shift invariance)
    \item Regulace - základní pojmy (schéma a popis regulátoru, regulovaný obvod, co je ovládání, regulace a kybernetické řízení, statické vlastnosti, lineární x kvazilineární, druhy nelinearit)
    \item Arduino - práce s I/O periférií (ošetření tlačítka proti zákmitům) dle přiložených propozic k úloze
    \item Arduino - práce s I/O periférií (multifunkční shield a práce se 7segment zobrazovačem) dle přiložených propozic k úloze
    \item DSP - využití pravděpodobnosti (procesy na pozadí, histogram, pdf (pmf), cdf, příklady)
    \item DSP - digitální šum (princip generování náhodných čísel, využití v praxi, přesnost a pojmy)
    \item DSP - ADC a DAC, princip fce (funkční bloky AD a DA, specifika a význam pro DSP vzorkovací teorém)
    \item Dělení regulátorů (kriteriální podle energií, napájení, průběhu přenosu signálů, přesnosti, spolehlivosti a linearity, srovnání diskrétní x spojité řízení, principy diskrétního řízení)
    \item Arduino - práce s I/O periférií (zpracování dat ze senzoru a zpracování výstupu) dle přiložených propozic k úloze
    \item DSP - konvoluce, princip a její vlastnosti (algoritmy ze strany vstupu / výstupu)
    \item DSP - filtry pro ADC a DAC (vlastnosti, požadavky, praktická řešení)
    \item MCU - práce s interními perifériemi (I/O a Timer) v asynchronním režimu (režim přerušení) dle přiložených propozic k úloze
    \item Arduino - základní vlastnosti MCU Atmel 328p (paměti a periferie), board Arduino UNO (pinout, způsob programování)
    \item PLC - popis LOGO! (princip funkce, technické řešení, blokové schéma řízeného procesu, popis jeho jednotlivých částí, vstupy, výstupy, příklady procesů)
    \item MCU - blokové schéma PIC16F84A (vnitřní architektura CPU, interní periférie, instrukční sada)
    \item MCU - rtOS (základní princip ET (event trigering) a TT (time trigening) aplikací, popis jádra rtOS, postup návrhu multitaskových aplikací)
    \item MCU - funkčnost vnitřních periférií PIC16F84A (Timer0, EEPROM, I/O porty PA a PB)
    \item Snímače - základní pojmy (snímač, čidlo, měřící řetězec se SMART snímačem), požadavky na snímače, parametry (přesnost, rozlišení, linearita, životnost, šum)
    \item Arduino - programování s využitím knihoven (měření, zobrazení, periferie) dle přiložených propozic k úloze
    \item Sběrnice I2C a 1-Wire - popis, protokol, rychlost, zapojení a využití
    \item Snímače - druhy snímačů, princip činnosti, průběh výstupního signálu, konstrukce snímačů (teplota, tlak, síla, poloha, úhel, rychlost, otáčky, hladina)
\end{enumerate}

\setstretch{1}
\setlist{nolistsep} % No list separator

\section{DSP - co je to, oblasti využití}
\section{Automatizace ve výrobě}
\section{DSP - využití statistiky}
\section{Kybernetika - řízení a toky signálů}
\section{DSP - lineární systémy}
\section{Regulace - základní pojmy}
\section{Arduino - práce s I/O periférií - Tlačítko}
\section{Arduino - práce s I/O periférií - MFShielf, 7 Segment}
\section{DSP - využití pravděpodobnosti}
\section{DSP - digitální šum}
\section{DSP - ADC a DAC, princip fce}
\section{Dělení regulátorů}
\section{Arduino - práce s I/O periférií - zpracování dat z I/O}
\section{DSP - konvoluce, princip a její vlastnosti}
\section{DSP - filtry pro ADC a DAC}
\section{MCU - práce s interními perifériemi (I/O a Timer) v asynchronním režimu}
\section{Arduino - základní vlastnosti MCU Atmel 328p, board Arduino UNO}
\section{PLC - popis LOGO!}
\section{MCU - blokové schéma PIC16F84A}
\section{MCU - rtOS}
\section{MCU - funkčnost vnitřních periférií PIC16F84A}
\include{AUT/Snimace-obecne/Snimace-obecne.tex}
\section{Arduino - programování s využitím knihoven}
\section{Sběrnice I2C a 1-Wire}
\section{Snímače - druhy snímačů, princip činnosti, průběh výstupního signálu, konstrukce snímačů}


\end{document}
