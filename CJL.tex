\documentclass[a4paper,11pt]{article}
\usepackage[czech]{babel}
\usepackage[a4paper, total={7in, 10in}]{geometry} % Border sizes
\usepackage[T1]{fontenc} % LaTeX default font encoding 
\usepackage{multirow} % LaTeX multirow package for table
\usepackage{titlesec} % LaTeX titlesec package for section title changes
\usepackage{sectsty} % For section styles
\usepackage{charter} % use the charter font
\usepackage{tabularx} % For tables
\usepackage{csquotes} % for Czech quotes
\usepackage{hyperref} % For hyperlinks across the PDF
\usepackage{fancyhdr} % For headers and footers
\usepackage[bottom,norule]{footmisc} % For footnotes

% footnote in footer
\newcommand{\fancyfootnotetext}[2]{%
  \fancypagestyle{dingens}{%
    \fancyfoot[LO,RE]{\parbox{12cm}{\footnotemark[#1]\footnotesize #2}}%
  }%
  \thispagestyle{dingens}%
}

\DeclareQuoteAlias{german}{czech} % For Czech quotes
\MakeOuterQuote{"} % For Czech quotes

\sectionfont{\centering} % Center section title
\pagestyle{fancy} % Fancy page style

\begin{document}
\pagenumbering{gobble} % Remove page numbers

\section*{Seznam maturitní četby}
\begin{center}
    \Large Karel Bašta - V4D
\end{center}
\subsection*{A)	Česká a světová literatura do konce 18. století - 2}
\begin{enumerate}
    \item \hyperref[sec:lakomec]{Molière: Lakomec}
    \item \hyperref[sec:romeoajulie]{Shakespeare, William: Romeo a Julie}
\end{enumerate}
\subsection*{B)	Česká a světová literatura do konce 19. století - 7}
\begin{enumerate}
    \setcounter{enumi}{2}
    \item \hyperref[sec:krallavra]{Borovský, Karel Havlíček: Král Lávra}
    \item \hyperref[sec:kytice]{Erben, Karel Jaromír: Kytice}
    \item \hyperref[sec:kulicka]{Maupassant, Guy de: Kulička}
    \item \hyperref[sec:marysa]{Mrštíkové, A. a V.: Maryša}
    \item \hyperref[sec:pikovadama]{Puškin, Alexandr Sergejevič: Piková dáma}
    \item \hyperref[sec:ostrovpokladu]{Stevenson, Robert Louis: Ostrov pokladů}
    \item \hyperref[sec:nocnakarlstejne]{Vrchlický, Jaroslav: Noc na Karlštejně}
\end{enumerate}
\subsection*{C)	Světová literatura 20. a 21. století - 5}
\begin{enumerate}
    \setcounter{enumi}{9}
    \item \hyperref[sec:orientexpres]{Christie, Agatha: Vražda v Orient expresu}
    \item \hyperref[sec:farmazvirat]{Orwell, George: Farma zvířat}
    \item \hyperref[sec:malyprinc]{Saint-Exupéry, Antoine de: Malý princ}
    \item \hyperref[sec:omysichalidech]{Steinbeck, John: O myších a lidech}
    \item \hyperref[sec:hobit]{Tolkien, John Ronald Reuel: Hobit}
\end{enumerate}
\subsection*{D)	Česká literatura 20. a 21. století - 6}
\begin{enumerate}
    \setcounter{enumi}{14}
    \item \hyperref[sec:bilanemoc]{Čapek, Karel: Bílá Nemoc}
    \item \hyperref[sec:hrdybudzes]{Dousková, Irena: Hrdý Budžes}
    \item \hyperref[sec:spalovacmrtvol]{Fuks, Ladislav: Spalovač mrtvol}
    \item \hyperref[sec:saturnin]{Jirotka, Zdeněk: Saturnin}
    \item \hyperref[sec:edison]{Nezval, Vítězslav: Edison}
    \item \hyperref[sec:hospodanamytince]{Smoljak, Ladislav a Svěrák, Zdeněk: Hospoda na mýtince}
\end{enumerate}

\section{Lakomec}
    \subsection*{Základní informace}
        \begin{center}
            \begin{tabular}{l|l}
                \textbf{Literární forma:} Forma & \textbf{Literární druh:} Druh \\
                \hline
                \textbf{Slohový postup:} Postup & \textbf{Typ vypravěče:} Typ \\
                \hline
                \textbf{Způsob vypravování:} Způsob & \textbf{Žánr:} Žánr \\
                \hline
                \multicolumn{2}{l}{\textbf{Prostředí:} Prostředí} \\
                \hline
                \multicolumn{2}{l}{\textbf{Jazykové prostředky:} Prostředky} \\
            \end{tabular}
        \end{center}
    \subsection*{Postavy}
        \begin{center}
            \begin{tabular}{l|l}
                \multirow{3}{15em}{\textbf{Obsáhlá postava}} & Popis postavy \\
                & Popis postavy \\
                & Popis postavy \\
                \hline
                \textbf{Jednoduchá postava} & Popis postavy \\
            \end{tabular}
        \end{center}
    \subsection*{Děj}
        První odstavec děje

        Druhý odstavec
        
        Třetí odstavec
    \subsection*{Autor}
        \begin{center}
            \begin{tabular}{l|l}
                \textbf{Jméno:} Jméno & \textbf{Období:} Období\\
                \hline
                \textbf{Původ:} Původ & \textbf{Zaměstnání:} Zaměstnání\\
                \hline
                \multicolumn{2}{l}{Povídání o autorovi} \\
                \multicolumn{2}{l}{Povídání o autorovi} \\
                \multicolumn{2}{l}{Povídání o autorovi} \\
                \hline
                \multicolumn{2}{l}{Další díla:} \\
                \multicolumn{2}{l}{\textbf{Dílo 1 -} Popis díla 1} \\
                \multicolumn{2}{l}{\textbf{Dílo 2 -} Popis díla 2} \\
                \multicolumn{2}{l}{\textbf{Dílo 3 -} Popis díla 3} \\
                \hline
                \multicolumn{2}{l}{Podobní autoři:} \\
                \multicolumn{2}{l}{\textbf{Směr/období 1:} Autor 1, autor 2, autor 3} \\
                \multicolumn{2}{l}{\textbf{Směr/období 2:} Autor 1, autor 2, autor 3} \\
                \multicolumn{2}{l}{\textbf{Směr/období 3:} Autor 1, autor 2, autor 3} \\
            \end{tabular}
        \end{center}
\section{Romeo a Julie}
\subsection*{Základní informace}
\begin{tabularx}{\linewidth}{l|l}
    \textbf{Literární forma:} Drama                                & \textbf{Literární druh:} Drama - Tragédie   \\
    \hline
    \textbf{Slohový postup:} Drama                                 & \textbf{Typ vypravěče:} Drama               \\
    \hline
    \textbf{Způsob vypravování:} Chronologické, prolog + 5 dějství & \textbf{Směr:} Renesance                    \\
    \hline
    \multicolumn{2}{l}{\textbf{Prostředí:} 16. století, Verona, Itálie}                                          \\
    \hline
    \multicolumn{2}{l}{\textbf{Jazykové prostředky:} Blankverse, metafory, citové zabarvení, oxymoróny, inverze} \\
\end{tabularx}
\subsection*{Postavy}
\begin{tabularx}{\linewidth}{l|l}
    \textbf{Montek, Kapulet}           & Hlavy dvou znepřátelených rodů                                       \\
    \hline
    \multirow{2}{15em}{\textbf{Romeo}} & Syn Monteka, zamilovaný do Julie                                     \\
                                       & Hrdý, čestný, citlivý, romantický                                    \\
    \hline
    \textbf{Merkucio}                  & Přítel Romea, agresivní a bojovný                                    \\
    \hline
    \textbf{Benvolio}                  & Přítel Romea, rozumný, mírný, čestný, spravedlivý, zábavný           \\
    \hline
    \multirow{3}{15em}{\textbf{Julie}} & Dcera Kapuletova, zamilovaná do Romea                                \\
                                       & Tvrdohlavá, obětavá, neposlouchala rodiče                            \\
                                       & 14 let                                                               \\
    \hline
    \textbf{Tybalt}                    & Bratranec Julie, bojovný, útočný, provokativní                       \\
    \hline
    \textbf{Paris}                     & Nápadník Julie, vytvrvalý, upřímný                                   \\
    \hline
    \textbf{Chůva}                     & Chůva Julie, nerozumná, ale spolehlivá a důvěryhodná, pomáhala Julii \\
    \hline
    \textbf{Otec Vavřinec}             & Starý kněz ochotný pomoci, moudrý, vzdělaný, čestný                  \\
\end{tabularx}
\subsection*{Děj}
Romeo a Julie čerpá z příběhu tragické lásky dvou mladých lidí ze znepřátelených rodů Monteků a Kapuletů v italské Veroně.
Rod Kapuletů pro svou jedinou dceru Julii připravuje zásnubní maškarní bál, kde ji chce zaslíbit mladému šlechtici Parisovi.
Na tento bál pronikne i jediný Montekův syn Romeo a spolu s Julií se během něj do sebe zamilují a následně se i tajně vezmou.

Romeo je u šarvátky, v níž jeho přítele Merkucia zabije Juliin bratranec Tybalt.
Vyzve ho proto na souboj, ve kterém jej zabije, za což je vypovězen z města.

Julie chce uniknout svatbě s Parisem, vypije proto nápoj, po kterém její tělo ustrne v klinické smrti.
K Romeovi se však zpráva o jejím záměru nedostane, a pokládá ji proto za mrtvou.
Romeo nad její hrobkou nejprve probodne v souboji Parise a poté sám vypije jed.
Julie se bohužel probudí pozdě, a jakmile zjistí, že Romeo je mrtev, vezme jeho dýku a v žalu se probodne.

Oba rody se nad mrtvolami svých potomků následně usmiřují...

\subsection*{Autor}
\begin{tabularx}{\linewidth}{l|l}
    \textbf{Jméno:} William Shakespeare & \textbf{Období:} 1564-1616 (Druhá pol. 16. st. - Začátek 17. st.) \\
    \hline
    \textbf{Původ:} Anglie              & \textbf{Zaměstnání:} Básník a dramatik                            \\
    \hline
    \multicolumn{2}{l}{Vystudoval gymánium a stal se spolumajitelem divadla}                                \\
    \multicolumn{2}{l}{Ze začátku tvořil komedie, poté tragédie, tvořil také historické hry}                \\
    \multicolumn{2}{l}{Náměty čerpal ze životopisů slavných osobností}                                      \\
    \hline
    \multicolumn{2}{l}{Další díla:}                                                                         \\
    \multicolumn{2}{l}{\textbf{Zkrocení zlé ženy -} Komedie}                                                \\
    \multicolumn{2}{l}{\textbf{Hamlet -} Tragédie}                                                          \\
    \multicolumn{2}{l}{\textbf{Jinřich IV. -} Historická hra}                                               \\
    \hline
    \multicolumn{2}{l}{Podobní autoři:}                                                                     \\
    \multicolumn{2}{l}{\textbf{Renesance:} Dante Alighiery, Giovanni Boccaccio}                             \\
\end{tabularx}

\section{Král Lávra}
\label{sec:krallavra}
\subsection*{Základní informace}
\begin{tabularx}{\linewidth}{l|l}
  \textbf{Literární forma:} Poezie (Alegorická satirická báseň) & \textbf{Literární druh:} Lyricko-Epický                        \\
  \hline
  \textbf{Slohový postup:} Vyprávěcí                            & \textbf{Typ vypravěče:} Autorský, Er-forma (Lyrický subjekt)   \\
  \hline
  \textbf{Způsob vypravování:} Chronologické                    & \textbf{Žánr:} Realismus                                       \\
  \hline
  \multicolumn{2}{l}{\textbf{Prostředí:} Čechy za rakouské nadvlády}                                                             \\
  \hline
  \multicolumn{2}{l}{\textbf{Jazykové prostředky:} ABCBDDB rým, spisovný a prostý jazyk, lidová mluva, personifikace, metonymie} \\
\end{tabularx}
\subsection*{Postavy}
\begin{tabularx}{\linewidth}{l|l}
  \multirow{2}{15em}{\textbf{Král Lávra}} & Dobrý panovník, nechce, aby nikdo věděl jeho tajemství          \\
                                          & Stydí se za svoje uši                                           \\
  \hline
  \textbf{Kukulín}                        & Holič, čestný, spravedlivý, bojácný, smířený s osudem, utrápený \\
  \hline
  \textbf{Kukulínova matka}               & Vdova, milující matka, která zachránila syna před šibenicí      \\
  \hline
  \textbf{Poustevník}                     & Žije opodál                                                     \\
  \hline
  \textbf{Basista Červíček}               & Hrál na basu a ztratil kolíček                                  \\
  \hline
  \textbf{Vypravěč}                       & Vypráví děj, nezasahuje do něj                                  \\
  \hline
\end{tabularx}
\subsection*{Děj}
Irská pohádka s českými motivy vypráví o podivuhodném králi, který byl dobrý král, ale měl jednu chybu.
Jednou do roka k sobě zavolal holiče, na kterého padl los, nechal se oholit, a pak dal lazebníka popravit.
Lid se velmi divil tomu, že slušně vypadající král může být takový tyran.

Jednou padl los na mladého Kukulína, jediného syna staré vdovy.
Vdova, ale šla za králem a hezky mu pověděla, co si o něm myslí.
Král se velmi zastyděl, a tak k sobě zavolal Kukulína.
Kukulín musel přísahat, že nikomu nepoví, co viděl.
Kukulín přísahal a stal se pak dvorním holičem.

Za nějaký čas však Kukulína tajemství velmi tížilo.
Matka to na něm poznala a poradila mu, aby zašel za poustevníkem do pustého lesa.
Poustevník mu poradil, aby tajemství pošeptal do vykotlané vrby.
Kukulín tak učinil a velmi se mu ulevilo.

Po králově ,,uzdravení“ byl uspořádán bál.
Na bál šel hrát také pan Červíček, který však cestou ztratil kolíček z basy.
Uřízl si tedy větev z vrby a vyřezal si nový kolíček.
Nevěděl však, do tím způsobí.
Když pak začal v paláci hrát, tu řve basa: ,,Král Lávra má oslí uši, král je ušatec!“
Král dal Červíčka okamžitě vyhodit.
Basu však nebylo možno pověsit, a tak král nechal svoje vlasy úplně ostříhat a nosil své dlouhé uši veřejně bez futrálu.
\subsection*{Autor}
\begin{tabularx}{\linewidth}{l|l}
  \textbf{Jméno:} Karel Havlíček Borovský & \textbf{Období:} 1821-1856 (4. fáze národního obrození)             \\
  \hline
  \textbf{Původ:} Česká republika         & \textbf{Zaměstnání:} Básník, novinář, ekonom, překladatel a politik \\
  \hline
  \multicolumn{2}{l}{Jméno Borovský je odevozeno z místa narození - Borová u Přibyslavi}                        \\
  \multicolumn{2}{l}{Napadá absolutismus a církev}                                                              \\
  \multicolumn{2}{l}{Umírá na tuberkulózu}                                                                      \\
  \hline
  \multicolumn{2}{l}{Další díla:}                                                                               \\
  \multicolumn{2}{l}{\textbf{Tyrolské elegie -} žalozpěv}                                                       \\
  \multicolumn{2}{l}{\textbf{Křest svatého Vladimíra -} satira}                                                 \\
  \multicolumn{2}{l}{\textbf{Obrazy z Rus -} cestopis}                                                          \\
  \hline
  \multicolumn{2}{l}{Podobní autoři:}                                                                           \\
  \multicolumn{2}{l}{\textbf{Realismus:} Josef Kajetán Tyl, Karel Jaromír Erben, Božena Němcová}                \\
  \multicolumn{2}{l}{\textbf{Zahraničí:} Honoré de Balzac, Guy de Maupassant}                                   \\
\end{tabularx}

\section{Kulička}
\subsection*{Základní informace}
\begin{tabularx}{\linewidth}{l|l}
    \textbf{Literární forma:} Próza (Povídka)    & \textbf{Literární druh:} Epika                                         \\
    \hline
    \textbf{Slohový postup:} Vyprávěcí a popisný & \textbf{Typ vypravěče:} Autorský, er-forma                             \\
    \hline
    \textbf{Způsob vypravování:} Chronologické   & \textbf{Žánr:} Naturalismus (Realismus)                                \\
    \hline
    \multicolumn{2}{l}{\textbf{Prostředí:} Zima roku 1870 - Normandie, Francie}                                           \\
    \hline
    \multicolumn{2}{l}{\textbf{Jazykové prostředky:} Hovorové prostředky, nářečí, archaismy, spisovné, přirovnání, popis} \\
\end{tabularx}
\subsection*{Postavy}
\begin{tabularx}{\linewidth}{l|l}
    \multirow{3}{15em}{\textbf{Alžběta Roussetová}}           & Hlavní postava, lehká děva menšího, ale širšího vzrůstu.              \\
                                                              & Je velmi hodná, ale i přesto jí spolucestující opovrhují.             \\
                                                              & Má syna, která vyrůstá v jiné vesnici, vídá ho jednou ročně.          \\
    \hline
    \multirow{2}{15em}{\textbf{Pan a paní Loiseauovi}}        & \textbf{Pan:} Majitel obchodu s vínem, vychytralý, humorný            \\
                                                              & \textbf{Paní:} Statná, rázná, vnášela do obchodu řád a počty          \\
    \hline
    \multirow{2}{15em}{\textbf{Pan a paní Carré-Lamndovi}}    & \textbf{Pan:} Vážený člověk, majitel 3 prádelen, člen městské rady    \\
                                                              & \textbf{Paní:} Mladší drobná a hezká                                  \\
    \hline
    \multirow{2}{15em}{\textbf{Hrabě a hraběnka de Bréville}} & \textbf{Hrabě:} Člen městské rady, majetný, vedoucí strany orleanistů \\
                                                              & \textbf{Hraběnka:} Skvělá hostitelka, měla dobré vystupování          \\
    \hline
    \textbf{Jeptišky}                                         & Dvě neustále modlící se jeptišky, mladá a stará                       \\
    \hline
    \textbf{Cornudet}                                         & Demokrat, zdědil peníze, které prochlastal, bojoval za revoluci       \\
    \hline
    \textbf{Follenvie}                                        & Tlustý majitel hostince, mluvil alsaskou francouzštinou               \\
    \hline
    \textbf{Pruští vojáci, obyvatelé měst}                    & Pracující lid a vojáci pobývající v obydlí francouzského lidu         \\
\end{tabularx}
\subsection*{Děj}
Hlavní postavy se časně ráno vydávají do města Le Havre, aby se vyhnuli nebezpečí, které představovali Němci.
Cestou kočár zapadá v závějích a cesta trvá mnohem déle, něž bylo plánováno.
Zároveň nemohou narazit na žádný hostinec, a tak se musí Kulička podělit o své zásoby se spolucestujícími.

Noc společnost přespí v hostinci v Tôtes, jenž je pod správou pruského důstojníka, který je odmítne pustit na další cestu, dokud se s ním Kulička nevyspí.
Ta, poněvadž je velká vlastenka (odmítne proto i Cornudeta), odmítne.
Ostatní s ní nejprve souhlasí, ale nakonec se jí rozhodnou přesvědčit o opaku.
Díky dobrému hereckému výkonu a pomoci od jeptišek, které musí ošetřovat v Havru nemocné Kulička nakonec udělá, co jí bylo nařízeno.

Prušák je tedy pustí, ale Kulička se od ostatních v dostavníku nedočká vděčnosti, jenom opovržení - je to tentokrát ona, kdo v kočáře hladoví…
\subsection*{Autor}
\begin{tabularx}{\linewidth}{l|l}
    \textbf{Jméno:} Guy de Maupassant          & \textbf{Období:} 1850-1893 (Druhá pol. 19. st.)                        \\
    \hline
    \textbf{Původ:} Francie, ze zámožné rodiny & \textbf{Zaměstnání:} Spisovatel, novinář a dramatik                    \\
    \hline
    \multicolumn{2}{l}{Vystudoval práva, účastnil se prusko-francouzské války}                                          \\
    \multicolumn{2}{l}{Po úspěchu knihy Kulička se věnoval jen literatuře a žurnalistice}                               \\
    \multicolumn{2}{l}{Po pokusu o sebevraždu kvůli syfilis byl v ústavu pro duševně choré v Passy u Paříže, kde umřel} \\
    \hline
    \multicolumn{2}{l}{Další díla:}                                                                                     \\
    \multicolumn{2}{l}{\textbf{Miláček -} Román o cílevědomém kariéristovi využívající ženy ke svému vzestupu}          \\
    \multicolumn{2}{l}{\textbf{Petr a Jan -} Román popisující příběh dvou bratru, jeden z nich se naštve že nedědí}     \\
    \multicolumn{2}{l}{\textbf{Silná jako smrt -} Román o malíři ve vztahu s vdanou ženou, zjistí, že je radši sám}     \\
    \hline
    \multicolumn{2}{l}{Podobní autoři:}                                                                                 \\
    \multicolumn{2}{l}{\textbf{Francie:} Gustav Flaubert, Honré de Balzac, Emile Zola}                                  \\
    \multicolumn{2}{l}{\textbf{Rusko:} Fjodor Michajlovič Dostojevskij}                                                 \\
    \multicolumn{2}{l}{\textbf{Anglie:} Charles Dickens}                                                                \\
\end{tabularx}
\section{Maryša}
\label{sec:marysa}
\subsection*{Základní informace}
\begin{tabularx}{\linewidth}{l|l}
  \textbf{Literární forma:} Drama                       & \textbf{Literární druh:} Drama - Tragédie                          \\
  \hline
  \textbf{Slohový postup:} Drama                        & \textbf{Typ vypravěče:} Drama                                      \\
  \hline
  \textbf{Způsob vypravování:} Chronologické, 5 jednání & \textbf{Směr:} Kritický Realismus                                  \\
  \hline
  \multicolumn{2}{l}{\textbf{Prostředí:} Rok 1886, moravská vesnice na Slovácku}                                             \\
  \hline
  \multicolumn{2}{l}{\textbf{Jazykové prostředky:} Jihomoravské nářečí, realistický popis, spisovná mluva u mladší generace} \\
\end{tabularx}
\subsection*{Postavy}
\begin{tabularx}{\linewidth}{l|l}
  \textbf{Lízal}         & bezcitný, lakomý, vychytralý                                \\
  \hline
  \textbf{Maryša}        & Dcera Lízala, mladá, citlivá dívka - miluje Francka         \\
  \hline

  \textbf{Lízalova žena} & Žena Lízala, krutá, přísná                                  \\
  \hline
  \textbf{Filip Vávra}   & sebevědomý, krutý, agresivní mlynář - stane se mužem Maryši \\
  \hline
  \textbf{Francek}       & pracujicí, věrný, statečný rekrut - miluje Maryšu           \\
\end{tabularx}
\subsection*{Děj}
Sedlák Lízal chtěl svou jedinou dceru Maryšu provdat za mlynáře Vávru (pro peníze), ale ona milovala chudého Francka.
Francek je odveden na vojnu.

Maryšu nutí rodiče ke sňatku s mlynářem Vávrou, otcem tří dětí, který slibuje, že se o ni dobře postará.
Jde mu hlavně o peníze, které dostane Maryša věnem, aby mohl zaplatit své dluhy.
Vávra se začíná opíjet, nestará se o rodinu a soudí se se starým Lízalem o Maryšino věno.

Lízal si konečně uvědomuje, za koho svou dceru provdal a odmítne mu peníze dát.
Po návratu najde Francek Maryšu provdanou za Vávru a vidí jenom její utrpení.
Připravuje plán společného útěku do Brna, kde našel pro sebe i pro Maryšu práci.
Maryša odmítá.

V rozčilení a opilosti chce Vávra Francka zabít, ale Maryša mu v tom zabrání.
Ráno Vávra už po několikáté lituje svého chování a slibuje, že se polepší.
Avšak Maryša mu už nevěří a ve chvíli zoufalství mu nasype do kávy jed a vzápětí se k tomu přizná.
\subsection*{Autoři}
\begin{tabularx}{\linewidth}{l|l}
  \textbf{Jméno:} Alois a Vilém Mrštíkovi & \textbf{Období:} Druhá pol. 19. st. - Začátek 20. st.       \\
  \hline
  \textbf{Původ:} Jimramov, Vysočina      & \textbf{Zaměstnání:} Dramatici a prozaikové                 \\
  \hline
  \multicolumn{2}{l}{\textbf{Alois -} Vystudovaný učitel, střídal školy}                                \\
  \multicolumn{2}{l}{\textbf{Vilém -} Překládal z ruštiny, nedokončil práva, bojuje za mravnost}        \\
  \multicolumn{2}{l}{Oba přispívali svými články do časopisů, novin.}                                   \\
  \hline
  \multicolumn{2}{l}{Další díla:}                                                                       \\
  \multicolumn{2}{l}{\textbf{Alois -} Sbírka povídek Nit stříbrná, povídky Dobré duše}                  \\
  \multicolumn{2}{l}{\textbf{Vilém -} Román Santa Lucia, literární kritika Moje sny}                    \\
  \hline
  \multicolumn{2}{l}{Podobní autoři:}                                                                   \\
  \multicolumn{2}{l}{\textbf{Čeští :} Božena Němcová, Alois Jirásek, Karel Havlíček Borovský}           \\
  \multicolumn{2}{l}{\textbf{Realismus :} Alexandr Nikolajevič Ostrovskij, Nikolai Gogol, Henrik Ibsen} \\
\end{tabularx}

\section{Piková dáma}
\label{sec:pikovadama}
\subsection*{Základní informace}
\begin{tabularx}{\linewidth}{l|l}
  \textbf{Literární forma:} Próza (Novela s fantastickým námětem) & \textbf{Literární druh:} Epika                                   \\
  \hline
  \textbf{Slohový postup:} Vyprávěcí                              & \textbf{Typ vypravěče:} Er-forma, autorský                       \\
  \hline
  \textbf{Způsob vypravování:} Chronologicky                      & \textbf{Žánr:} Romantismus                                       \\
  \hline
  \multicolumn{2}{l}{\textbf{Prostředí:} Rusko, 18. století}                                                                         \\
  \hline
  \multicolumn{2}{l}{\textbf{Jazykové prostředky:} Spisovný jazyk, přímá řeč, citáty, metafory, archaismy, přirovnání personifikace} \\
\end{tabularx}
\subsection*{Postavy}
\begin{tabularx}{\linewidth}{l|l}
  \multirow{2}{15em}{\textbf{Heřman}} & Ctižádostivý muž                                                    \\
                                      & Udělal by cokoli pro splacení dluhu                                 \\
  \hline
  \textbf{Hraběnka Anna Fedotovna}    & Rozumná, mazaná, komanduje Lizavetu, nepřipouští si konec své slávy \\
  \hline
  \textbf{Lizaveta Ivanovna}          & Služka hraběnky, mladá, naivní, hodná, srdečná, pracovitá, citlivá  \\
\end{tabularx}
\subsection*{Děj}
Vše začíná u karetní hry, kde Tomskij začne vyprávět příběh své babičky hraběnky Anny Fedotovny.
Ta jednou prohrála všechny peníze při karetní hře, ale znala jednoho hraběte, který jí poradil jak vyhrát peníze zpět - vsadit na tři karty.
Když se o této prověřené metodě jak získat peníze dozví Heřman, naplánuje taktiku, jak by se mohl s hraběnkou setkat.
Začne psát milostné dopisy její služebné Lizavetě Ivanově.
Ta mu prozradí, jak se dostat do ložnice hraběnky.
Heřman vtrhne do pokoje a vyhrožuje hraběnce s pistolí v ruce.
Hraběnka nechápe a pak náhle umře.

Za pár dní se mu zjeví duch staré hraběnky, poví mu tajemství tří karet, ovšem za podmínky, že si vezme Lizavetu.
První den vyhraje v kartách na trojku, druhý den na sedmu.
Třetí den vsadí všechny peníze na eso - nebo si to alespoň myslel.
Až padla poslední karta, zjistil, že vsadil na pikovou dámu.
Podíval se na kartu a zdálo se mu, že na něho mrkla hraběnka.
\subsection*{Autor}
\begin{tabularx}{\linewidth}{l|l}
  \textbf{Jméno:} Alexandr Sergejevič Puškin & \textbf{Období:} 1799 - 1837 (1. pol. 19. st.)                                  \\
  \hline
  \textbf{Původ:} Rusko                      & \textbf{Zaměstnání:} Básník, prozaik, dramatik                                  \\
  \hline
  \multicolumn{2}{l}{Považován za zakladatelel moderní ruské prózy}                                                            \\
  \multicolumn{2}{l}{Kvůli tvorbě musel odejít do vyhnanství na jih Ruska}                                                     \\
  \multicolumn{2}{l}{Až do své smrti zůstal pod policejním dohledem, zemřel na následky zranění v souboji}                     \\
  \hline
  \multicolumn{2}{l}{Další díla:}                                                                                              \\
  \multicolumn{2}{l}{\textbf{Boris Godunov -} drama}                                                                           \\
  \multicolumn{2}{l}{\textbf{Kapitánská dcera -} próza}                                                                        \\
  \multicolumn{2}{l}{\textbf{Kavkazský jezdec -} lyricka-epická skladba}                                                       \\
  \hline
  \multicolumn{2}{l}{Podobní autoři:}                                                                                          \\
  \multicolumn{2}{l}{\textbf{Rusko:} Nikolaj Vasilijevič Gogol, Michail Jurijevič Lermontov, Vasilij Alexandrejevič Žukovskij} \\
  \multicolumn{2}{l}{\textbf{Směr/období 2:} Victor Hugo, Geroge Gordon Byron, Walter Scott}                                   \\
\end{tabularx}

\section{Noc na Karlštejně}
\label{sec:nocnakarlstejne}
\subsection*{Základní informace}
\begin{tabularx}{\linewidth}{l|l}
    \textbf{Literární forma:} Drama                        & \textbf{Literární druh:} Drama - Komedie \\
    \hline
    \textbf{Slohový postup:} Drama                         & \textbf{Typ vypravěče:} Drama            \\
    \hline
    \textbf{Způsob vypravování:} Chronologické - 3 dějství & \textbf{Žánr:} Novoromantismus           \\
    \hline
    \multicolumn{2}{l}{\textbf{Prostředí:} Karlštejn, červen 1363}                                    \\
    \hline
    \multicolumn{2}{l}{\textbf{Jazykové prostředky:} Archaismy, historismy, dialogy, spisovné}        \\
\end{tabularx}
\subsection*{Postavy}
\begin{tabularx}{\linewidth}{l|l}
    \multirow{3}{15em}{\textbf{Karel IV.}} & Rozumný, čestný, spravedlivý a schopný vládce                            \\
                                           & Miluje svoji ženu Alžbětu                                                \\
                                           & Dokáže se povznést nad prohřešky                                         \\
    \hline
    \textbf{Ješek z Vartenberka}           & Purkrabí na Karlštejně, zodpovědný a dosbrosrdečný                       \\
    \hline
    \textbf{Alžběta (Eliška) Pomořanská}   & Císařovna, miluje Karla, stýská se jí a žárlí                            \\
    \hline
    \textbf{Alena}                         & Neteř Ješka, miluje Peška, odvážná, vtipná a podnikavá                   \\
    \hline
    \textbf{Pešek Hlavně}                  & Šenk jeho výsosti, zmatkař, miluje Alenu, touží po rytířských ostruhách  \\
    \hline
    \textbf{Arnošt z Pardubic}             & Arcibiskup, pomáhá všem                                                  \\
    \hline
    \textbf{Petr}                          & Král kyperský a jeruzalémský, jeho halvnímy zájmy jsou víno, zpěv a ženy \\
    \hline
    \textbf{Štěpán}                        & Vévoda bavorksý, ziskuchtivý, ale čestný, dokáže ustoupit                \\
\end{tabularx}
\subsection*{Děj}
Roku 1363 se na hrad sjíždí panovník a jím pozvaní hosté, cyperský a jeruzalémský král Petr a Štěpán Bavorský.
Přestože ženy mají na hrad vstup zakázán, přijíždějí nezávisle na sobě královna Alžběta, která neunesla stesk a žárlivost, a neteř purkrabího Alena.
Ta z lásky ke zde sloužícímu číšníku Peškovi chce vyhrát sázku uzavřenou se svým otcem, že pokud se dostane na hrad, svolí otec ke sňatku.
Obě ženy se přestrojí za pážata.

Král Petr něco tuší, žádá po pážeti polibek a při vzájemném zápase páže zlomí meč.
Císař v něm poznává svou ženu.
Následně je vyzrazena i přítomnost Aleny.
Císař je za tuto shodu okolností vděčný, neboť Alenino přestrojení může zachránit čest císařovny.

Z toho důvodu je Aleně odpuštěno, její Pešek je pasován na rytíře a císařovna úspěšně předstírá svůj příjezd na hrad s omluvou, že se při lovu ztratila v lesích.
Hra končí odjezdem Karla a jeho ženy na její hrad Karlík a novým císařovým rozhodnutím, jež zpřístupňuje Karlštejn i ženám.
\subsection*{Autor}
\begin{tabularx}{\linewidth}{l|l}
    \textbf{Jméno:} Jaroslav Vrchlický (Emil Frída) & \textbf{Období:} (2. pol. 19. st.)                                \\
    \hline
    \textbf{Původ:} Česká republika                 & \textbf{Zaměstnání:} Lumírovec, Básník, dramatik, novinář, kritik \\
    \hline
    \multicolumn{2}{l}{Vystudoval filozofickou fakultu, poté dělal vychovatele v Itálii}                                \\
    \multicolumn{2}{l}{Získal doktorát Karlovy univerzity a titul profesora}                                            \\
    \multicolumn{2}{l}{Prodělal mozkovou mrtvici, po které nemohl komunikovat, číst a psát}                             \\
    \hline
    \multicolumn{2}{l}{Další díla:}                                                                                     \\
    \multicolumn{2}{l}{\textbf{Meš Damoklův -} Vydáno posmrtně}                                                         \\
    \multicolumn{2}{l}{\textbf{Okna v bouři -} Intimní lyrická básnická sbírka}                                         \\
    \multicolumn{2}{l}{\textbf{Zlomky epopeje -} Básnická sbírka o vývoji lidstva}                                      \\
    \hline
    \multicolumn{2}{l}{Podobní autoři:}                                                                                 \\
    \multicolumn{2}{l}{\textbf{Ruchovci:} Eliška Krásnohorská, Alois Jirásek, Svatopluk Čech, Josef Václav Sládek}     \\
    \multicolumn{2}{l}{\textbf{Lumírovci:} Julius Zeyer, Josef Václav Sládek, Jaroslav Vrchlický}                       \\
\end{tabularx}

\section{Vražda v Orient expresu}
\label{sec:orientexpres}
\subsection*{Základní informace}
\begin{tabularx}{\linewidth}{l|l}
    \textbf{Literární forma:} Próza (Detektivní Román) & \textbf{Literární druh:} Epika               \\
    \hline
    \textbf{Slohový postup:} Vyprávěcí                 & \textbf{Typ vypravěče:} Autorský, er-forma   \\
    \hline
    \textbf{Způsob vypravování:} Chronologické         & \textbf{Žánr:} Žánr                          \\
    \hline
    \multicolumn{2}{l}{\textbf{Prostředí:} Vlak Orient Expres, Zima roku 1929, Jugoslávie}            \\
    \hline
    \multicolumn{2}{l}{\textbf{Jazykové prostředky:} Francoužština, přímá řeč, ironie, personifikace} \\
\end{tabularx}
\subsection*{Postavy}
\begin{tabularx}{\linewidth}{l|l}
    \multirow{3}{15em}{\textbf{Hercule Poirot}} & Belgický detektiv, velice inteligentní, přemýšlivý, bystrý a jízlivě vtipný \\
                                                & Trochu výstřední, samolibý a puntičkářský, ale genialní                     \\
                                                & Vyzná se v lidech a psychologii a při vyšetřování spoléhá jen na svůj mozek \\
    \hline
    \textbf{Samuel Edward Ratchett (Casetti)}   & Byla z něj cítit zloba, strach, krutost, byl zavražděn                      \\
    \hline
    \textbf{Hector MacQueen}                    & Američan, tajemník Ratchetta, nevěděl o tom, že Ratchett je Casetti         \\
    \hline
    \textbf{Monsieur Bouc}                      & Ředitel vlakové společnosti, přítel Poirota                                 \\
    \hline
    \textbf{Doktor Constantine}                 & Lékař, který se podílí na vyšetřování.                                      \\
    \hline
    \textbf{Armstrongovi}                       & Ostatní cestující, všichni se podíleli na vraždě                            \\ 
\end{tabularx}
\subsection*{Děj}
Poirot přijede vlakem z Aleppa do Istanbulu, kde se ubytuje v hotelu Tokatlian.
Zde dostane telegram vyzývající ho k návratu do Londýna.
Zakoupí si jízdenku na Orient expres a večer odjede.

Během první noci je jeden z cestujících zavražděn.
Nedaleko města Vinkovci vlak uvízne v závěji a musí čekat na vyproštění.
Poirot postupně vyšetřuje 13 podezřelých cestujících a ukáže se, že všichni jsou nějakým způsobem spojeni s rodinnou tragédií Johna Armstronga, kterému byla unesena a zabita dcera.
Únoscem byl zavražděný cestující.

\subsection*{Autor}
\begin{tabularx}{\linewidth}{l|l}
    \textbf{Jméno:} Agatha Christie & \textbf{Období:} 1890 - 1976 (2. pol. 20. století), neorealismus, existencionalismus \\
    \hline
    \textbf{Původ:} Anglie          & \textbf{Zaměstnání:} Lékárnice a sestra                                                 \\
    \hline
    \multicolumn{2}{l}{Druhá nejprodávanější spisovatelka všech dob (první je W. Shakespeare)}                                \\
    \multicolumn{2}{l}{Vyučována doma, poté na studiích v Paříži}                                                             \\
    \multicolumn{2}{l}{V první světové válce pracovala jako dobrovolná sestra a lékárnice}                                    \\
    \hline
    \multicolumn{2}{l}{Další díla:}                                                                                           \\
    \multicolumn{2}{l}{\textbf{Past na myši -} divadelní hra}                                                                 \\
    \multicolumn{2}{l}{\textbf{Svědek obžaloby -} divadelní hra}                                                              \\
    \multicolumn{2}{l}{\textbf{Smrt na Nilu -} detektivka s vyšetřovatelem Poirotem}                                          \\
    \hline
    \multicolumn{2}{l}{Podobní autoři:}                                                                                       \\
    \multicolumn{2}{l}{\textbf{Období :} J. R. R. Tolkien, J. K. Rowling, Vladimir Nabokov}                                   \\
    \multicolumn{2}{l}{\textbf{Umělecký směr :} Edgar Allan Poe, Robert Louis Stevenson, Oscar Wilde}                         \\
    \multicolumn{2}{l}{\textbf{Období :} P. D. Jamesová, Dick Francis, William Golding}                                                     \\
\end{tabularx}

\section{Farma zvířat}
\label{sec:farmazvirat}
\subsection*{Základní informace}
\begin{tabularx}{\linewidth}{l|l}
  \textbf{Literární forma:} Próza (Antiutopická bajka, aleg. novela) & \textbf{Literární druh:} Epika             \\
  \hline
  \textbf{Slohový postup:} Vyprávěcí a popisný                       & \textbf{Typ vypravěče:} Autorský, er-forma \\
  \hline
  \textbf{Způsob vypravování:} Chronologické                         & \textbf{Směr:} Neorealismus                \\
  \hline
  \multicolumn{2}{l}{\textbf{Prostředí:} Anglický venkov v 50. letech 20. století}                                \\
  \hline
  \multicolumn{2}{l}{\textbf{Jazykové prostředky:} Spisovný jazyk, metafory, archaismy, personifikace}            \\
\end{tabularx}
\subsection*{Postavy}
\begin{tabularx}{\linewidth}{l|l}
  \textbf{Pan Jones} & Původní majitel farmy, alkoholik              \\
  \hline
  \textbf{Napoleon}  & Prase, nejchytřejší, ale podvodník a pokrytec \\
  \hline
  \textbf{Pištík}    & Prase, lhář, přítel Napoleona                 \\
  \hline
  \textbf{Kuliš}     & Prase, chytré s dobrými nápady, čestné        \\
  \hline
  \textbf{Boxer}     & Kůň, pracovitý, čestný, dříč                  \\
  \hline
  \textbf{Benjamin}  & Osel, pasivní                                 \\
  \hline
  \textbf{Major}     & Inteligentní kanec                            \\
  \hline
  \textbf{Molina}    & Klisna, „madam“ mezi zvířaty, ráda se parádí  \\
  \hline
  \textbf{Lidé}      & Kapitalisti, vykořisťovatelé                  \\
\end{tabularx}
\subsection*{Děj}
Dílo vypráví o farmě, kde zvířata trpěla hladem.
Pan Jones propíjel peníze a svůj čas trávil v hospodě - nestaral se o ně dobře a často neměla dost jídla.
Začne je štvát, že musí sloužit lidem, chtějí pracovat jen sami pro sebe. Vzbudí se revoluce, která je úspěšná.
Po vyhnání lidí se farma přejmenuje na „Zvířecí farmu“ a jsou ustanovena pravidla (Sedm přikázání), která nesmí být překročena.

Zvířata začnou znovu pracovat, ale díky pocitu svobody tentokrát mnohem výkonněji.
V čele jsou dvě prasata (Kuliš a Napoleon), která se pořád hádají. Když jedno podá návrh, druhé ho zamítne, v lepším případě podá jiný návrh.
Napoleon vyhnal chudáka Kuliše a poslal na něj divoké psy, které si sám vychoval.
Postupně vraždí i ostatní zvířata. Napovídal jim totiž, že za tu dobu zapomněla Sedm přikázání.

Nejpracovitější zvířátko je Boxer, který se ale předře a je odvezen na jatka.
Napoleon si vychoval další prasata k obrazu svému a ty začínají vládnout celé farmě.
Začnou spolupracovat zpátky s lidmi a přejmenují farmu zpět na Panskou. S lidmi popíjejí a hrají karty.
Ostatní zvířata je od sebe nedokážou rozeznat. Prasata totiž vypadají jako lidé.
\subsection*{Autor}
\begin{tabularx}{\linewidth}{l|l}
  \textbf{Jméno:} George Orwell (Eric Arthur Blair) & \textbf{Období:} 1903-1950 (První pol. 20. st.)     \\
  \hline
  \textbf{Původ:} Anglie (Narodil se v Indii)       & \textbf{Zaměstnání:} Novinář, esejista a spisovatel \\
  \hline
  \multicolumn{2}{l}{Zapojil se do španělské občanské války}                                              \\
  \multicolumn{2}{l}{Jeho knihy nemohli v Československu oficiálně vycházet}                              \\
  \multicolumn{2}{l}{Považoval se za socialistu}                                                          \\
  \hline
  \multicolumn{2}{l}{Další díla:}                                                                         \\
  \multicolumn{2}{l}{\textbf{1984}}                                                                       \\
  \multicolumn{2}{l}{\textbf{Na dně v Paříži a Londýně}}                                                  \\
  \multicolumn{2}{l}{\textbf{Nadechnout se}}                                                              \\
  \hline
  \multicolumn{2}{l}{Podobní autoři:}                                                                     \\
  \multicolumn{2}{l}{\textbf{Sci-fi 20. st.:} Isaac Asimov, Arthur C. Clarke, Ray Bradbury}               \\
  \multicolumn{2}{l}{\textbf{Čeští sci-fi 20. st.:} Karel Čapek, Ludvík Souček}                           \\
\end{tabularx}

\section{Malý princ}
\subsection*{Základní informace}
\begin{tabularx}{\linewidth}{l|l}
    \textbf{Literární forma:} Próza (Filozofická pohádka)                           & \textbf{Literární druh:} Epika               \\
    \hline
    \textbf{Typ vypravěče:} Začátek a konec er-forma, pasáž malého prince ich-forma & \textbf{Slohový postup:} Vyprávěcí a popisný \\
    \hline
    \textbf{Způsob vypravování:} Chronologické, pasáž malého prince retrospektivní  & \textbf{Žánr:} Realismus                     \\
    \hline
    \multicolumn{2}{l}{\textbf{Prostředí:} Pravděpodobně 20. století, čas není určen, poušť a vesmír}                              \\
    \hline
    \multicolumn{2}{l}{\textbf{Jazykové prostředky:} Metafory, přirovnání, personifikace, nedokončené výpovědi, epiteton, symboly} \\
\end{tabularx}
\subsection*{Postavy}
\begin{tabularx}{\linewidth}{l|l}
    \multirow{3}{15em}{\textbf{Malý princ}}    & Malý, blonďatý kudrnatý chlapec z planety B612                   \\
                                               & Vydal se do světa kvůli potížím s květinou a hledat smysl života \\
                                               & Postava působí nevinně, plný porozumění a zvídavých otázek       \\
    \hline
    \multirow{2}{15em}{\textbf{Pilot}}         & Byl jediný kdo Malému princi rozumněl                            \\
                                               & Chápal jeho dětský styl myšlení                                  \\
    \hline
    \textbf{Květina (Růže)}                    & Domýšlivá, marnivá, nafoukaná, z planety B612                    \\
    \hline
    \textbf{Liška}                             & Moudrá, přátelská, pozitivně ovliví jeho myšlení                 \\
    \hline
    \textbf{Had}                               & Jeho uštknutí vrátí prince zpět na jeho planetku                 \\
    \hline
    \multirow{2}{15em}{\textbf{Další postavy}} & Král, Domýšlivec, Pijan, Byznzsmen, Lampář, Zeměpisec            \\
                                               & Výhybkář, Obchodník
\end{tabularx}
\subsection*{Děj}
Pilot letadla nouzově přistává na Sahaře.
Když se svůj stroj snaží opravit, objeví se Malý princ (postava z cizí planetky, kterou opustil, protože pochyboval o lásce své růže, kterou miloval), který pilota žádá, aby mu nakreslil beránka.
Pilot kreslit neumí, nejprve namaluje hroznýše se slonem v žaludku, poté několik nepovedených kreseb beránka až nakonec beránka v krabici.
Malý princ je nadšený.

Poté pilotovi vypravuje o svých návštěvách na jiných planetách, kde potkával různé dospělé.
\begin{enumerate}
    \itemsep-0.5em
    \item Planetka: zde potkal krále, který všechny považuje za své poddané
    \item Planetka: Domýšlivec – chce být všemi uctíván a respektován, přestože zde žije zcela sám
    \item Planetka: Pijan, který pije, protože se stydí za to, že pije
    \item Planetka: Byznzsmen, který si myslí, že je všechno jeho a že vše lze koupit, počítá hvězdy
    \item Planetka: Lampář, který bez odpočinku zhasíná a rozsvěcí svou lampu kvůli rychle se střídajícímu dni a noci
    \item Planetka: Zeměpisec, který se neustále stará pouze o mapy
    \item Planetka = Země: zde potkává lišku, ta mu vysvětlí, že jediné správné vnímání světa, je vnímání srdcem.
          Na Zemi princ vidí lidskou uspěchanost.
          Mezitím se pilotovi podaří opravit své letadlo.
          Malý princ je smutný, jelikož jeho planeta je daleko, a tak se nechá uštknout hadem, aby jeho cesta zpět byla jednodušší.
          Po smrti se jeho duše vydává zpět na planetu k milované růži, za kterou se po nyní, po poznání skutečných citů, cítí zodpovědný.
\end{enumerate}

\subsection*{Autor}
\begin{tabularx}{\linewidth}{l|l}
    \textbf{Jméno:} Antoine de Saint-Exupéry (Marie Roger) & \textbf{Období:} Meziválečná literatura       \\
    \hline
    \textbf{Původ:} Francie, šlechtický původ              & \textbf{Zaměstnání:} Letec                    \\
    \hline
    \multicolumn{2}{l}{Francouzský letec - byl sestřelen druhé světové války}                              \\
    \multicolumn{2}{l}{Vytýkal lidem omezené a jednostranné vnímání světa}                                 \\
    \multicolumn{2}{l}{Pro svá díla čerpal z oblasti letectví}                                             \\
    \hline
    \multicolumn{2}{l}{Další díla:}                                                                        \\
    \multicolumn{2}{l}{\textbf{Kurýr na jih, Noční let, Letec}}                                            \\
    \hline
    \multicolumn{2}{l}{Podobní autoři:}                                                                    \\
    \multicolumn{2}{l}{\textbf{Meziválečná literatura:} Alexandr Dumas, Thomas Mann, Erich Maria Remarque} \\
    \multicolumn{2}{l}{\textbf{Francie:} Anatole France, Romain Rolland, Henri Barbusse}                   \\
\end{tabularx}

\section{O myších a lidech}
\label{sec:omysichalidech}
\subsection*{Základní informace}
\begin{tabularx}{\linewidth}{l|l}
  \textbf{Literární forma:} Próza (Novela s prvky balady\footnotemark[1]) & \textbf{Literární druh:} Epika                    \\
  \hline
  \textbf{Slohový postup:} Vyprávěcí                                      & \textbf{Typ vypravěče:} Er-forma, Autorský        \\
  \hline
  \textbf{Způsob vypravování:} Chronologické                              & \textbf{Žánr:} Realismus                          \\
  \hline
  \multicolumn{2}{l}{\textbf{Prostředí:} 30. léta 20. století v Kalifornii na jedné farmě blízko města Soledad}               \\
  \hline
  \multicolumn{2}{l}{\textbf{Jazykové prostředky:} Nespisovný jazyk, dialogy, slangy, hovorový jazyk, vulgarismy, konstrasty} \\
\end{tabularx}
\subsection*{Postavy}
\begin{tabularx}{\linewidth}{l|l}
  \multirow{3}{15em}{\textbf{Lennie Small}}  & Urostlý a velice silný chlap, pracovitý, působy jako hodný                    \\
                                             & Bohužel trpí mentální zaostalostí a to se mu stane osudným                    \\
                                             & Rád sahá na hebké věci, nedokáže zvládat sílu a emoce                         \\
  \hline
  \multirow{2}{15em}{\textbf{George Milton}} & Člověk, který se stará o Lennieho, velické hodný a obětový                    \\
                                             & Rád by žil spokojený život, ale není mu přáno                                 \\
  \hline
  \textbf{Candy}                             & Starý bezruký uklízeč na farmě, také touží po spokojeném životě               \\
  \hline
  \textbf{Curley}                            & Malý, žárlivý, zakomplexovaný syn majitele farmy, rád se pere                 \\
  \hline
  \textbf{Curleyho žena}                     & Mladá a přitažlivá žena, hledá společnost, ale lidé se ji kvůli muži vyhýbají \\
  \hline
  \textbf{Slim}                              & Respektovaný dělník farmy, spravedlivý a objektivní                           \\
  \hline
  \textbf{Cooks}                             & Jediný černoch mezi dělníky, hodný a inteligentní, ale je odstrkovaný         \\
  \hline
  \textbf{Další dělníci}                     & Whit, Carlson                                                                 \\
\end{tabularx}
\subsection*{Děj}
Lennieho se ujal kamarád George Milton.
Oba dva se vydávají na farmu, kde se seznamují s ostatními pracovníky.
Oba sní o tom, že si jednoho dne pořídí své vlastní hospodářství a budou chovat králíky s heboučkou srstí.

Curley, syn majitele farmy, je ženatý, ale jeho žena s ním není moc šťastná.
Proto často vyhledává společnost dělníků.
A tak se stalo, že jednoho dne navštívila i Lennieho.
Ten se jí svěřil, že má v oblibě hebké věci.
Dovolila mu tedy pohladit si její vlasy.
Lennie je však tiskl tak silně, až Curleyho žena začala strachy křičet.
Lennie, celý vyděšený, ji nechtěně zlomí vaz.

Uteče se schovat k řece, jak mu poradil George.
Curley zjistil, co Lennie udělal jeho ženě, a vydal se ho zabít.
Avšak George Lennieho vyhledá dříve, chce ho ušetřit od bolesti a krutého zacházení, a tak jej v okamžiku, kdy spolu mluví o svém snu, střelí do hlavy.
\subsection*{Autor}
\begin{tabularx}{\linewidth}{l|l}
  \textbf{Jméno:} John Steinbeck           & \textbf{Období:} 1902 - 1968                                                              \\
  \hline
  \textbf{Původ:} Salinas, Kalifornie, USA & \textbf{Zaměstnání:} Tesař, zeměměřič, úředník v obchodním domě, pomáhal na ranči         \\
  \hline
  \multicolumn{2}{l}{Začal psát historické romance, poté se přesunul na sociální problémy}                                             \\
  \multicolumn{2}{l}{Byl také oceánologem a zkousel biologické teorie života}                                                          \\
  \multicolumn{2}{l}{Vystudoval Stanfordovu univerzidu v Kalifornii, roku 1962 docal nobelovu cenu za literaturu}                      \\
  \hline
  \multicolumn{2}{l}{Další díla:}                                                                                                      \\
  \multicolumn{2}{l}{\textbf{Hrony zpěvu -} Impozatní román z období krize vyjadřující duch 30. let}                                   \\
  \multicolumn{2}{l}{\textbf{Na východ od ráje -} Nebiblický epos rodového bloudění a generačního střetu}                              \\
  \multicolumn{2}{l}{\textbf{Bitva -} O stávce, zkoumá nálady davu}                                                                    \\
  \hline
  \multicolumn{2}{l}{Podobní autoři:}                                                                                                  \\
  \multicolumn{2}{l}{\textbf{Realismus :} Honoré de Balzac, Ernest Hemingway, Trolstoj, Gogol}                                         \\
  \multicolumn{2}{l}{\textbf{Český Realismus:} Mrštíkové, Eliška Krásnohorská, Božena Němcová, Alois Jirásek, Karel Havlíček Borovský} \\
\end{tabularx}
\fancyfootnotetext{1}{Balada = neveselý děj a tragický konec příběhu}

\section{Bílá nemoc}
\label{sec:bilanemoc}
\subsection*{Základní informace}
\begin{tabularx}{\linewidth}{l|l}
  \textbf{Literární forma:} Drama (Protiválečné a protifašistické) & \textbf{Literární druh:} Tragédie                   \\
  \hline
  \textbf{Slohový postup:} Drama                                   & \textbf{Typ vypravěče:} Drama                       \\
  \hline
  \textbf{Způsob vypravování:} Drama                               & \textbf{Žánr:} Pragmatismus (Snaží se najít pravdu) \\
  \hline
  \multicolumn{2}{l}{\textbf{Prostředí:} Maršálova země (parodie na Německo), doba nacismu, před 2. sv. válkou}          \\
  \hline
  \multicolumn{2}{l}{\textbf{Jazykové prostředky:} Spisovný jazyk, dialogy, bohatá slovní zásoba, metafory, cizí jazyky} \\
\end{tabularx}
\subsection*{Postavy}
\begin{tabularx}{\linewidth}{l|l}
  \multirow{3}{15em}{\textbf{Doktor Galén}} & Mladý lékař, laskavý, pomáhá chudým, ale naivní                \\
                                            & Jako jediný objeví lék proti bílé nemoci                       \\
                                            & Pomocí ní vymáhá nastolení míru                                \\
  \hline
  \textbf{Maršál}                           & Velitel vojsk, diktátor, chce válku                            \\
  \hline
  \textbf{Baron Krüg}                       & Maršálův přítel, dodává zbraně                                 \\
  \hline
  \textbf{Dvorní rada Sigelius}             & Lékar, ředitel nemocnice, snaží se najít lék proti Bílé nemoci \\
  \hline
  \textbf{Další postavy}                    & Matka, otec, syn, dcera, malomocní, zdravotníci, \dots         \\
\end{tabularx}
\subsection*{Děj}
Blíže neurčená země začala trpět epidemií tzv. Bílé nemoci, která se projevuje malomocenstvím, bílými skvrnami na kůži.
Lidé se infikují dotykem.
Nejprve zaútočí jen na staré a chudé.
Zemi ovládá diktátor Maršál, který se společně s baronem připravuje na výbojnou válku s vedlejším menším státem.

Dr. Galénovi se mezitím podaří najít na lék nemoc.
Rozhodne se lék podávat jen chudým a starým, ne bohatým vlivným lidem.
Když se nakazí baron, doktor Galén mu dává podmínku, aby zastavil válku, jinak nedostane lék.
Baron tedy žádá Maršála, aby válku zastavil, ten však odmítá.
Baron ze zoufalství spáchá sebevraždu.
Pak se nakazí sám Maršál, doktor Galén mu dá stejnou podmínku.

Na naléhání své dcery nakonec Maršál souhlasí.
Když přichází Galén k Maršálovu paláci, střetává se skandujícím davem, který jej ušlape a s ním i lék.
Dav netuší, že právě pohřbil oslavovaného zachránce.
Vypuká válka.
\subsection*{Autor}
\begin{tabularx}{\linewidth}{l|l}
  \textbf{Jméno:} Karel Čapek     & \textbf{Období:} Meziválečná literatura (20. - 30. léta 20. st.)         \\
  \hline
  \textbf{Původ:} Česká republika & \textbf{Zaměstnání:} Prozaik, dramatik, básník, překladatel              \\
  \hline
  \multicolumn{2}{l}{Redaktor Národních listů a Lidových novin}                                              \\
  \multicolumn{2}{l}{Přítel T. G. Masaryka, mluvčí hradu}                                                    \\
  \multicolumn{2}{l}{Zemřel na zápal plic}                                                                   \\
  \hline
  \multicolumn{2}{l}{Další díla:}                                                                            \\
  \multicolumn{2}{l}{\textbf{Válka s mloky, R.U.R} - Utopistická díla}                                       \\
  \multicolumn{2}{l}{\textbf{Obyčejný život} - Filozofické dílo}                                             \\
  \multicolumn{2}{l}{\textbf{Dášenka čili život štěněte} - Pohádka}                                          \\
  \hline
  \multicolumn{2}{l}{Podobní autoři:}                                                                        \\
  \multicolumn{2}{l}{\textbf{Období :} Eduard Bass, Karel Poláček, Josef Čapek, Jiří Werich a Jiří Voskovec} \\
\end{tabularx}

\section{Hrdý Budžes}
\label{sec:hrdybudzes}
\subsection*{Základní informace}
\begin{tabularx}{\linewidth}{l|l}
  \textbf{Literární forma:} Próza (Groteskní román) & \textbf{Literární druh:} Epika                         \\
  \hline
  \textbf{Slohový postup:} Vyprávěcí                & \textbf{Typ vypravěče:} Osobní, ich-forma              \\
  \hline
  \textbf{Způsob vypravování:} Chronologické        & \textbf{Směr:} Postmodernismnus                        \\
  \hline
  \multicolumn{2}{l}{\textbf{Prostředí:} 70. léta 20. st., Ničín}                                            \\
  \hline
  \multicolumn{2}{l}{\textbf{Jazykové prostředky:} Vulgarismy, nespisovný jazyk, satira, přímá řec, dialekt} \\
\end{tabularx}
\subsection*{Postavy}
\begin{tabularx}{\linewidth}{l|l}
  \multirow{2}{15em}{\textbf{Helenka Součková}}        & Žákyně druhé třídy základní školy v Ničíně                \\
                                                       & Nemá kamarády, je tlustá, ale přátelská a hodná           \\
  Také: \textbf{Freinsteinová} nebo \textbf{Brďochová} & Chce zůstat silná jako její imaginární hrdina Hrdý Budžes \\
  \hline
  \textbf{Kačenka}                                     & Maminka Helenky, herečka, zatvrzelá vůči režimu           \\
  \hline
  \textbf{Pepa Brďoch}                                 & Otčín Helenky, podobné rysy jako Kačenka                  \\
  \hline
  \textbf{Karel Freinstein}                            & Otec Helenky, žije v New Yorku, posílá Helence dárky      \\
  \hline
  \textbf{Pepíček}                                     & Nevlastní bratr Helenky                                   \\
  \hline
  \textbf{Babička}                                     & Helenka ji má na konci ráda, trvdohlavá a přísná          \\
  \hline
  \textbf{Dědeček}                                     & Helenka ho má radši než babičku, laskavý a ochotný        \\
\end{tabularx}
\subsection*{Děj}
Helenka chodí do druhé třídy. Je jiná než ostatní.
Spolužáci se jí smějí za to, že je tlustá, že její rodiče jsou herci a za to, že její příjmení je Freinsteinová, i když si říká Součková.
Chce zůstat silná a vytrvat jako její vzor Hrdý Budžes, o němž slyšela v básničce.
Chodí na němčinu k paní Freimanové, na sochání k panu Peckovi a nakonec jí rodiče dovolili chodit i na Jiskřičky, přestože si myslí, že jsou to mladí komunisté.
Helenka ráda kreslí, má bujnou fantazii, miluje Prahu a Milušku Voborníkovou.
O víkendech jezdí za babičkou a dědečkem do Zákopů.
Babička Kačenku stále pomlouvá a tajně píše Freinsteinovi o tom, jak se Helenka bez něho trápí, což není pravda.
Kačenka s Pepou se nechtěli stát komunisty, a proto pomalu v divadle přicházeli o role, až Kačenku vyhodili úplně.

Helenka sní o tom, že se s rodinou přestěhuje do Prahy.
Tento sen se jí splní, když Kačenka získá v Praze byt po tetě.
V závěru knihy zjišťuje, že žádný Hrdý Budžes neexistuje, že jde o splynutí slov Hrdý buď, žes…
Od Freinsteina dostala velké balení barevných fixek, které si původně moc přála, ale nahází jednu po druhé do kanálu.

\subsection*{Autor}
\begin{tabularx}{\linewidth}{l|l}
  \textbf{Jméno:} Irena Dousková (Irena Freidstatová) & \textbf{Období:} Druhá pol. 20. st.                            \\
  \hline
  \textbf{Původ:} Příbram, Česká republika            & \textbf{Zaměstnání:} Spisovatelska, prozaička, dříve novinářka \\
  \hline
  \multicolumn{2}{l}{Vystudovala práva, ale nikdy je nedělala}                                                         \\
  \multicolumn{2}{l}{Otec emigorval do Izraele, rodina se přestěhovala do Prahy}                                       \\
  \multicolumn{2}{l}{Rodice jejího otce nepřežila 2. světovou válku}                                                   \\
  \hline
  \multicolumn{2}{l}{Další díla:}                                                                                      \\
  \multicolumn{2}{l}{\textbf{Oněgin byl Rusák -} Pokračování Hrdého Budžese}                                           \\
  \multicolumn{2}{l}{\textbf{Někdo s nožem}}                                                                           \\
  \multicolumn{2}{l}{\textbf{Golstein píše dceři} - Pásmo dopisů}                                                      \\
  \hline
  \multicolumn{2}{l}{Podobní autoři:}                                                                                  \\
  \multicolumn{2}{l}{\textbf{Období:} Michal Viewegh, Ludvík Vaculík}                                                  \\
  \multicolumn{2}{l}{\textbf{Podobné dílo:} Bylo nás pět - Karel Poláček}                                              \\
\end{tabularx}

\section{Spalovač mrtvol}
\label{sec:spalovacmrtvol}
\subsection*{Základní informace}
\begin{tabularx}{\linewidth}{l|l}
  \textbf{Literární forma:} Próza (Psychologická novela\footnotemark[1]) & \textbf{Literární druh:} Epika             \\
  \hline
  \textbf{Slohový postup:} Vyprávěcí                                     & \textbf{Typ vypravěče:} Er-forma, Autorský \\
  \hline
  \textbf{Způsob vypravování:} Chronologické                             & \textbf{Žánr:} Druhá vlna válečné prózy    \\
  \hline
  \multicolumn{2}{l}{\textbf{Prostředí:} Praha v období protektorátu (1937)}                                          \\
  \hline
  \multicolumn{2}{l}{\textbf{Jazykové prostředky:} Spisovné, děsivé, složité metafory a symboliky, oslovení}          \\
\end{tabularx}
\subsection*{Postavy}
\begin{tabularx}{\linewidth}{l|l}
  \multirow{3}{15em}{\textbf{Karel Kopfrkingl}} & Nechává si říkat Roman, protože má rád romantiku                  \\
                                                & Nejprve je zásadový, ale jeho názory se pod vlivem Willhema mění  \\
                                                & Býval zaměstananec krematoria, ale poté se měni v udavače a vraha \\
  \hline
  \textbf{Willhelm Reinke}                      & Hrdý Němec přesvědčuje Karla k přechodu na německou stranu        \\
\end{tabularx}
\subsection*{Děj}
Pan Kopfrkingl se zpočátku profiluje jako mírný a hodný člověk, který žije jen pro svou rodinu.
Zdá se i romanticky založený. Vzpomíná na okamžik, kdy se se svou ženou seznámil před leopardí klecí, a také označuje své rodinné příslušníky jmény jako „čarokrásná, nebeská, nadoblačná apod“.

Ideální rodinná atmosféra je ovšem neustále narušována K prací, o které neustále všem vypráví - dělá spalovače mrtvol v pražském krematoriu.
Rád se obklopuje věcmi ze svého pracoviště, v kuchyni má například kremační tabulku, které přezdívá „jízdní řád smrti“, v jeho knihovně nechybí kremační zákon a také kniha o Tibetu.

Svébytný svět protagonisty narušuje návštěva dávného přítele, nyní přesvědčeného nacisty, Williho Reinkeho.
Ten přiměje K uvěřit v rasovou nadřazenost („vzpomeňte si na kapku německé krve“) a nutnost zbavit se pojítek s bytostmi nižšími a méněcennými.
Nakonec nátlaku podléhá a začíná udávat lidi kolem sebe, čímž si zajistí cestu do nacistického Casina a je také povýšen na ředitele krematoria.
Manželčin židovský původ mu brání v kariéře, proto postupně zavraždí nejen ji (Co abych tě drahá oběsil?), ale i svého změkčilého syna Miliho, kterého obzvlášť krutě utluče v kremační místnosti.
Na základě pokřivené interpretace tibetské filozofie hodnotí páchané zločiny jako pomoc nešťastníkům, kteří nechápou vyšší zájmy světa.

Když se pokusí zabít i svou osmnáctiletou dceru, přeruší jej jeho dvojník - výplod choré mysli, tibetský vyslanec, nazve ho inkarnací Buddhy a vyzve ho, aby se ujal trůnu ve Lhase, neboť dalajláma zemřel a mniši již 19 let hledají jeho nástupce.
Prožitek osvícení završují tři „andělé“, kteří odvádějí Kopfrkingla do sanitky a odváží do blázince.
V závěru novely je protagonista konfrontován s průvodem vyhublých lidí, kteří se vracejí po skončení války z koncentračního tábora.
Ani tehdy se nevzdává přesvědčení, že je mesiášem.
Kniha končí K slovy: „Šťastné lidstvo. Spasil jsem je.“
\subsection*{Autor}
\begin{tabularx}{\linewidth}{l|l}
  \textbf{Jméno:} Ladislav Fuks          & \textbf{Období:} 1923 - 1994            \\
  \hline
  \textbf{Původ:} Praha, Česká republika & \textbf{Zaměstnání:} Prozaik            \\
  \hline
  \multicolumn{2}{l}{Měl problém při válce kvůli jeho homosexuální orientaci}      \\
  \multicolumn{2}{l}{Oženil se s bohatou italkou, od které krátce po svatbě utekl} \\
  \multicolumn{2}{l}{Byl hospitalizován na psychiatrii}                            \\
  \hline
  \multicolumn{2}{l}{Další díla:}                                                  \\
  \multicolumn{2}{l}{\textbf{Mí černovlastí bratři}}                               \\
  \multicolumn{2}{l}{\textbf{Pan Theodor Munstock}}                                \\
  \multicolumn{2}{l}{\textbf{Zámek Kynžvart}}                                      \\
  \hline
  \multicolumn{2}{l}{Podobní autoři:}                                              \\
  \multicolumn{2}{l}{\textbf{Období:} Ota Pavel, Bohumil Hrabal, Karel Čapek}      \\
\end{tabularx}
\fancyfootnotetext{1}{Novela je narozdíl od románu kratší a rozvíjí pouze jednu dějovou linii.}

\section{Saturnin}
\subsection*{Základní informace}
\begin{tabularx}{\linewidth}{l|l}
    \textbf{Literární forma:} Próza (Humoristický román)                         & \textbf{Literární druh:} Epika            \\
    \hline
    \textbf{Slohový postup:} Vyprávěcí a popisný                                 & \textbf{Typ vypravěče:} Osobní, ich-forma \\
    \hline
    \textbf{Způsob vypravování:} Chronologické, ale byla použita i retrospektiva & \textbf{Směr:} Žánr                       \\
    \hline
    \multicolumn{2}{l}{\textbf{Prostředí:} Praha a venkov, 20. léta 20. st.}                                                 \\
    \hline
    \multicolumn{2}{l}{\textbf{Jazykové prostředky:} Spisovný jazyk, satira, archaismy, vulgarismy, přísloví}                \\
\end{tabularx}
\subsection*{Postavy}
\begin{tabularx}{\linewidth}{l|l}
    \textbf{Vypravěč}                & Měšťanský typ, gentleman bez většího smyslu pro humor, zamilovaný do Barbory      \\
    \hline
    \textbf{Saturnin}                & Vypravěčův sluha, velký smysl pro humor, řídí většinu děje                        \\
    \hline
    \textbf{Dědeček}                 & Starý bohatý pán, ředitel elektrické elektrárny, všichni se snaží dostat dědictví \\
    \hline
    \textbf{Teta Kateřina}           & Je hamižná a hrabivá, používá neustále přísloví                                   \\
    \hline
    \textbf{Milouš}                  & Syn Kateřiny, ta s ním i v 18 letech zachází jako dítě                            \\
    \hline
    \textbf{Slečna Barbora Terebová} & Moderní, milá žena                                                                \\
    \hline
    \textbf{Doktor Vlach}            & Rodinný přítel, vymýšlí teorie o chování lidí                                     \\
\end{tabularx}
\subsection*{Děj}
Mladý muž přijme Saturnina jako sluhu.
Tím se jeho svět otočí naruby.
Z klidného bytu se přestěhují na hausbót, kde čelí každý den vlnobití a projíždějícím parníkům.
Pak se rovněž na popud Saturnina vydá tento mladý muž chytit lva, který utekl ze zoo.
Když přijede na návštěvu teta Kateřina s Miloušem, namluví jim Saturnin, že jsou na lodi myši, aby se dotěrných příbuzných zbavili.

Poté se rozhodnou navštívit dědečka mladého pána, který žije na venkově.
Zde se mladý muž seznamuje s Barborou, do které se brzy zamiluje.
K dědečkovi však přijede i doktor Vlach a teta Kateřina s Milošem.
Zažívají spolu řadu komických situací, nicméně nakonec všechno dobře dopadne.

Mladý muž se sblíží s Barborou a odstěhují se spolu do Prahy, teta se znova bohatě provdá a Saturnin se stará o dědečka.

\subsection*{Autor}
\begin{tabularx}{\linewidth}{l|l}
    \textbf{Jméno:} Zdeněk Jirotka  & \textbf{Období:} 1911-2003 (První pol. 20. st.)                          \\
    \hline
    \textbf{Původ:} Česká republika & \textbf{Zaměstnání:} Spisovatel, fejetonista, autor románů, povídek\dots \\
    \hline
    \multicolumn{2}{l}{Vystudoval stavební průmyslovou školu v Ostravě}                                        \\
    \multicolumn{2}{l}{Nikdy nepřekonal svůj první román Saturnin}                                             \\
    \multicolumn{2}{l}{Pracoval v Lidových novinách, Svobodných novinách\dots}                                 \\
    \hline
    \multicolumn{2}{l}{Další díla:}                                                                            \\
    \multicolumn{2}{l}{\textbf{Muž se psem -} parodie na detektivní romány}                                    \\
    \multicolumn{2}{l}{\textbf{Sedmilháři, Pravda se změnila -} sbírky povídek}                                \\
    \multicolumn{2}{l}{\textbf{Hvězdy nad starým Vavrouchem -} rozhlasová hra}                                 \\
    \hline
    \multicolumn{2}{l}{Podobní autoři:}                                                                        \\
    \multicolumn{2}{l}{\textbf{Čeští:} Karel Poláček, Eduard Bass, Vladislav Vančura}                          \\
\end{tabularx}

\section{Edison}
    \subsection*{Základní informace}
        \begin{center}
            \begin{tabular}{l|l}
                \textbf{Literární forma:} Forma & \textbf{Literární druh:} Druh \\
                \hline
                \textbf{Slohový postup:} Postup & \textbf{Typ vypravěče:} Typ \\
                \hline
                \textbf{Způsob vypravování:} Způsob & \textbf{Žánr:} Žánr \\
                \hline
                \multicolumn{2}{l}{\textbf{Prostředí:} Prostředí} \\
                \hline
                \multicolumn{2}{l}{\textbf{Jazykové prostředky:} Prostředky} \\
            \end{tabular}
        \end{center}
    \subsection*{Postavy}
        \begin{center}
            \begin{tabular}{l|l}
                \multirow{3}{15em}{\textbf{Obsáhlá postava}} & Popis postavy \\
                & Popis postavy \\
                & Popis postavy \\
                \hline
                \textbf{Jednoduchá postava} & Popis postavy \\
            \end{tabular}
        \end{center}
    \subsection*{Děj}
        První odstavec děje

        Druhý odstavec
        
        Třetí odstavec
    \subsection*{Autor}
        \begin{center}
            \begin{tabular}{l|l}
                \textbf{Jméno:} Jméno & \textbf{Období:} Období\\
                \hline
                \textbf{Původ:} Původ & \textbf{Zaměstnání:} Zaměstnání\\
                \hline
                \multicolumn{2}{l}{Povídání o autorovi} \\
                \multicolumn{2}{l}{Povídání o autorovi} \\
                \multicolumn{2}{l}{Povídání o autorovi} \\
                \hline
                \multicolumn{2}{l}{Další díla:} \\
                \multicolumn{2}{l}{\textbf{Dílo 1 -} Popis díla 1} \\
                \multicolumn{2}{l}{\textbf{Dílo 2 -} Popis díla 2} \\
                \multicolumn{2}{l}{\textbf{Dílo 3 -} Popis díla 3} \\
                \hline
                \multicolumn{2}{l}{Podobní autoři:} \\
                \multicolumn{2}{l}{\textbf{Směr/období 1:} Autor 1, autor 2, autor 3} \\
                \multicolumn{2}{l}{\textbf{Směr/období 2:} Autor 1, autor 2, autor 3} \\
                \multicolumn{2}{l}{\textbf{Směr/období 3:} Autor 1, autor 2, autor 3} \\
            \end{tabular}
        \end{center}
\section{Hospoda na mýtince}
\label{sec:hospodanamytince}
\subsection*{Základní informace}
\begin{tabularx}{\linewidth}{l|l}
  \textbf{Literární forma:} Drama (Opereta (Hra se zpěvy)) & \textbf{Literární druh:} Drama - Komedie                \\
  \hline
  \textbf{Slohový postup:} Drama                           & \textbf{Typ vypravěče:} Drama                           \\
  \hline
  \textbf{Způsob vypravování:} Drama                       & \textbf{Žánr:} Poetismus                                \\
  \hline
  \multicolumn{2}{l}{\textbf{Prostředí:} Konec 19. st., Hospoda uprostřed nespecifikovaného českého lesa}            \\
  \hline
  \multicolumn{2}{l}{\textbf{Jazykové prostředky:} V předehře jazyk spisovný a odborný, ve hře hovorový, dvojsmysly} \\
\end{tabularx}
\subsection*{Postavy}
\begin{tabularx}{\linewidth}{l|l}
  \textbf{Hostinský}               & Vlastní hospodu na mýtince                          \\
  \hline
  \textbf{Hrabě Ferdinand}         & Havaroval se vzducholodí, pašerák                   \\
  \hline
  \textbf{Vězeň Kulhánek}          & 20 let strávil ve věžení za pašeráctví, co neudělal \\
  \hline
  \textbf{Inspektor Trachta}       & Sedí u hospody 15 let                               \\
  \hline
  \textbf{Fiktivní vnučka Růženka} & Zamilují se do ní hrabě i vězeň                     \\
\end{tabularx}
\subsection*{Děj}
Děj je rozdělen na dvě části.

V první části děje se mluví o osobnosti Cimrmana.
O tom že neprošel pubertou, že pořídil klavírní výtah do divadla a o teorii absolutního rýmu.
Poté se zde mluví o samotné tvorbě operety "Proso", kterou Cimrmal napsal při opravě ztroskotané vzducholodi.
Samotné dílo poté Cimrman poslal do Vídně do soutěže operet, kde dílo bylo zcizeno.
Útržky originálu poté byly nesmazatelně zachyceny na nahrávkách.

V druhé části se odehrává samotný děj operety.
Hra začíná vysvětlením jak přišel pan honstinský k hodpodě uprostřed lesa.
Jednoho dne ztroskotá vzducholoď hraběte Ferdinanda u hospody.
Hostinský využívá této příležitiosti pohádkou o smyšlené dčeři a donutí hrabětě měsíc pobývat v hostinci.
Po měsící se na scénu dostává uprchlý věžeň, kterého měli další den po 20 letech pustit.
Poté nastávají boje o vymyšlenou dceru mezi hrabětem a vězněm.

Příběh končí odhalením, že důvod proč vězeň byl ve vězení je pan hrabě, který je následně odveden ze scény.
\subsection*{Autor}
\begin{tabularx}{\linewidth}{l|l}
  \textbf{Jméno:} Zdeněk Svěrák, Ladislav Smoljak & \textbf{Období:} 2. pol. 20. st.                                                      \\
  \hline
  \textbf{Původ:} Praha, Česká republika          & \textbf{Zaměstnání:} Scénaristé, herci, režiséři, dramatici                           \\
  \hline
  \multicolumn{2}{l}{Svěrák: Jeho film Kolja vyhrál Oscara, spolupracoval s Jaroslvem Uhlířem na písničkách}                              \\
  \multicolumn{2}{l}{Smoljak: Pedagog matematiky a fyziky}                                                                                \\
  \multicolumn{2}{l}{Divadlo Járy Cimrmana: Inteligentní humor, hrají pouze muži, }                                                       \\
  \hline
  \multicolumn{2}{l}{Další díla:}                                                                                                         \\
  \multicolumn{2}{l}{\textbf{Svěrák:} Kolja, Marečku, podejte mi pero, Obecná škola, Tři Bratři}                                          \\
  \multicolumn{2}{l}{\textbf{Smoljak:} Vrchní prchni! Jáchyme hoď ho do stroje! Kulový blesk (Většinou spoluautor)}                       \\
  \multicolumn{2}{l}{\textbf{Divadlo Járy Cimrmana:} Vyšetřování ztráty třídní knihy, Vražda v salóním kupé, Dlouhý Široký a Krátkozraký} \\
  \hline
  \multicolumn{2}{l}{Podobní autoři:}                                                                                                     \\
  \multicolumn{2}{l}{\textbf{Divadla té doby:} Divadlo Na zábradlí, Divadlo na provázku, Semafor (SEdm MAlých FORem)}                     \\
  \multicolumn{2}{l}{\textbf{Podobní autoři:} Suchý a Šlitr, Bolek Polívka, F. Hrubín}                                                    \\
\end{tabularx}


\pagestyle{empty}
\section*{Tropy}
\begin{description}
    \item[Metafora] - vnější podobnost (hlad je nejlepší kuchař)
    \item[Metonymie] - vnitřní podobnost (město čeká)
    \item[Personifikace] - zosobnění (stromy šeptaly)
    \item[Přirovnání] (chová se jako vůl)
    \item[Epiteton] - básnický přívlastek (nejčokoládovější čokoláda)
    \item[Hyperbola] - nadsázka
    \item[Eufemismus] - zjemnění nepříjemné skutečnosti (zesnout)
    \item[Dysfemismus] - zhrubění skutečnosti (pazoura)
    \item[Oxymóron] - nelogické spojení slov (mrtvé milenky cit)
    \item[Synekdocha] - záměna části za celek (přišel o střechu nad hlavou)
    \item[Ironie] 
\end{description}

\end{document}
