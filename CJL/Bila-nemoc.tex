\section{Bílá nemoc}
\label{sec:bilanemoc}
\subsection*{Základní informace}
\begin{tabularx}{\linewidth}{l|l}
  \textbf{Literární forma:} Drama (Protiválečné a protifašistické) & \textbf{Literární druh:} Tragédie                   \\
  \hline
  \textbf{Slohový postup:} Drama                                   & \textbf{Typ vypravěče:} Drama                       \\
  \hline
  \textbf{Způsob vypravování:} Drama                               & \textbf{Žánr:} Pragmatismus (Snaží se najít pravdu) \\
  \hline
  \multicolumn{2}{l}{\textbf{Prostředí:} Maršálova země (parodie na Německo), doba nacismu, před 2. sv. válkou}          \\
  \hline
  \multicolumn{2}{l}{\textbf{Jazykové prostředky:} Spisovný jazyk, dialogy, bohatá slovní zásoba, metafory, cizí jazyky} \\
\end{tabularx}
\subsection*{Postavy}
\begin{tabularx}{\linewidth}{l|l}
  \multirow{3}{15em}{\textbf{Doktor Galén}} & Mladý lékař, laskavý, pomáhá chudým, ale naivní                \\
                                            & Jako jediný objeví lék proti bílé nemoci                       \\
                                            & Pomocí ní vymáhá nastolení míru                                \\
  \hline
  \textbf{Maršál}                           & Velitel vojsk, diktátor, chce válku                            \\
  \hline
  \textbf{Baron Krüg}                       & Maršálův přítel, dodává zbraně                                 \\
  \hline
  \textbf{Dvorní rada Sigelius}             & Lékar, ředitel nemocnice, snaží se najít lék proti Bílé nemoci \\
  \hline
  \textbf{Další postavy}                    & Matka, otec, syn, dcera, malomocní, zdravotníci, \dots         \\
\end{tabularx}
\subsection*{Děj}
Blíže neurčená země začala trpět epidemií tzv. Bílé nemoci, která se projevuje malomocenstvím, bílými skvrnami na kůži.
Lidé se infikují dotykem.
Nejprve zaútočí jen na staré a chudé.
Zemi ovládá diktátor Maršál, který se společně s baronem připravuje na výbojnou válku s vedlejším menším státem.

Dr. Galénovi se mezitím podaří najít na lék nemoc.
Rozhodne se lék podávat jen chudým a starým, ne bohatým vlivným lidem.
Když se nakazí baron, doktor Galén mu dává podmínku, aby zastavil válku, jinak nedostane lék.
Baron tedy žádá Maršála, aby válku zastavil, ten však odmítá.
Baron ze zoufalství spáchá sebevraždu.
Pak se nakazí sám Maršál, doktor Galén mu dá stejnou podmínku.

Na naléhání své dcery nakonec Maršál souhlasí.
Když přichází Galén k Maršálovu paláci, střetává se skandujícím davem, který jej ušlape a s ním i lék.
Dav netuší, že právě pohřbil oslavovaného zachránce.
Vypuká válka.
\subsection*{Autor}
\begin{tabularx}{\linewidth}{l|l}
  \textbf{Jméno:} Karel Čapek     & \textbf{Období:} Meziválečná literatura (20. - 30. léta 20. st.)         \\
  \hline
  \textbf{Původ:} Česká republika & \textbf{Zaměstnání:} Prozaik, dramatik, básník, překladatel              \\
  \hline
  \multicolumn{2}{l}{Redaktor Národních listů a Lidových novin}                                              \\
  \multicolumn{2}{l}{Přítel T. G. Masaryka, mluvčí hradu}                                                    \\
  \multicolumn{2}{l}{Zemřel na zápal plic}                                                                   \\
  \hline
  \multicolumn{2}{l}{Další díla:}                                                                            \\
  \multicolumn{2}{l}{\textbf{Válka s mloky, R.U.R} - Utopistická díla}                                       \\
  \multicolumn{2}{l}{\textbf{Obyčejný život} - Filozofické dílo}                                             \\
  \multicolumn{2}{l}{\textbf{Dášenka čili život štěněte} - Pohádka}                                          \\
  \hline
  \multicolumn{2}{l}{Podobní autoři:}                                                                        \\
  \multicolumn{2}{l}{\textbf{Období :} Eduard Bass, Karel Poláček, Josef Čapek, Jiří Werich a Jiří Voskovec} \\
\end{tabularx}
