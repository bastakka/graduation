\section{Edison}
\label{sec:edison}
\subsection*{Základní informace}
\begin{tabularx}{\linewidth}{l|l}
    \textbf{Literární forma:} Poezie (skladba - polytematická báseň)  & \textbf{Literární druh:} Lyricko-epický                  \\
    \hline
    \textbf{Slohový postup:} Popisný                    & \textbf{Typ vypravěče:} Lyrický subjekt                  \\
    \hline
    \textbf{Způsob vypravování:} Chronologické, 5 částí & \textbf{Žánr:} Poetismus                                 \\
    \hline
    \multicolumn{2}{l}{\textbf{Prostředí:} 1847-1931 Praha a Amerika}                                              \\
    \hline
    \multicolumn{2}{l}{\textbf{Jazykové prostředky:} Sdružený rým, personifikace, anafory, metafory, polytematika} \\
\end{tabularx}
\subsection*{Postavy}
\begin{tabularx}{\linewidth}{l|l}
    \multirow{3}{15em}{\textbf{Básík}} & Lyrický subjekt                                          \\
                                       & Vypráví a komentuje příběhy                              \\
                                       & Charakterizuje a cituje hlavní postavu příběhu           \\
    \hline
    \textbf{Sebevrah}                  & Protiklad Edisona, hazardér, neúspěšný, má prázdný život \\
    \hline
    \textbf{Edison}                    & Americký technik, vynálezce a podnikatel                 \\
\end{tabularx}
\subsection*{Děj}
První zpěv vypráví, jak autor večer cestou domů potkává hazardního hráče, který se chce zabít.
Dojdou spolu domů, ale tam už je autor sám.
Připadalo mi, že autor a sebevrah je jeden a ten samý člověk, který se dívá sám na sebe, na své druhé já.

Druhý zpěv již vypráví o samotném Edisonovi, o jeho mládí.
Jak roznášel noviny, jak zachránil život malému chlapci a jak odjel do New Yorku.

Třetí zpěv popisuje Edisonovy objevy, hlavně vynález žárovky.

Ve čtvrtém zpěvu skladba vrcholí, motiv uznání Edisonových vynálezů.
Smyslem života je štěstí z nových objevů a dobývání světa.

Radost ze života, touha po životě, básník se probouzí ze snění a uvědomí si, že mluvil sám se sebou
\subsection*{Autor}
\begin{tabularx}{\linewidth}{l|l}
    \textbf{Jméno:} Vítěslav Nezval & \textbf{Období:} 1900-1958 (První pol. 20. st.)             \\
    \hline
    \textbf{Původ:} Česká republika & \textbf{Zaměstnání:} Básník, dramatik, prozaik, překladatel \\
    \hline
    \multicolumn{2}{l}{Vstoupil do KSČ a publikoval v Rudém právu, získal pocty a funkce}         \\
    \multicolumn{2}{l}{V životě ho hodně ovlivnili jeho cesty do zahraničí}                       \\
    \multicolumn{2}{l}{Angažoval se ve skupinách surrealistů a poetistů}                          \\
    \hline
    \multicolumn{2}{l}{Další díla:}                                                               \\
    \multicolumn{2}{l}{\textbf{Manon Lescaut}}                                                    \\
    \multicolumn{2}{l}{\textbf{Anička, skřítek a Slaměný Hubert}}                                 \\
    \multicolumn{2}{l}{\textbf{Valérie a týden divů}}                                             \\
    \hline
    \multicolumn{2}{l}{Podobní autoři:}                                                           \\
    \multicolumn{2}{l}{\textbf{Čeští:} Josef Hora, Jindřích Hořejší}                              \\
    \multicolumn{2}{l}{\textbf{Poetismus:} František Halas, Jaroslav Seifert}                     \\
\end{tabularx}

