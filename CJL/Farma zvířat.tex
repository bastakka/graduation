\section{Farma zvířat}
\subsection*{Základní informace}
\begin{tabularx}{\linewidth}{l|l}
    \textbf{Literární forma:} Próza (Antiutopická bajka, aleg. novela) & \textbf{Literární druh:} Epika             \\
    \hline
    \textbf{Slohový postup:} Vyprávěcí a popisný                       & \textbf{Typ vypravěče:} Autorský, er-forma \\
    \hline
    \textbf{Způsob vypravování:} Chronologické                         & \textbf{Žánr:} Neorealismus                \\
    \hline
    \multicolumn{2}{l}{\textbf{Prostředí:} Anglický venkov v 50. letech 20. století}                                \\
    \hline
    \multicolumn{2}{l}{\textbf{Jazykové prostředky:} Spisovný jazyk, metafory, archaismy, personifikace}            \\
\end{tabularx}
\subsection*{Postavy}
\begin{tabularx}{\linewidth}{l|l}
    \textbf{Pan Jones} & Původní majitel farmy, alkoholik              \\
    \hline
    \textbf{Napoleon}  & Prase, nejchytřejší, ale podvodník a pokrytec \\
    \hline
    \textbf{Pištík}    & Prase, lhář, přítel Napoleona                 \\
    \hline
    \textbf{Kuliš}     & Prase, chytré s dobrými nápady, čestné        \\
    \hline
    \textbf{Boxer}     & Kůň, pracovitý, čestný, dříč                  \\
    \hline
    \textbf{Benjamin}  & Osel, pasivní                                 \\
    \hline
    \textbf{Major}     & Inteligentní kanec                            \\
    \hline
    \textbf{Molina}    & Klisna, „madam“ mezi zvířaty, ráda se parádí  \\
    \hline
    \textbf{Lidé}      & Kapitalisti, vykořisťovatelé                  \\
\end{tabularx}
\subsection*{Děj}
Dílo vypráví o farmě, kde zvířata trpěla hladem.
Pan Jones propíjel peníze a svůj čas trávil v hospodě - nestaral se o ně dobře a často neměla dost jídla.
Začne je štvát, že musí sloužit lidem, chtějí pracovat jen sami pro sebe. Vzbudí se revoluce, která je úspěšná.
Po vyhnání lidí se farma přejmenuje na „Zvířecí farmu“ a jsou ustanovena pravidla (Sedm přikázání), která nesmí být překročena.

Zvířata začnou znovu pracovat, ale díky pocitu svobody tentokrát mnohem výkonněji.
V čele jsou dvě prasata (Kuliš a Napoleon), která se pořád hádají. Když jedno podá návrh, druhé ho zamítne, v lepším případě podá jiný návrh.
Napoleon vyhnal chudáka Kuliše a poslal na něj divoké psy, které si sám vychoval.
Postupně vraždí i ostatní zvířata. Napovídal jim totiž, že za tu dobu zapomněla Sedm přikázání.

Nejpracovitější zvířátko je Boxer, který se ale předře a je odvezen na jatka.
Napoleon si vychoval další prasata k obrazu svému a ty začínají vládnout celé farmě.
Začnou spolupracovat zpátky s lidmi a přejmenují farmu zpět na Panskou. S lidmi popíjejí a hrají karty.
Ostatní zvířata je od sebe nedokážou rozeznat. Prasata totiž vypadají jako lidé.
\subsection*{Autor}
\begin{tabularx}{\linewidth}{l|l}
    \textbf{Jméno:} George Orwell (Eric Arthur Blair) & \textbf{Období:} 1903-1950 (První pol. 20. st.)     \\
    \hline
    \textbf{Původ:} Anglie (Narodil se v Indii)       & \textbf{Zaměstnání:} Novinář, esejista a spisovatel \\
    \hline
    \multicolumn{2}{l}{Zapojil se do španělské občanské války}                                              \\
    \multicolumn{2}{l}{Jeho knihy nemohli v Československu oficiálně vycházet}                              \\
    \multicolumn{2}{l}{Považoval se za socialistu}                                                          \\
    \hline
    \multicolumn{2}{l}{Další díla:}                                                                         \\
    \multicolumn{2}{l}{\textbf{1984}}                                                                       \\
    \multicolumn{2}{l}{\textbf{Na dně v Paříži a Londýně}}                                                  \\
    \multicolumn{2}{l}{\textbf{Nadechnout se}}                                                              \\
    \hline
    \multicolumn{2}{l}{Podobní autoři:}                                                                     \\
    \multicolumn{2}{l}{\textbf{Sci-fi 20. st.:} Isaac Asimov, Arthur C. Clarke, Ray Bradbury}               \\
    \multicolumn{2}{l}{\textbf{Čeští sci-fi 20. st.:} Karel Čapek, Ludvík Souček}                           \\
\end{tabularx}
