\section{Hobit}
\label{sec:hobit}
\subsection*{Základní informace}
\begin{tabularx}{\linewidth}{l|l}
    \textbf{Literární forma:} Próza (Fantasy román) & \textbf{Literární druh:} Druh                                        \\
    \hline
    \textbf{Slohový postup:} Vyprávěcí a popisný    & \textbf{Typ vypravěče:} Er-forma, autorský                           \\
    \hline
    \textbf{Způsob vypravování:} Chronologický      & \textbf{Žánr:} Fantasy                                               \\
    \hline
    \multicolumn{2}{l}{\textbf{Prostředí:} Fiktivní oblast Středozemě ve fiktivním čase}                                   \\
    \hline
    \multicolumn{2}{l}{\textbf{Jazykové prostředky:} Spisovná čeština, knižní výrazy, zdrobněliny, dialogy, vlastní jazyk} \\
\end{tabularx}
\subsection*{Postavy}
\begin{tabularx}{\linewidth}{l|l}
    \multirow{3}{15em}{\textbf{Bilbo Pytlík}} & Hobit, zpočátku líný, má rád klid, později dobrodruh a hrdina             \\
                                              & Po návratu domů ztrácí obdiv ostatních hobitů, ale on je spokojený        \\
                                              & Hobit = Bytost menší jak lidé, nenosí boty                                \\
    \hline
    \multirow{2}{15em}{\textbf{Gandalf}}      & Kouzelník vysoké postavy, špičatý klobouk, dlouhý bílý vous a neznámý věk \\
                                              & Je laskavý, pomáhá všem, veselý a rád se směje, začal dobrodružství       \\
    \hline
    \textbf{Thorin Pavéza}                    & "Král pod horou", Král trpaslíků, odhodlaný, miluji bohatství             \\
    \hline
    \textbf{Medděd}                           & Napůl člověk, napůl medvěd, mocný a silný                                 \\
    \hline
    \textbf{Glum}                             & Schizofrenní stvoření žijící v jeskyních, šišlající, miluje prsten        \\
    \hline
    \textbf{Bard}                             & Statečný, čestný elf, který zabije Šmaka                                  \\
    \hline
    \textbf{Trpaslíci}                        & Dvalin, Balin, Kili, Fili, Dori, Nori, Ori, Goin, Bifur, Bofur, Bombur    \\
\end{tabularx}
\subsection*{Děj}
Bilbo si žije svůj poklidný život ve své hobití noře.
Jednoho dne ale potká Gandalfa, který hledá někoho, kdo by se zúčastnil dobrodružství (vydobýt zpět království trpaslíků).
Bilbo ho „vydobryjitruje“, ale Gandalf si už Bilba vybral.
Pozve k němu 13 trpaslíků, Bilbo prvně odmítne, ale pak si to rozmyslí a cesta začíná…

Cestou je potká mnoho dobrodružství: pokusí se je sežrat zlobři, unesou je skřeti a při útěku z jejich jeskyní se Bilbo ztratí a potká Gluma, od kterého získá Prsten moci, uvězní je elfové z temného hvozdu, ježe naštěstí díky Bilbovi se jim podaří utéct.
Nakonec se jim konečně podaří dostat se k Osamělé hoře po strastiplné cestě, jenže tam sídlí drak Šmak.
Toho se nakonec podaří zabít a jejich království a hlavně bohatství je opět jejich.

Bohužel si na bohatství hory nedělají nároky jenom oni, dostanou se do blokády (lidmi, elfy, skřety…).
Král trpaslíků se ale zblázní (dostane Dračí nemoc- tj. chorobnou hrabivost), a proto aby se předešlo válce se Bilbo vydává tajně s arcikamem (po tom Thorin-král trpaslíků- nejvíc touží) pryč a tím donutí Thorina vzdát se.
Ten mu za trest nedá slíbený podíl za dobytí Hory, tak se vrací jen s obnosem, který mu jen tak tak stačí na vykoupení jeho majetku, který mezitím u něj doma rozprodali, v domnění, že je mrtvý.
\subsection*{Autor}
\begin{tabularx}{\linewidth}{l|l}
    \textbf{Jméno:} John Ronald Reuel Tolkien            & \textbf{Období:} 1892-1973, 1. světová válka                        \\
    \hline
    \textbf{Původ:} Jihoafrická republika, poté Británie & \textbf{Zaměstnání:} Prozaik, filolog, kritik, profesor, spisovatel \\
    \hline
    \multicolumn{2}{l}{Díla těží z autorovy vynikající znalostí starogermánské a keltské mytologie}                            \\
    \multicolumn{2}{l}{Účastnil se první světové války}                                                                        \\
    \multicolumn{2}{l}{Vztah s Edith Brattovou přenásí do díla Silmarilion, jména postav mají na náhrobku}                                                                                                       \\
    \hline
    \multicolumn{2}{l}{Další díla:}                                                                                            \\
    \multicolumn{2}{l}{\textbf{Pán prstenů -} Pokračování Hobita}                                                              \\
    \multicolumn{2}{l}{\textbf{Silmarillion -} Dějiny Středozemě}                                                              \\
    \multicolumn{2}{l}{\textbf{Sir Gawain a Zelený rytíř}}                                                                     \\
    \hline
    \multicolumn{2}{l}{Podobní autoři:}                                                                                        \\
    \multicolumn{2}{l}{\textbf{Období:} Clive Staples Lewis (Narnie), Robert Ervin Howard}                                     \\
    \multicolumn{2}{l}{\textbf{Český:} Karel Čapek}                                                                            \\
    \multicolumn{2}{l}{\textbf{Fantasy:} J. K. Rownlingová}                                                                    \\
\end{tabularx}
