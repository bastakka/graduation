\section{Hospoda na mýtince}
\label{sec:hospodanamytince}
\subsection*{Základní informace}
\begin{tabularx}{\linewidth}{l|l}
  \textbf{Literární forma:} Drama (Opereta (Hra se zpěvy)) & \textbf{Literární druh:} Drama - Komedie                \\
  \hline
  \textbf{Slohový postup:} Drama                           & \textbf{Typ vypravěče:} Drama                           \\
  \hline
  \textbf{Způsob vypravování:} Drama                       & \textbf{Žánr:} Poetismus                                     \\
  \hline
  \multicolumn{2}{l}{\textbf{Prostředí:} Konec 19. st., Hospoda uprostřed nespecifikovaného českého lesa}            \\
  \hline
  \multicolumn{2}{l}{\textbf{Jazykové prostředky:} V předehře jazyk spisovný a odborný, ve hře hovorový, dvojsmysly} \\
\end{tabularx}
\subsection*{Postavy}
\begin{tabularx}{\linewidth}{l|l}
  \textbf{Hostinský}               & Vlastní hospodu na mýtince                          \\
  \hline
  \textbf{Hrabě Ferdinand}         & Havaroval se vzducholodí, pašerák                   \\
  \hline
  \textbf{Vězeň Kulhánek}          & 20 let strávil ve věžení za pašeráctví, co neudělal \\
  \hline
  \textbf{Inspektor Trachta}       & Sedí u hospody 15 let                               \\
  \hline
  \textbf{Fiktivní vnučka Růženka} & Zamilují se do ní hrabě i vězeň                     \\
\end{tabularx}
\subsection*{Děj}
Děj je rozdělen na dvě části.

V první části děje se mluví o osobnosti Cimrmana.
O tom že neprošel pubertou, že pořídil klavírní výtah do divadla a o teorii absolutního rýmu.
Poté se zde mluví o samotné tvorbě operety "Proso", kterou Cimrmal napsal při opravě ztroskotané vzducholodi.
Samotné dílo poté Cimrman poslal do Vídně do soutěže operet, kde dílo bylo zcizeno.
Útržky originálu poté byly nesmazatelně zachyceny na nahrávkách.

V druhé části se odehrává samotný děj operety.
Hra začíná vysvětlením jak přišel pan honstinský k hodpodě uprostřed lesa.
Jednoho dne ztroskotá vzducholoď hraběte Ferdinanda u hospody.
Hostinský využívá této příležitiosti pohádkou o smyšlené dčeři a donutí hrabětě měsíc pobývat v hostinci.
Po měsící se na scénu dostává uprchlý věžeň, kterého měli další den po 20 letech pustit.
Poté nastávají boje o vymyšlenou dceru mezi hrabětem a vězněm.

Příběh končí odhalením, že důvod proč vězeň byl ve vězení je pan hrabě, který je následně odveden ze scény.
\subsection*{Autor}
\begin{tabularx}{\linewidth}{l|l}
  \textbf{Jméno:} Zdeněk Svěrák, Ladislav Smoljak & \textbf{Období:} 2. pol. 20. st.                                                      \\
  \hline
  \textbf{Původ:} Praha, Česká republika          & \textbf{Zaměstnání:} Scénaristé, herci, režiséři, dramatici                           \\
  \hline
  \multicolumn{2}{l}{Svěrák: Jeho film Kolja vyhrál Oscara, spolupracoval s Jaroslvem Uhlířem na písničkách}                              \\
  \multicolumn{2}{l}{Smoljak: Pedagog matematiky a fyziky}                                                                                \\
  \multicolumn{2}{l}{Divadlo Járy Cimrmana: Inteligentní humor, hrají pouze muži, }                                                       \\
  \hline
  \multicolumn{2}{l}{Další díla:}                                                                                                         \\
  \multicolumn{2}{l}{\textbf{Svěrák:} Kolja, Marečku, podejte mi pero, Obecná škola, Tři Bratři}                                          \\
  \multicolumn{2}{l}{\textbf{Smoljak:} Vrchní prchni! Jáchyme hoď ho do stroje! Kulový blesk (Většinou spoluautor)}                       \\
  \multicolumn{2}{l}{\textbf{Divadlo Járy Cimrmana:} Vyšetřování ztráty třídní knihy, Vražda v salóním kupé, Dlouhý Široký a Krátkozraký} \\
  \hline
  \multicolumn{2}{l}{Podobní autoři:}                                                                                                     \\
  \multicolumn{2}{l}{\textbf{Divadla té doby:} Divadlo Na zábradlí, Divadlo na provázku, Semafor (SEdm MAlých FORem)}                     \\
  \multicolumn{2}{l}{\textbf{Podobní autoři:} Suchý a Šlitr, Bolek Polívka, F. Hrubín}                                                    \\                                                               \\
\end{tabularx}
