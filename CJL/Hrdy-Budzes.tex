\section{Hrdý Budžes}
\subsection*{Základní informace}
\begin{tabularx}{\linewidth}{l|l}
    \textbf{Literární forma:} Próza (Groteskní román) & \textbf{Literární druh:} Epika                         \\
    \hline
    \textbf{Slohový postup:} Vyprávěcí                & \textbf{Typ vypravěče:} Osobní, ich-forma              \\
    \hline
    \textbf{Způsob vypravování:} Chronologické        & \textbf{Žánr:} Postmodernismnus                        \\
    \hline
    \multicolumn{2}{l}{\textbf{Prostředí:} 70. léta 20. st., Ničín}                                            \\
    \hline
    \multicolumn{2}{l}{\textbf{Jazykové prostředky:} Vulgarismy, nespisovný jazyk, satira, přímá řec, dialekt} \\
\end{tabularx}
\subsection*{Postavy}
\begin{tabularx}{\linewidth}{l|l}
    \multirow{2}{15em}{\textbf{Helenka Součková}}        & Žákyně druhé třídy základní školy v Ničíně                \\
                                                         & Nemá kamarády, je tlustá, ale přátelská a hodná           \\
    Také: \textbf{Freinsteinová} nebo \textbf{Brďochová} & Chce zůstat silná jako její imaginární hrdina Hrdý Budžes \\
    \hline
    \textbf{Kačenka}                                     & Maminka Helenky, herečka, zatvrzelá vůči režimu           \\
    \hline
    \textbf{Pepa Brďoch}                                 & Otčín Helenky, podobné rysy jako Kačenka                  \\
    \hline
    \textbf{Karel Freinstein}                            & Otec Helenky, žije v New Yorku, posílá Helence dárky      \\
    \hline
    \textbf{Pepíček}                                     & Nevlastní bratr Helenky                                   \\
    \hline
    \textbf{Babička}                                     & Helenka ji má na konci ráda, trvdohlavá a přísná          \\
    \hline
    \textbf{Dědeček}                                     & Helenka ho má radši než babičku, laskavý a ochotný        \\
\end{tabularx}
\subsection*{Děj}
Helenka chodí do druhé třídy. Je jiná než ostatní.
Spolužáci se jí smějí za to, že je tlustá, že její rodiče jsou herci a za to, že její příjmení je Freinsteinová, i když si říká Součková.
Chce zůstat silná a vytrvat jako její vzor Hrdý Budžes, o němž slyšela v básničce.
Chodí na němčinu k paní Freimanové, na sochání k panu Peckovi a nakonec jí rodiče dovolili chodit i na Jiskřičky, přestože si myslí, že jsou to mladí komunisté.
Helenka ráda kreslí, má bujnou fantazii, miluje Prahu a Milušku Voborníkovou.
O víkendech jezdí za babičkou a dědečkem do Zákopů.
Babička Kačenku stále pomlouvá a tajně píše Freinsteinovi o tom, jak se Helenka bez něho trápí, což není pravda.
Kačenka s Pepou se nechtěli stát komunisty, a proto pomalu v divadle přicházeli o role, až Kačenku vyhodili úplně.

Helenka sní o tom, že se s rodinou přestěhuje do Prahy.
Tento sen se jí splní, když Kačenka získá v Praze byt po tetě.
V závěru knihy zjišťuje, že žádný Hrdý Budžes neexistuje, že jde o splynutí slov Hrdý buď, žes…
Od Freinsteina dostala velké balení barevných fixek, které si původně moc přála, ale nahází jednu po druhé do kanálu.

\subsection*{Autor}
\begin{tabularx}{\linewidth}{l|l}
    \textbf{Jméno:} Irena Dousková (Irena Freidstatová) & \textbf{Období:} 2. pol. 20. st.                               \\
    \hline
    \textbf{Původ:} Příbram, Česká republika            & \textbf{Zaměstnání:} Spisovatelska, prozaička, dříve novinářka \\
    \hline
    \multicolumn{2}{l}{Vystudovala práva, ale nikdy je nedělala}                                                         \\
    \multicolumn{2}{l}{Otec emigorval do Izraele, rodina se přestěhovala do Prahy}                                       \\
    \multicolumn{2}{l}{Rodice jejího otce nepřežila 2. světovou válku}                                                   \\
    \hline
    \multicolumn{2}{l}{Další díla:}                                                                                      \\
    \multicolumn{2}{l}{\textbf{Oněgin byl Rusák -} Pokračování Hrdého Budžese}                                           \\
    \multicolumn{2}{l}{\textbf{Někdo s nožem}}                                                                           \\
    \multicolumn{2}{l}{\textbf{Golstein píše dceři} - Pásmo dopisů}                                                      \\
    \hline
    \multicolumn{2}{l}{Podobní autoři:}                                                                                  \\
    \multicolumn{2}{l}{\textbf{Období:} Michal Viewegh, Ludvík Vaculík}                                                  \\
    \multicolumn{2}{l}{\textbf{Podobné dílo:} Bylo nás pět - Karel Poláček}                                              \\
\end{tabularx}
