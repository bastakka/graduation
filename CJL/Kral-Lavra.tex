\section{Král Lávra}
\subsection*{Základní informace}
\begin{tabularx}{\linewidth}{l|l}
    \textbf{Literární forma:} Poezie (Alegorická satirická báseň) & \textbf{Literární druh:} Lyricko-Epický                        \\
    \hline
    \textbf{Slohový postup:} Vyprávěcí                            & \textbf{Typ vypravěče:} Autorský, Er-forma (Lyrický subjekt)   \\
    \hline
    \textbf{Způsob vypravování:} Chronologické                    & \textbf{Žánr:} Realismus                                       \\
    \hline
    \multicolumn{2}{l}{\textbf{Prostředí:} Čechy za rakouské nadvlády}                                                             \\
    \hline
    \multicolumn{2}{l}{\textbf{Jazykové prostředky:} ABCBDDB rým, spisovný a prostý jazyk, lidová mluva, personifikace, metonymie} \\
\end{tabularx}
\subsection*{Postavy}
\begin{tabularx}{\linewidth}{l|l}
    \multirow{2}{15em}{\textbf{Král Lávra}} & Dobrý panovník, nechce, aby nikdo věděl jeho tajemství          \\
                                            & Stydí se za svoje uši                                           \\
    \hline
    \textbf{Kukulín}                        & Holič, čestný, spravedlivý, bojácný, smířený s osudem, utrápený \\
    \hline
    \textbf{Kukulínova matka}               & Vdova, milující matka, která zachránila syna před šibenicí      \\
    \hline
    \textbf{Poustevník}                     & Žije opodál                                                     \\
    \hline
    \textbf{Basista Červíček}               & Hrál na basu a ztratil kolíček                                  \\
    \hline
    \textbf{Vypravěč}                       & Vypráví děj, nezasahuje do něj                                  \\
    \hline
\end{tabularx}
\subsection*{Děj}
Irská pohádka s českými motivy vypráví o podivuhodném králi, který byl dobrý král, ale měl jednu chybu.
Jednou do roka k sobě zavolal holiče, na kterého padl los, nechal se oholit, a pak dal lazebníka popravit.
Lid se velmi divil tomu, že slušně vypadající král může být takový tyran.

Jednou padl los na mladého Kukulína, jediného syna staré vdovy.
Vdova, ale šla za králem a hezky mu pověděla, co si o něm myslí.
Král se velmi zastyděl, a tak k sobě zavolal Kukulína.
Kukulín musel přísahat, že nikomu nepoví, co viděl.
Kukulín přísahal a stal se pak dvorním holičem.

Za nějaký čas však Kukulína tajemství velmi tížilo.
Matka to na něm poznala a poradila mu, aby zašel za poustevníkem do pustého lesa.
Poustevník mu poradil, aby tajemství pošeptal do vykotlané vrby.
Kukulín tak učinil a velmi se mu ulevilo.

Po králově ,,uzdravení“ byl uspořádán bál.
Na bál šel hrát také pan Červíček, který však cestou ztratil kolíček z basy.
Uřízl si tedy větev z vrby a vyřezal si nový kolíček.
Nevěděl však, do tím způsobí.
Když pak začal v paláci hrát, tu řve basa: ,,Král Lávra má oslí uši, král je ušatec!“
Král dal Červíčka okamžitě vyhodit.
Basu však nebylo možno pověsit, a tak král nechal svoje vlasy úplně ostříhat a nosil své dlouhé uši veřejně bez futrálu.
\subsection*{Autor}
\begin{tabularx}{\linewidth}{l|l}
    \textbf{Jméno:} Karel Havlíček Borovský & \textbf{Období:} 1821-1856 (4. fáze národního obrození)             \\
    \hline
    \textbf{Původ:} Česká republika         & \textbf{Zaměstnání:} Básník, novinář, ekonom, překladatel a politik \\
    \hline
    \multicolumn{2}{l}{Jméno Borovský je odevozeno z místa narození - Borová u Přibyslavi}                        \\
    \multicolumn{2}{l}{Napadá absolutismus a církev}                                                              \\
    \multicolumn{2}{l}{Umírá na tuberkulózu}                                                                      \\
    \hline
    \multicolumn{2}{l}{Další díla:}                                                                               \\
    \multicolumn{2}{l}{\textbf{Tyrolské elegie -} žalozpěv}                                                       \\
    \multicolumn{2}{l}{\textbf{Křest svatého Vladimíra -} satira}                                                 \\
    \multicolumn{2}{l}{\textbf{Obrazy z Rus -} cestopis}                                                          \\
    \hline
    \multicolumn{2}{l}{Podobní autoři:}                                                                           \\
    \multicolumn{2}{l}{\textbf{Realismus:} Josef Kajetán Tyl, Karel Jaromír Erben, Božena Němcová}                \\
    \multicolumn{2}{l}{\textbf{Zahraničí:} Honoré de Balzac, Guy de Maupassant}                                   \\
\end{tabularx}
