\section{Kulička}
    \subsection*{Základní informace}
        \begin{center}
            \begin{tabular}{l|l}
                \textbf{Literární forma:} Próza (Povídka) & \textbf{Literární druh:} Epika \\
                \hline
                \textbf{Slohový postup:} Vyprávěcí a popisný & \textbf{Typ vypravěče:} Autorský, er-forma \\
                \hline
                \textbf{Způsob vypravování:} Chronologické & \textbf{Žánr:} Naturalismus (Realismus) \\
                \hline
                \multicolumn{2}{l}{\textbf{Prostředí:} Zima roku 1870 - Normandie, Francie} \\
                \hline
                \multicolumn{2}{l}{\textbf{Jazykové prostředky:} Hovorové prostředky, nářečí, archaismy, spisovné, přirovnání, popis} \\
            \end{tabular}
        \end{center}
    \subsection*{Postavy}
        \begin{center}
            \begin{tabular}{l|l}
                \multirow{3}{15em}{\textbf{Alžběta Roussetová}} & Hlavní postava, lehká děva menšího, ale širšího vzrůstu. \\
                & Je velmi hodná, ale i přesto jí spolucestující opovrhují. \\
                & Má syna, která vyrůstá v jiné vesnici, vídá ho jednou ročně. \\
                \hline
                \multirow{2}{15em}{\textbf{Pan a paní Loiseauovi}} & \textbf{Pan:} Majitel obchodu s vínem, vychytralý, humorný \\
                & \textbf{Paní:} Statná, rázná, vnášela do obchodu řád a počty \\
                \hline
                \multirow{2}{15em}{\textbf{Pan a paní Carré-Lamndovi}} & \textbf{Pan:} Vážený člověk, majitel 3 prádelen, člen městské rady \\
                & \textbf{Paní:} Mladší drobná a hezká \\
                \hline
                \multirow{2}{15em}{\textbf{Hrabě a hraběnka de Bréville}} & \textbf{Hrabě:} Člen městské rady, majetný, vedoucí strany orleanistů\\
                & \textbf{Hraběnka:} Skvělá hostitelka, měla dobré vystupování \\
                \hline
                \textbf{Jeptišky} & Dvě neustále modlící se jeptišky, mladá a stará \\
                \hline
                \textbf{Cornudet} & Demokrat, zdědil peníze, které prochlastal, bojoval za revoluci \\
                \hline
                \textbf{Follenvie} & Tlustý majitel hostince, mluvil alsaskou francouzštinou \\
                \hline
                \textbf{Pruští vojáci, obyvatelé měst} & Pracující lid a vojáci pobývající v obydlí francouzského lidu \\
            \end{tabular}
        \end{center}
    \subsection*{Děj}
        Hlavní postavy se časně ráno vydávají do města Le Havre, aby se vyhnuli nebezpečí, které představovali Němci.
        Cestou kočár zapadá v závějích a cesta trvá mnohem déle, něž bylo plánováno.
        Zároveň nemohou narazit na žádný hostinec, a tak se musí Kulička podělit o své zásoby se spolucestujícími.

        Noc společnost přespí v hostinci v Tôtes, jenž je pod správou pruského důstojníka, který je odmítne pustit na další cestu, dokud se s ním Kulička nevyspí.
        Ta, poněvadž je velká vlastenka (odmítne proto i Cornudeta), odmítne.
        Ostatní s ní nejprve souhlasí, ale nakonec se jí rozhodnou přesvědčit o opaku.
        Díky dobrému hereckému výkonu a pomoci od jeptišek, které musí ošetřovat v Havru nemocné Kulička nakonec udělá, co jí bylo nařízeno.

        Prušák je tedy pustí, ale Kulička se od ostatních v dostavníku nedočká vděčnosti, jenom opovržení - je to tentokrát ona, kdo v kočáře hladoví…
    \subsection*{Autor}
        \begin{center}
            \begin{tabular}{l|l}
                \textbf{Jméno:} Guy de Maupassant & \textbf{Období:} 1850-1893 (Druhá pol. 19. st.)\\
                \hline
                \textbf{Původ:} Francie, ze zámožné rodin & \textbf{Zaměstnání:} Spisovatel, novinář a dramatik\\
                \hline
                \multicolumn{2}{l}{Vystudoval práva, účastnil se prusko-francouzské války} \\
                \multicolumn{2}{l}{Po úspěchu knihy Kulička se věnoval jen literatuře a žurnalistice} \\
                \multicolumn{2}{l}{Po pokusu o sebevraždu kvůli syfilis byl v ústavu pro duševně choré v Passy u Paříže, kde umřel} \\
                \hline
                \multicolumn{2}{l}{Další díla:} \\
                \multicolumn{2}{l}{\textbf{Miláček -} Román o cílevědomém kariéristovi využívající ženy ke svému vzestupu} \\
                \multicolumn{2}{l}{\textbf{Petr a Jan -} Román popisující příběh dvou bratru, jeden z nich se naštve že nedědí} \\
                \multicolumn{2}{l}{\textbf{Silná jako smrt -} Román o malíři ve vztahu s vdanou ženou, zjistí, že je radši sám} \\
                \hline
                \multicolumn{2}{l}{Podobní autoři:} \\
                \multicolumn{2}{l}{\textbf{Francie:} Gustav Flaubert, Honré de Balzac, Emile Zola} \\
                \multicolumn{2}{l}{\textbf{Rusko:} Fjodor Michajlovič Dostojevskij} \\
                \multicolumn{2}{l}{\textbf{Anglie:} Charles Dickens} \\
            \end{tabular}
        \end{center}