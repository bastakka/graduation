\section{Kytice}
\label{sec:kytice}
\subsection*{Základní informace}
\begin{tabularx}{\linewidth}{l|l}
  \textbf{Literární forma:} Poezie (Balady\footnotemark[1]) & \textbf{Literární druh:} Lyricko-epické                     \\
  \hline
  \textbf{Slohový postup:} Popisný                          & \textbf{Typ vypravěče:} Er-forma, autorský, lirický subjekt \\
  \hline
  \textbf{Způsob vypravování:} Chronologické                & \textbf{Žánr:} Romantismus                                  \\
  \hline
  \multicolumn{2}{l}{\textbf{Prostředí:} Dle básně}                                                                       \\
  \hline
  \multicolumn{2}{l}{\textbf{Jazykové prostředky:} Rýmy, přirovnání, apostrofy, dialogy, epiteton, personifikace}         \\
\end{tabularx}
\subsection*{Básně}
Jedná se o sbírků 13 básní v následujícím pořadí:\\
\begin{tabularx}{\linewidth}{l|l}
  \multirow{2}{15em}{\textbf{Kytice}}          & Tři sirotci chodí každé ráno na hrob své matby a věří, že se jejich           \\
                                               & maminka převtěluje do kytky, které pokrývají její hrob - Mateřídouška.        \\
  \hline
  \multirow{2}{15em}{\textbf{Poklad}}          & Vdova vešla s dítětem místo na vši do skály, kde bylo zlato. Zlova to se      \\
                                               & doma promněnilo v kamění a díte tam zůstalo. Za rok si ho vyzvedla.           \\
  \hline
  \multirow{2}{15em}{\textbf{Svatební košile}} & Dívka si v modlitbě zažádá o návrat milého nebo o smrt. Mrtvý milý si         \\
                                               & pro ni přichází. Na konci pochopení že je mrtvý a zachrání se modlitbou.      \\
  \hline
  \multirow{2}{15em}{\textbf{Polednice}}       & Matka se snaží utišit své dítě aby mohla dovařit oběd pro svého muže.         \\
                                               & Zavolá na své dítě polednici a při ochraně syna matka ze strachu ho udusí.    \\
  \hline
  \multirow{2}{15em}{\textbf{Zlatý kolovrat}}  & Král potká Dorua požádá její matku o její ruku. Masta místo toho její         \\
                                               & nevlastní dceru zabije a podstrčí vlastní. Nevlastní obžije a ty dvě umřou.   \\
  \hline
  \multirow{2}{15em}{\textbf{Štědrý den}}      & Hana a Marie na Štědrý večer se u jezera koukají na jejich budoucnost.        \\
                                               & Haně se objeví Václav a Marii rakev. Oběma se budoucnost vyplní.              \\
  \hline
  \multirow{2}{15em}{\textbf{Holoubek}}        & Žena u hřbitova vzpomíná na manžela a kolemjdoucí ji nabádá si ho vzít.       \\
                                               & Žena si ho vezme, ale za nepřekoná žal kvůli holoubkovi a tak se utopí.       \\
  \hline
  \multirow{2}{15em}{\textbf{Záhořovo lože}}   & Mladý hoch jde do pekla a cestou potká Záhoře. Ten ho nezabije pokud mu       \\
                                               & hoch řekne jaké to je v pekle. Hoch mu řekne, že na něj čekají a tak se poká. \\
  \hline
  \multirow{2}{15em}{\textbf{Vodník}}          & Dcera si udělá s vodníkem dítě i přes matčin zlý pocit. Když chce poté        \\
                                               & mamku navštívit vodník se naštve. Když se nedaří návrat domů dítě zabije.     \\
  \hline
  \multirow{2}{15em}{\textbf{Vrba}}            & Ženě, co přes noc jakoby nežila, muž přesekl vrbu, kde žila její duše. Aby    \\
                                               & svůj skutek spravil udělal z ní kolébku pro děťátko a ona mu bude hrát.       \\
  \hline
  \multirow{2}{15em}{\textbf{Lilie}}           & Na hrobě dívky vykvetla lilie. Pán si ji nechal zasadit do zahrady.           \\
                                               & Z Lilie vyroste paní, kterou matka pána nechá zvadnout a jejich dítě umřít.   \\
  \hline
  \multirow{2}{15em}{\textbf{Dcečina kletba}}  & Dialog matky a dcery. Dívka zabila své dítě a chce se oběsit. Nenávidí otce   \\
                                               & dítěte a její vlastní matku, kteří nezabránili tomu se toto stát.             \\
  \hline
  \multirow{2}{15em}{\textbf{Věštkyně}}        & Věštkyně prědpovícající budoucnost českého národa. Vlastenecká balada.        \\
                                               & Zdůrazňuje nepochopení slov Libuše a Svatopluka i přes uplynutí dlouhé doby.  \\
\end{tabularx}
\subsection*{Autor}
\begin{tabularx}{\linewidth}{l|l}
  \textbf{Jméno:} Karel Jaromír Erben      & \textbf{Období:} 1811 - 1870                                                               \\
  \hline
  \textbf{Původ:} Miletín, Česká republika & \textbf{Zaměstnání:} Básník,, historik, sběratel lidové slovesnosti                        \\
  \hline
  \multicolumn{2}{l}{Vytudoval práva a filozofii, při studiu se seznámil s Farntiškem Palackým}                                         \\
  \multicolumn{2}{l}{Byl redaktorem Pražským novin, sbíral archivní studijní materiály (historii českých rodů, lidové písně a pohádky)} \\
  \multicolumn{2}{l}{Dělal sekretáře Českému muzeu po projití soudní praxí}                                                             \\
  \hline
  \multicolumn{2}{l}{Další díla:}                                                                                                       \\
  \multicolumn{2}{l}{\textbf{Prostonárodní písně, pohádky a říkadla}}                                                                   \\
  \multicolumn{2}{l}{\textbf{Dlouhý, Široký a Bystrozraký, Pták ohnivák, Liška Ryška}}                                                  \\
  \hline
  \multicolumn{2}{l}{Podobní autoři:}                                                                                                   \\
  \multicolumn{2}{l}{\textbf{Období:} Josef Kajetán Tyl, Božena Němcová}                                                                \\
  \multicolumn{2}{l}{\textbf{Zahraničí:} George Gordon Byron, Victor Hugo, Puškin}                                                      \\
\end{tabularx}
\fancyfootnotetext{1}{Balada je lyricko-epický veršovaný útvar mnohdy s ponurým dějem, ovšem i šťastným, oslavným a většinou nešťastným, někdy až tragickým koncem.}
