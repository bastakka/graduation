\section{Lakomec}
\subsection*{Základní informace}
\begin{tabularx}{\linewidth}{l|l}
    \textbf{Literární forma:} Drama                       & \textbf{Literární druh:} Drama - Komedie                 \\
    \hline
    \textbf{Slohový postup:} Drama                        & \textbf{Typ vypravěče:} Drama                            \\
    \hline
    \textbf{Způsob vypravování:} Chronologické, 5 dějství & \textbf{Žánr:} Klasicismus                               \\
    \hline
    \multicolumn{2}{l}{\textbf{Prostředí:} Paříž roku 1670}                                                          \\
    \hline
    \multicolumn{2}{l}{\textbf{Jazykové prostředky:} Dialogy, spisovné, místy archaické, anarfory, satira, metafory} \\
\end{tabularx}
\subsection*{Postavy}
\begin{tabularx}{\linewidth}{l|l}
    \multirow{2}{15em}{\textbf{Harpagon}} & Bohatý vdovec - chamtivý, bezcitný, lichvář                         \\
                                          & Obětuje cokoliv - rodinu, děti i svoji lásku pro peníze             \\
    \hline
    \multirow{2}{15em}{\textbf{Kleantes}} & Syn Harpagona - chytrý a podnikavý                                  \\
                                          & Zamilovaný do Mariany a nechce se jí za žádnou cenu vzdát           \\
    \hline
    \multirow{2}{15em}{\textbf{Eliška}}   & Dcera Harpagona - upřímná, spravedlivá                              \\
                                          & Zamilovaná do Valéra, otec ji nařídil vzít si Anselma               \\
    \hline
    \textbf{Anselm}                       & Otec Valéra a Mariany - štědrý a dobrosrdečný šlechtic              \\
    \hline
    \multirow{2}{15em}{\textbf{Mariana}}  & Dcera Anselma - krásná, chudá                                       \\
                                          & Zamilovaná do Kleanta                                               \\
    \hline
    \multirow{2}{15em}{\textbf{Valér}}    & Ztracený bratr Mariany - Zamilovaný do Elišky                       \\
                                          & Maldý Harpagonův sluha                                              \\
    \hline
    \multirow{2}{15em}{\textbf{Čipera}}   & Kleantův sluha a přítel - mazaný                                    \\
                                          & Ukradne Harpagonovu truhlici s penězi, aby Kleantovi pomohl.        \\
    \hline
    \textbf{Frosina}                      & Všetečná rádkyně a intrikánka - pomáhá Harpagonovi a poté Kleantovi \\
\end{tabularx}
\subsection*{Dějství}
1. Eliška a Kleant se bojí říct otci, že mají partnery.
Harpagon však Kleantovi řekne, že si chce sám vzít za ženu jeho vysněnou lásku Marianu, a jeho chce oženit s bohatou vdovou. \\
2. Kleant si chce půjčit peníze od lichváře.
Ukáže se ale, že lichvářem je jeho otec Harpagon. \\
3. Do Harpagonova domu přichází Mariana na námluvy.
Když zjistí, že její milý Kleant je syn od Harpagona, má z toho velkou radost. \\
4. Páry se spolu domlouvají, jak zkazit Harpagonovu svatbu.
A tak Čipera schoval pokladnici svého pána, a tím obrátil Harpagonovu pozornost na peníze. \\
5. Harpagon si zavolá Komisaře, který pak všechny v domě vyslýchá.
Nakonec vyjde najevo, že Anselm je otcem Valéra a Mariany.
Když slíbí, že uhradí náklady za obě svatby, Harpagonovi již nic nebrání v tom, aby jim svatbu povolil, protože touha po penězích je veliká a navíc nebude muset na svatby vydat ani korunu.
\subsection*{Autor}
\begin{tabularx}{\linewidth}{l|l}
    \textbf{Jméno:} Moliére (Jean-Baptiste Poquelin) & \textbf{Období:} 1622-1673 (Druhá pol. 17. st.)            \\
    \hline
    \textbf{Původ:} Francie                          & \textbf{Zaměstnání:} Herec, dramatik, režisér              \\
    \hline
    \multicolumn{2}{l}{Měl převzít živnost otce, studoval v Paříži práva pro kariéru notáře. Stal se komediantem} \\
    \multicolumn{2}{l}{Ve svých hrách nejčastěji zesměšňoval společnost, jeho hry však popuzovaly královský dvůr} \\
    \multicolumn{2}{l}{Zemřel na jevišti při výstupu, když měl hrát umírajícího ve hře Zdravý nemocný.}           \\
    \hline
    \multicolumn{2}{l}{Další díla:}                                                                               \\
    \multicolumn{2}{l}{\textbf{Tartuffe -} Cenzurovaná}                                                           \\
    \multicolumn{2}{l}{\textbf{Zdravý nemocný -} Typ hypochondra}                                                 \\
    \multicolumn{2}{l}{\textbf{Měšťák šlechticem}}                                                                \\
    \hline
    \multicolumn{2}{l}{Podobní autoři:}                                                                           \\
    \multicolumn{2}{l}{\textbf{Klasicismus:} Pierre Corneille, Johann Wolfgang Goethe, Jean Racine}               \\
    \multicolumn{2}{l}{\textbf{Francie:} Denis Diderot, Voltaire, Carlo Goldoni}                                  \\
\end{tabularx}
