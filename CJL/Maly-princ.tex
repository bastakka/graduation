\section{Malý princ}
\subsection*{Základní informace}
\begin{tabularx}{\linewidth}{l|l}
    \textbf{Literární forma:} Próza (Filozofická pohádka)                           & \textbf{Literární druh:} Epika               \\
    \hline
    \textbf{Typ vypravěče:} Začátek a konec er-forma, pasáž malého prince ich-forma & \textbf{Slohový postup:} Vyprávěcí a popisný \\
    \hline
    \textbf{Způsob vypravování:} Chronologické, pasáž malého prince retrospektivní  & \textbf{Žánr:} Realismus                     \\
    \hline
    \multicolumn{2}{l}{\textbf{Prostředí:} Pravděpodobně 20. století, čas není určen, poušť a vesmír}                              \\
    \hline
    \multicolumn{2}{l}{\textbf{Jazykové prostředky:} Metafory, přirovnání, personifikace, nedokončené výpovědi, epiteton, symboly} \\
\end{tabularx}
\subsection*{Postavy}
\begin{tabularx}{\linewidth}{l|l}
    \multirow{3}{15em}{\textbf{Malý princ}}    & Malý, blonďatý kudrnatý chlapec                                  \\
                                               & Vydal se do světa kvůli potížím s květinou a hledat smysl života \\
                                               & Postava působí nevinně, plná porozumění a zvídavých otázek       \\
    \hline
    \multirow{2}{15em}{\textbf{Pilot}}         & Byl jediný kdo Malému princi rozumněl                            \\
                                               & Chápal jeho dětský styl myšlení                                 \\
    \hline
    \textbf{Květina (Růže)}                    & Popis postavy                                                    \\
    \hline
    \textbf{Liška}                             & Popis postavy                                                    \\
    \hline
    \textbf{Had}                               & Popis postavy                                                    \\
    \hline
    \multirow{2}{15em}{\textbf{Další postavy}} & Král, Lampář, Zeměpisec, Domýšlivec                              \\
                                               & Pijan, Byznzsmen, Výhybkář, Obchodník                            \\
\end{tabularx}
\subsection*{Děj}
První odstavec děje

Druhý odstavec

Třetí odstavec
\subsection*{Autor}
\begin{tabularx}{\linewidth}{l|l}
    \textbf{Jméno:} Antoine de Saint-Exupéry (Marie Roger) & \textbf{Období:} Meziválečná literatura       \\
    \hline
    \textbf{Původ:} Francie, šlechtický původ              & \textbf{Zaměstnání:} Letec                    \\
    \hline
    \multicolumn{2}{l}{Francouzský letec - byl sestřelen druhé světové války}                              \\
    \multicolumn{2}{l}{Vytýkal lidem omezené a jednostranné vnímání světa}                                 \\
    \multicolumn{2}{l}{Pro svá díla čerpal z oblasti letectví}                                             \\
    \hline
    \multicolumn{2}{l}{Další díla:}                                                                        \\
    \multicolumn{2}{l}{\textbf{Kurýr na jih}}                                                              \\
    \multicolumn{2}{l}{\textbf{Noční let}}                                                                 \\
    \multicolumn{2}{l}{\textbf{Letec}}                                                                     \\
    \hline
    \multicolumn{2}{l}{Podobní autoři:}                                                                    \\
    \multicolumn{2}{l}{\textbf{Meziválečná literatura:} Alexandr Dumas, Thomas Mann, Erich Maria Remarque} \\
    \multicolumn{2}{l}{\textbf{Francie:} Anatole France, Romain Rolland, Henri Barbusse}                   \\
\end{tabularx}
