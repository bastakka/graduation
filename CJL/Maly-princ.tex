\section{Malý princ}
\subsection*{Základní informace}
\begin{tabularx}{\linewidth}{l|l}
    \textbf{Literární forma:} Próza (Filozofická pohádka)                           & \textbf{Literární druh:} Epika               \\
    \hline
    \textbf{Typ vypravěče:} Začátek a konec er-forma, pasáž malého prince ich-forma & \textbf{Slohový postup:} Vyprávěcí a popisný \\
    \hline
    \textbf{Způsob vypravování:} Chronologické, pasáž malého prince retrospektivní  & \textbf{Žánr:} Realismus                     \\
    \hline
    \multicolumn{2}{l}{\textbf{Prostředí:} Pravděpodobně 20. století, čas není určen, poušť a vesmír}                              \\
    \hline
    \multicolumn{2}{l}{\textbf{Jazykové prostředky:} Metafory, přirovnání, personifikace, nedokončené výpovědi, epiteton, symboly} \\
\end{tabularx}
\subsection*{Postavy}
\begin{tabularx}{\linewidth}{l|l}
    \multirow{3}{15em}{\textbf{Malý princ}}    & Malý, blonďatý kudrnatý chlapec z planety B612                   \\
                                               & Vydal se do světa kvůli potížím s květinou a hledat smysl života \\
                                               & Postava působí nevinně, plný porozumění a zvídavých otázek       \\
    \hline
    \multirow{2}{15em}{\textbf{Pilot}}         & Byl jediný kdo Malému princi rozumněl                            \\
                                               & Chápal jeho dětský styl myšlení                                  \\
    \hline
    \textbf{Květina (Růže)}                    & Domýšlivá, marnivá, nafoukaná, z planety B612                    \\
    \hline
    \textbf{Liška}                             & Moudrá, přátelská, pozitivně ovliví jeho myšlení                 \\
    \hline
    \textbf{Had}                               & Jeho uštknutí vrátí prince zpět na jeho planetku                 \\
    \hline
    \multirow{2}{15em}{\textbf{Další postavy}} & Král, Domýšlivec, Pijan, Byznzsmen, Lampář, Zeměpisec            \\
                                               & Výhybkář, Obchodník
\end{tabularx}
\subsection*{Děj}
Pilot letadla nouzově přistává na Sahaře.
Když se svůj stroj snaží opravit, objeví se Malý princ (postava z cizí planetky, kterou opustil, protože pochyboval o lásce své růže, kterou miloval), který pilota žádá, aby mu nakreslil beránka.
Pilot kreslit neumí, nejprve namaluje hroznýše se slonem v žaludku, poté několik nepovedených kreseb beránka až nakonec beránka v krabici.
Malý princ je nadšený.

Poté pilotovi vypravuje o svých návštěvách na jiných planetách, kde potkával různé dospělé.
\begin{enumerate}
    \itemsep-0.5em
    \item Planetka: zde potkal krále, který všechny považuje za své poddané
    \item Planetka: Domýšlivec – chce být všemi uctíván a respektován, přestože zde žije zcela sám
    \item Planetka: Pijan, který pije, protože se stydí za to, že pije
    \item Planetka: Byznzsmen, který si myslí, že je všechno jeho a že vše lze koupit, počítá hvězdy
    \item Planetka: Lampář, který bez odpočinku zhasíná a rozsvěcí svou lampu kvůli rychle se střídajícímu dni a noci
    \item Planetka: Zeměpisec, který se neustále stará pouze o mapy
    \item Planetka = Země: zde potkává lišku, ta mu vysvětlí, že jediné správné vnímání světa, je vnímání srdcem.
          Na Zemi princ vidí lidskou uspěchanost.
          Mezitím se pilotovi podaří opravit své letadlo.
          Malý princ je smutný, jelikož jeho planeta je daleko, a tak se nechá uštknout hadem, aby jeho cesta zpět byla jednodušší.
          Po smrti se jeho duše vydává zpět na planetu k milované růži, za kterou se po nyní, po poznání skutečných citů, cítí zodpovědný.
\end{enumerate}

\subsection*{Autor}
\begin{tabularx}{\linewidth}{l|l}
    \textbf{Jméno:} Antoine de Saint-Exupéry (Marie Roger) & \textbf{Období:} 1900-1944 (První pol. 20. st.) \\
    \hline
    \textbf{Původ:} Francie, šlechtický původ              & \textbf{Zaměstnání:} Letec                      \\
    \hline
    \multicolumn{2}{l}{Francouzský letec - byl sestřelen druhé světové války}                                \\
    \multicolumn{2}{l}{Vytýkal lidem omezené a jednostranné vnímání světa}                                   \\
    \multicolumn{2}{l}{Pro svá díla čerpal z oblasti letectví}                                               \\
    \hline
    \multicolumn{2}{l}{Další díla:}                                                                          \\
    \multicolumn{2}{l}{\textbf{Kurýr na jih, Noční let, Letec}}                                              \\
    \hline
    \multicolumn{2}{l}{Podobní autoři:}                                                                      \\
    \multicolumn{2}{l}{\textbf{Meziválečná literatura:} Alexandr Dumas, Thomas Mann, Erich Maria Remarque}   \\
    \multicolumn{2}{l}{\textbf{Francie:} Anatole France, Romain Rolland, Henri Barbusse}                     \\
\end{tabularx}
