\section{Maryša}
\label{sec:marysa}
\subsection*{Základní informace}
\begin{tabularx}{\linewidth}{l|l}
  \textbf{Literární forma:} Drama                       & \textbf{Literární druh:} Drama - Tragédie                          \\
  \hline
  \textbf{Slohový postup:} Drama                        & \textbf{Typ vypravěče:} Drama                                      \\
  \hline
  \textbf{Způsob vypravování:} Chronologické, 5 jednání & \textbf{Směr:} Kritický Realismus                                  \\
  \hline
  \multicolumn{2}{l}{\textbf{Prostředí:} Rok 1886, moravská vesnice na Slovácku}                                             \\
  \hline
  \multicolumn{2}{l}{\textbf{Jazykové prostředky:} Jihomoravské nářečí, realistický popis, spisovná mluva u mladší generace} \\
\end{tabularx}
\subsection*{Postavy}
\begin{tabularx}{\linewidth}{l|l}
  \textbf{Lízal}         & bezcitný, lakomý, vychytralý                                \\
  \hline
  \textbf{Maryša}        & Dcera Lízala, mladá, citlivá dívka - miluje Francka         \\
  \hline

  \textbf{Lízalova žena} & Žena Lízala, krutá, přísná                                  \\
  \hline
  \textbf{Filip Vávra}   & sebevědomý, krutý, agresivní mlynář - stane se mužem Maryši \\
  \hline
  \textbf{Francek}       & pracujicí, věrný, statečný rekrut - miluje Maryšu           \\
\end{tabularx}
\subsection*{Děj}
Sedlák Lízal chtěl svou jedinou dceru Maryšu provdat za mlynáře Vávru (pro peníze), ale ona milovala chudého Francka.
Francek je odveden na vojnu.

Maryšu nutí rodiče ke sňatku s mlynářem Vávrou, otcem tří dětí, který slibuje, že se o ni dobře postará.
Jde mu hlavně o peníze, které dostane Maryša věnem, aby mohl zaplatit své dluhy.
Vávra se začíná opíjet, nestará se o rodinu a soudí se se starým Lízalem o Maryšino věno.

Lízal si konečně uvědomuje, za koho svou dceru provdal a odmítne mu peníze dát.
Po návratu najde Francek Maryšu provdanou za Vávru a vidí jenom její utrpení.
Připravuje plán společného útěku do Brna, kde našel pro sebe i pro Maryšu práci.
Maryša odmítá.

V rozčilení a opilosti chce Vávra Francka zabít, ale Maryša mu v tom zabrání.
Ráno Vávra už po několikáté lituje svého chování a slibuje, že se polepší.
Avšak Maryša mu už nevěří a ve chvíli zoufalství mu nasype do kávy jed a vzápětí se k tomu přizná.
\subsection*{Autoři}
\begin{tabularx}{\linewidth}{l|l}
  \textbf{Jméno:} Alois a Vilém Mrštíkovi & \textbf{Období:} Druhá pol. 19. st. - Začátek 20. st.       \\
  \hline
  \textbf{Původ:} Jimramov, Vysočina      & \textbf{Zaměstnání:} Dramatici a prozaikové                 \\
  \hline
  \multicolumn{2}{l}{\textbf{Alois -} Vystudovaný učitel, střídal školy}                                \\
  \multicolumn{2}{l}{\textbf{Vilém -} Překládal z ruštiny, nedokončil práva, bojuje za mravnost}        \\
  \multicolumn{2}{l}{Oba přispívali svými články do časopisů, novin.}                                   \\
  \hline
  \multicolumn{2}{l}{Další díla:}                                                                       \\
  \multicolumn{2}{l}{\textbf{Alois -} Sbírka povídek Nit stříbrná, povídky Dobré duše}                  \\
  \multicolumn{2}{l}{\textbf{Vilém -} Román Santa Lucia, literární kritika Moje sny}                    \\
  \hline
  \multicolumn{2}{l}{Podobní autoři:}                                                                   \\
  \multicolumn{2}{l}{\textbf{Čeští :} Božena Němcová, Alois Jirásek, Karel Havlíček Borovský}           \\
  \multicolumn{2}{l}{\textbf{Realismus :} Alexandr Nikolajevič Ostrovskij, Nikolai Gogol, Henrik Ibsen} \\
\end{tabularx}
