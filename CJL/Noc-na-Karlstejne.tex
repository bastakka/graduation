\section{Noc na Karlštejně}
\label{sec:nocnakarlstejne}
\subsection*{Základní informace}
\begin{tabularx}{\linewidth}{l|l}
    \textbf{Literární forma:} Drama                        & \textbf{Literární druh:} Drama - Komedie \\
    \hline
    \textbf{Slohový postup:} Drama                         & \textbf{Typ vypravěče:} Drama            \\
    \hline
    \textbf{Způsob vypravování:} Chronologické - 3 dějství & \textbf{Žánr:} Novoromantismus           \\
    \hline
    \multicolumn{2}{l}{\textbf{Prostředí:} Karlštejn, červen 1363}                                    \\
    \hline
    \multicolumn{2}{l}{\textbf{Jazykové prostředky:} Archaismy, historismy, dialogy, spisovné}        \\
\end{tabularx}
\subsection*{Postavy}
\begin{tabularx}{\linewidth}{l|l}
    \multirow{3}{15em}{\textbf{Karel IV.}} & Rozumný, čestný, spravedlivý a schopný vládce                            \\
                                           & Miluje svoji ženu Alžbětu                                                \\
                                           & Dokáže se povznést nad prohřešky                                         \\
    \hline
    \textbf{Ješek z Vartenberka}           & Purkrabí na Karlštejně, zodpovědný a dosbrosrdečný                       \\
    \hline
    \textbf{Alžběta (Eliška) Pomořanská}   & Císařovna, miluje Karla, stýská se jí a žárlí                            \\
    \hline
    \textbf{Alena}                         & Neteř Ješka, miluje Peška, odvážná, vtipná a podnikavá                   \\
    \hline
    \textbf{Pešek Hlavně}                  & Šenk jeho výsosti, zmatkař, miluje Alenu, touží po rytířských ostruhách  \\
    \hline
    \textbf{Arnošt z Pardubic}             & Arcibiskup, pomáhá všem                                                  \\
    \hline
    \textbf{Petr}                          & Král kyperský a jeruzalémský, jeho halvnímy zájmy jsou víno, zpěv a ženy \\
    \hline
    \textbf{Štěpán}                        & Vévoda bavorksý, ziskuchtivý, ale čestný, dokáže ustoupit                \\
\end{tabularx}
\subsection*{Děj}
Roku 1363 se na hrad sjíždí panovník a jím pozvaní hosté, cyperský a jeruzalémský král Petr a Štěpán Bavorský.
Přestože ženy mají na hrad vstup zakázán, přijíždějí nezávisle na sobě královna Alžběta, která neunesla stesk a žárlivost, a neteř purkrabího Alena.
Ta z lásky ke zde sloužícímu číšníku Peškovi chce vyhrát sázku uzavřenou se svým otcem, že pokud se dostane na hrad, svolí otec ke sňatku.
Obě ženy se přestrojí za pážata.

Král Petr něco tuší, žádá po pážeti polibek a při vzájemném zápase páže zlomí meč.
Císař v něm poznává svou ženu.
Následně je vyzrazena i přítomnost Aleny.
Císař je za tuto shodu okolností vděčný, neboť Alenino přestrojení může zachránit čest císařovny.

Z toho důvodu je Aleně odpuštěno, její Pešek je pasován na rytíře a císařovna úspěšně předstírá svůj příjezd na hrad s omluvou, že se při lovu ztratila v lesích.
Hra končí odjezdem Karla a jeho ženy na její hrad Karlík a novým císařovým rozhodnutím, jež zpřístupňuje Karlštejn i ženám.
\subsection*{Autor}
\begin{tabularx}{\linewidth}{l|l}
    \textbf{Jméno:} Jaroslav Vrchlický (Emil Frída) & \textbf{Období:} (2. pol. 19. st.)                                \\
    \hline
    \textbf{Původ:} Česká republika                 & \textbf{Zaměstnání:} Lumírovec, Básník, dramatik, novinář, kritik \\
    \hline
    \multicolumn{2}{l}{Vystudoval filozofickou fakultu, poté dělal vychovatele v Itálii}                                \\
    \multicolumn{2}{l}{Získal doktorát Karlovy univerzity a titul profesora}                                            \\
    \multicolumn{2}{l}{Prodělal mozkovou mrtvici, po které nemohl komunikovat, číst a psát}                             \\
    \hline
    \multicolumn{2}{l}{Další díla:}                                                                                     \\
    \multicolumn{2}{l}{\textbf{Meš Damoklův -} Vydáno posmrtně}                                                         \\
    \multicolumn{2}{l}{\textbf{Okna v bouři -} Intimní lyrická básnická sbírka}                                         \\
    \multicolumn{2}{l}{\textbf{Zlomky epopeje -} Básnická sbírka o vývoji lidstva}                                      \\
    \hline
    \multicolumn{2}{l}{Podobní autoři:}                                                                                 \\
    \multicolumn{2}{l}{\textbf{Ruchovci:} Eliška Krásnohorská, Alois Jirásek, Svatopluk Čech, Josef Václav Sládek}     \\
    \multicolumn{2}{l}{\textbf{Lumírovci:} Julius Zeyer, Josef Václav Sládek, Jaroslav Vrchlický}                       \\
\end{tabularx}
