\section{O myších a lidech}
\label{sec:omysichalidech}
\subsection*{Základní informace}
\begin{tabularx}{\linewidth}{l|l}
  \textbf{Literární forma:} Próza (Novela s prvky balady\footnotemark[1]) & \textbf{Literární druh:} Epika                    \\
  \hline
  \textbf{Slohový postup:} Vyprávěcí                                      & \textbf{Typ vypravěče:} Er-forma, Autorský        \\
  \hline
  \textbf{Způsob vypravování:} Chronologické                              & \textbf{Žánr:} Realismus                          \\
  \hline
  \multicolumn{2}{l}{\textbf{Prostředí:} 30. léta 20. století v Kalifornii na jedné farmě blízko města Soledad}               \\
  \hline
  \multicolumn{2}{l}{\textbf{Jazykové prostředky:} Nespisovný jazyk, dialogy, slangy, hovorový jazyk, vulgarismy, konstrasty} \\
\end{tabularx}
\subsection*{Postavy}
\begin{tabularx}{\linewidth}{l|l}
  \multirow{3}{15em}{\textbf{Lennie Small}}  & Urostlý a velice silný chlap, pracovitý, působy jako hodný                    \\
                                             & Bohužel trpí mentální zaostalostí a to se mu stane osudným                    \\
                                             & Rád sahá na hebké věci, nedokáže zvládat sílu a emoce                         \\
  \hline
  \multirow{2}{15em}{\textbf{George Milton}} & Člověk, který se stará o Lennieho, velické hodný a obětový                    \\
                                             & Rád by žil spokojený život, ale není mu přáno                                 \\
  \hline
  \textbf{Candy}                             & Starý bezruký uklízeč na farmě, také touží po spokojeném životě               \\
  \hline
  \textbf{Curley}                            & Malý, žárlivý, zakomplexovaný syn majitele farmy, rád se pere                 \\
  \hline
  \textbf{Curleyho žena}                     & Mladá a přitažlivá žena, hledá společnost, ale lidé se ji kvůli muži vyhýbají \\
  \hline
  \textbf{Slim}                              & Respektovaný dělník farmy, spravedlivý a objektivní                           \\
  \hline
  \textbf{Cooks}                             & Jediný černoch mezi dělníky, hodný a inteligentní, ale je odstrkovaný         \\
  \hline
  \textbf{Další dělníci}                     & Whit, Carlson                                                                 \\
\end{tabularx}
\subsection*{Děj}
Lennieho se ujal kamarád George Milton.
Oba dva se vydávají na farmu, kde se seznamují s ostatními pracovníky.
Oba sní o tom, že si jednoho dne pořídí své vlastní hospodářství a budou chovat králíky s heboučkou srstí.

Curley, syn majitele farmy, je ženatý, ale jeho žena s ním není moc šťastná.
Proto často vyhledává společnost dělníků.
A tak se stalo, že jednoho dne navštívila i Lennieho.
Ten se jí svěřil, že má v oblibě hebké věci.
Dovolila mu tedy pohladit si její vlasy.
Lennie je však tiskl tak silně, až Curleyho žena začala strachy křičet.
Lennie, celý vyděšený, ji nechtěně zlomí vaz.

Uteče se schovat k řece, jak mu poradil George.
Curley zjistil, co Lennie udělal jeho ženě, a vydal se ho zabít.
Avšak George Lennieho vyhledá dříve, chce ho ušetřit od bolesti a krutého zacházení, a tak jej v okamžiku, kdy spolu mluví o svém snu, střelí do hlavy.
\subsection*{Autor}
\begin{tabularx}{\linewidth}{l|l}
  \textbf{Jméno:} John Steinbeck           & \textbf{Období:} 1902 - 1968                                                              \\
  \hline
  \textbf{Původ:} Salinas, Kalifornie, USA & \textbf{Zaměstnání:} Tesař, zeměměřič, úředník v obchodním domě, pomáhal na ranči         \\
  \hline
  \multicolumn{2}{l}{Začal psát historické romance, poté se přesunul na sociální problémy}                                             \\
  \multicolumn{2}{l}{Byl také oceánologem a zkousel biologické teorie života}                                                          \\
  \multicolumn{2}{l}{Vystudoval Stanfordovu univerzidu v Kalifornii, roku 1962 docal nobelovu cenu za literaturu}                      \\
  \hline
  \multicolumn{2}{l}{Další díla:}                                                                                                      \\
  \multicolumn{2}{l}{\textbf{Hrony zpěvu -} Impozatní román z období krize vyjadřující duch 30. let}                                   \\
  \multicolumn{2}{l}{\textbf{Na východ od ráje -} Nebiblický epos rodového bloudění a generačního střetu}                              \\
  \multicolumn{2}{l}{\textbf{Bitva -} O stávce, zkoumá nálady davu}                                                                    \\
  \hline
  \multicolumn{2}{l}{Podobní autoři:}                                                                                                  \\
  \multicolumn{2}{l}{\textbf{Realismus :} Honoré de Balzac, Ernest Hemingway, Trolstoj, Gogol}                                         \\
  \multicolumn{2}{l}{\textbf{Český Realismus:} Mrštíkové, Eliška Krásnohorská, Božena Němcová, Alois Jirásek, Karel Havlíček Borovský} \\
\end{tabularx}
\fancyfootnotetext{1}{Balada = neveselý děj a tragický konec příběhu}
