\section{Piková dáma}
\label{sec:pikovadama}
\subsection*{Základní informace}
\begin{tabularx}{\linewidth}{l|l}
  \textbf{Literární forma:} Próza (Novela s fantastickým námětem) & \textbf{Literární druh:} Epika                                   \\
  \hline
  \textbf{Slohový postup:} Vyprávěcí                              & \textbf{Typ vypravěče:} Er-forma, autorský                       \\
  \hline
  \textbf{Způsob vypravování:} Chronologicky                      & \textbf{Žánr:} Romantismus                                       \\
  \hline
  \multicolumn{2}{l}{\textbf{Prostředí:} Rusko, 18. století}                                                                         \\
  \hline
  \multicolumn{2}{l}{\textbf{Jazykové prostředky:} Spisovný jazyk, přímá řeč, citáty, metafory, archaismy, přirovnání personifikace} \\
\end{tabularx}
\subsection*{Postavy}
\begin{tabularx}{\linewidth}{l|l}
  \multirow{2}{15em}{\textbf{Heřman}} & Ctižádostivý muž                                                    \\
                                      & Udělal by cokoli pro splacení dluhu                                 \\
  \hline
  \textbf{Hraběnka Anna Fedotovna}    & Rozumná, mazaná, komanduje Lizavetu, nepřipouští si konec své slávy \\
  \hline
  \textbf{Lizaveta Ivanovna}          & Služka hraběnky, mladá, naivní, hodná, srdečná, pracovitá, citlivá  \\
\end{tabularx}
\subsection*{Děj}
Vše začíná u karetní hry, kde Tomskij začne vyprávět příběh své babičky hraběnky Anny Fedotovny.
Ta jednou prohrála všechny peníze při karetní hře, ale znala jednoho hraběte, který jí poradil jak vyhrát peníze zpět - vsadit na tři karty.
Když se o této prověřené metodě jak získat peníze dozví Heřman, naplánuje taktiku, jak by se mohl s hraběnkou setkat.
Začne psát milostné dopisy její služebné Lizavetě Ivanově.
Ta mu prozradí, jak se dostat do ložnice hraběnky.
Heřman vtrhne do pokoje a vyhrožuje hraběnce s pistolí v ruce.
Hraběnka nechápe a pak náhle umře.

Za pár dní se mu zjeví duch staré hraběnky, poví mu tajemství tří karet, ovšem za podmínky, že si vezme Lizavetu.
První den vyhraje v kartách na trojku, druhý den na sedmu.
Třetí den vsadí všechny peníze na eso - nebo si to alespoň myslel.
Až padla poslední karta, zjistil, že vsadil na pikovou dámu.
Podíval se na kartu a zdálo se mu, že na něho mrkla hraběnka.
\subsection*{Autor}
\begin{tabularx}{\linewidth}{l|l}
  \textbf{Jméno:} Alexandr Sergejevič Puškin & \textbf{Období:} 1799 - 1837 (1. pol. 19. st.)                                  \\
  \hline
  \textbf{Původ:} Rusko                      & \textbf{Zaměstnání:} Básník, prozaik, dramatik                                  \\
  \hline
  \multicolumn{2}{l}{Považován za zakladatelel moderní ruské prózy}                                                            \\
  \multicolumn{2}{l}{Kvůli tvorbě musel odejít do vyhnanství na jih Ruska}                                                     \\
  \multicolumn{2}{l}{Až do své smrti zůstal pod policejním dohledem, zemřel na následky zranění v souboji}                     \\
  \hline
  \multicolumn{2}{l}{Další díla:}                                                                                              \\
  \multicolumn{2}{l}{\textbf{Boris Godunov -} drama}                                                                           \\
  \multicolumn{2}{l}{\textbf{Kapitánská dcera -} próza}                                                                        \\
  \multicolumn{2}{l}{\textbf{Kavkazský jezdec -} lyricka-epická skladba}                                                       \\
  \hline
  \multicolumn{2}{l}{Podobní autoři:}                                                                                          \\
  \multicolumn{2}{l}{\textbf{Rusko:} Nikolaj Vasilijevič Gogol, Michail Jurijevič Lermontov, Vasilij Alexandrejevič Žukovskij} \\
  \multicolumn{2}{l}{\textbf{Směr/období 2:} Victor Hugo, Geroge Gordon Byron, Walter Scott}                                   \\
\end{tabularx}
