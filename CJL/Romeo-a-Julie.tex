\section{Romeo a Julie}
\subsection*{Základní informace}
\begin{tabularx}{\linewidth}{l|l}
    \textbf{Literární forma:} Drama                                & \textbf{Literární druh:} Drama - Tragédie   \\
    \hline
    \textbf{Slohový postup:} Drama                                 & \textbf{Typ vypravěče:} Drama               \\
    \hline
    \textbf{Způsob vypravování:} Chronologické, prolog + 5 dějství & \textbf{Žánr:} Renesance                    \\
    \hline
    \multicolumn{2}{l}{\textbf{Prostředí:} 16. století, Verona, Itálie}                                          \\
    \hline
    \multicolumn{2}{l}{\textbf{Jazykové prostředky:} Blankverse, metafory, citové zabarvení, oxymoróny, inverze} \\
\end{tabularx}
\subsection*{Postavy}
\begin{tabularx}{\linewidth}{l|l}
    \textbf{Montek, Kapulet}           & Hlavy dvou znepřátelených rodů                                       \\
    \hline
    \multirow{2}{15em}{\textbf{Romeo}} & Syn Monteka, zamilovaný do Julie                                     \\
                                       & Hrdý, čestný, citlivý, romantický                                    \\
    \hline
    \textbf{Merkucio}                  & Přítel Romea, agresivní a bojovný                                    \\
    \hline
    \textbf{Benvolio}                  & Přítel Romea, rozumný, mírný, čestný, spravedlivý, zábavný           \\
    \hline
    \multirow{3}{15em}{\textbf{Julie}} & Dcera Kapuletova, zamilovaná do Romea                                \\
                                       & Tvrdohlavá, obětavá, neposlouchala rodiče                            \\
                                       & 14 let                                                               \\
    \hline
    \textbf{Tybalt}                    & Bratranec Julie, bojovný, útočný, provokativní                       \\
    \hline
    \textbf{Paris}                     & Nápadník Julie, vytvrvalý, upřímný                                   \\
    \hline
    \textbf{Chůva}                     & Chůva Julie, nerozumná, ale spolehlivá a důvěryhodná, pomáhala Julii \\
    \hline
    \textbf{Otec Vavřinec}             & Starý kněz ochotný pomoci, moudrý, vzdělaný, čestný                  \\
\end{tabularx}
\subsection*{Děj}
Romeo a Julie čerpá z příběhu tragické lásky dvou mladých lidí ze znepřátelených rodů Monteků a Kapuletů v italské Veroně.
Rod Kapuletů pro svou jedinou dceru Julii připravuje zásnubní maškarní bál, kde ji chce zaslíbit mladému šlechtici Parisovi.
Na tento bál pronikne i jediný Montekův syn Romeo a spolu s Julií se během něj do sebe zamilují a následně se i tajně vezmou.

Romeo je u šarvátky, v níž jeho přítele Merkucia zabije Juliin bratranec Tybalt.
Vyzve ho proto na souboj, ve kterém jej zabije, za což je vypovězen z města.

Julie chce uniknout svatbě s Parisem, vypije proto nápoj, po kterém její tělo ustrne v klinické smrti.
K Romeovi se však zpráva o jejím záměru nedostane, a pokládá ji proto za mrtvou.
Romeo nad její hrobkou nejprve probodne v souboji Parise a poté sám vypije jed.
Julie se bohužel probudí pozdě, a jakmile zjistí, že Romeo je mrtev, vezme jeho dýku a v žalu se probodne.

Oba rody se nad mrtvolami svých potomků následně usmiřují...

\subsection*{Autor}
\begin{tabularx}{\linewidth}{l|l}
    \textbf{Jméno:} William Shakespeare & \textbf{Období:} 1564-1616 (Druhá pol. 16. st. - Začátek 17. st.) \\
    \hline
    \textbf{Původ:} Anglie              & \textbf{Zaměstnání:} Básník a dramatik                            \\
    \hline
    \multicolumn{2}{l}{Vystudoval gymánium a stal se spolumajitelem divadla}                                \\
    \multicolumn{2}{l}{Ze začátku tvořil komedie, poté tragédie, tvořil také historické hry}                \\
    \multicolumn{2}{l}{Náměty čerpal ze životopisů slavných osobností}                                      \\
    \hline
    \multicolumn{2}{l}{Další díla:}                                                                         \\
    \multicolumn{2}{l}{\textbf{Zkrocení zlé ženy -} Komedie}                                                \\
    \multicolumn{2}{l}{\textbf{Hamlet -} Tragédie}                                                          \\
    \multicolumn{2}{l}{\textbf{Jinřich IV. -} Historická hra}                                               \\
    \hline
    \multicolumn{2}{l}{Podobní autoři:}                                                                     \\
    \multicolumn{2}{l}{\textbf{Renesance:} Dante Alighiery, Giovanni Boccaccio}                             \\
\end{tabularx}
