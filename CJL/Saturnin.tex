\section{Saturnin}
\subsection*{Základní informace}
\begin{tabularx}{\linewidth}{l|l}
    \textbf{Literární forma:} Próza (Humoristický román)                         & \textbf{Literární druh:} Epika            \\
    \hline
    \textbf{Slohový postup:} Vyprávěcí a popisný                                 & \textbf{Typ vypravěče:} Osobní, ich-forma \\
    \hline
    \textbf{Způsob vypravování:} Chronologické, ale byla použita i retrospektiva & \textbf{Žánr:} Žánr                       \\
    \hline
    \multicolumn{2}{l}{\textbf{Prostředí:} Praha a venkov, 20. léta 20. st.}                                                 \\
    \hline
    \multicolumn{2}{l}{\textbf{Jazykové prostředky:} Spisovný jazyk, satira, archaismy, vulgarismy, přísloví}                \\
\end{tabularx}
\subsection*{Postavy}
\begin{tabularx}{\linewidth}{l|l}
    \textbf{Vypravěč}                & Měšťanský typ, gentleman bez většího smyslu pro humor, zamilovaný do Barbory      \\
    \hline
    \textbf{Saturnin}                & Vypravěčův sluha, velký smysl pro humor, řídí většinu děje                        \\
    \hline
    \textbf{Dědeček}                 & Starý bohatý pán, ředitel elektrické elektrárny, všichni se snaží dostat dědictví \\
    \hline
    \textbf{Teta Kateřina}           & Je hamižná a hrabivá, používá neustále přísloví                                   \\
    \hline
    \textbf{Milouš}                  & Syn Kateřiny, ta s ním i v 18 letech zachází jako dítě                            \\
    \hline
    \textbf{Slečna Barbora Terebová} & Moderní, milá žena                                                                \\
    \hline
    \textbf{Doktor Vlach}            & Rodinný přítel, vymýšlí teorie o chování lidí                                     \\
    \hline
\end{tabularx}
\subsection*{Děj}
Mladý muž přijme Saturnina jako sluhu.
Tím se jeho svět otočí naruby.
Z klidného bytu se přestěhují na hausbót, kde čelí každý den vlnobití a projíždějícím parníkům.
Pak se rovněž na popud Saturnina vydá tento mladý muž chytit lva, který utekl ze zoo.
Když přijede na návštěvu teta Kateřina s Miloušem, namluví jim Saturnin, že jsou na lodi myši, aby se dotěrných příbuzných zbavili.

Poté se rozhodnou navštívit dědečka mladého pána, který žije na venkově.
Zde se mladý muž seznamuje s Barborou, do které se brzy zamiluje.
K dědečkovi však přijede i doktor Vlach a teta Kateřina s Milošem.
Zažívají spolu řadu komických situací, nicméně nakonec všechno dobře dopadne.

Mladý muž se sblíží s Barborou a odstěhují se spolu do Prahy, teta se znova bohatě provdá a Saturnin se stará o dědečka.

\subsection*{Autor}
\begin{tabularx}{\linewidth}{l|l}
    \textbf{Jméno:} Zdeněk Jirotka  & \textbf{Období:} 1911-2003 (První pol. 20. st.)                          \\
    \hline
    \textbf{Původ:} Česká republika & \textbf{Zaměstnání:} Spisovatel, fejetonista, autor románů, povídek\dots \\
    \hline
    \multicolumn{2}{l}{Vystudoval stavební průmyslovou školu v Ostravě}                                        \\
    \multicolumn{2}{l}{Nikdy nepřekonal svůj první román Saturnin}                                             \\
    \multicolumn{2}{l}{Pracoval v Lidových novinách, Svobodných novinách\dots}                                 \\
    \hline
    \multicolumn{2}{l}{Další díla:}                                                                            \\
    \multicolumn{2}{l}{\textbf{Muž se psem -} parodie na detektivní romány}                                    \\
    \multicolumn{2}{l}{\textbf{Sedmilháři, Pravda se změnila -} sbírky povídek}                                \\
    \multicolumn{2}{l}{\textbf{Hvězdy nad starým Vavrouchem -} rozhlasová hra}                                 \\
    \hline
    \multicolumn{2}{l}{Podobní autoři:}                                                                        \\
    \multicolumn{2}{l}{\textbf{Čeští:} Karel Poláček, Eduard Bass, Vladislav Vančura}                          \\
\end{tabularx}
