\section{Spalovač mrtvol}
\label{sec:spalovacmrtvol}
\subsection*{Základní informace}
\begin{tabularx}{\linewidth}{l|l}
  \textbf{Literární forma:} Próza (Psychologická novela\footnotemark[1]) & \textbf{Literární druh:} Epika             \\
  \hline
  \textbf{Slohový postup:} Vyprávěcí                                     & \textbf{Typ vypravěče:} Er-forma, Autorský \\
  \hline
  \textbf{Způsob vypravování:} Chronologické                             & \textbf{Žánr:} Druhá vlna válečné prózy    \\
  \hline
  \multicolumn{2}{l}{\textbf{Prostředí:} Praha v období protektorátu (1937)}                                          \\
  \hline
  \multicolumn{2}{l}{\textbf{Jazykové prostředky:} Spisovné, děsivé, složité metafory a symboliky, oslovení}          \\
\end{tabularx}
\subsection*{Postavy}
\begin{tabularx}{\linewidth}{l|l}
  \multirow{3}{15em}{\textbf{Karel Kopfrkingl}} & Nechává si říkat Roman, protože má rád romantiku                  \\
                                                & Nejprve je zásadový, ale jeho názory se pod vlivem Willhema mění  \\
                                                & Býval zaměstananec krematoria, ale poté se měni v udavače a vraha \\
  \hline
  \textbf{Willhelm Reinke}                      & Hrdý Němec přesvědčuje Karla k přechodu na německou stranu        \\
\end{tabularx}
\subsection*{Děj}
Pan Kopfrkingl se zpočátku profiluje jako mírný a hodný člověk, který žije jen pro svou rodinu.
Zdá se i romanticky založený. Vzpomíná na okamžik, kdy se se svou ženou seznámil před leopardí klecí, a také označuje své rodinné příslušníky jmény jako „čarokrásná, nebeská, nadoblačná apod“.

Ideální rodinná atmosféra je ovšem neustále narušována K prací, o které neustále všem vypráví - dělá spalovače mrtvol v pražském krematoriu.
Rád se obklopuje věcmi ze svého pracoviště, v kuchyni má například kremační tabulku, které přezdívá „jízdní řád smrti“, v jeho knihovně nechybí kremační zákon a také kniha o Tibetu.

Svébytný svět protagonisty narušuje návštěva dávného přítele, nyní přesvědčeného nacisty, Williho Reinkeho.
Ten přiměje K uvěřit v rasovou nadřazenost („vzpomeňte si na kapku německé krve“) a nutnost zbavit se pojítek s bytostmi nižšími a méněcennými.
Nakonec nátlaku podléhá a začíná udávat lidi kolem sebe, čímž si zajistí cestu do nacistického Casina a je také povýšen na ředitele krematoria.
Manželčin židovský původ mu brání v kariéře, proto postupně zavraždí nejen ji (Co abych tě drahá oběsil?), ale i svého změkčilého syna Miliho, kterého obzvlášť krutě utluče v kremační místnosti.
Na základě pokřivené interpretace tibetské filozofie hodnotí páchané zločiny jako pomoc nešťastníkům, kteří nechápou vyšší zájmy světa.

Když se pokusí zabít i svou osmnáctiletou dceru, přeruší jej jeho dvojník - výplod choré mysli, tibetský vyslanec, nazve ho inkarnací Buddhy a vyzve ho, aby se ujal trůnu ve Lhase, neboť dalajláma zemřel a mniši již 19 let hledají jeho nástupce.
Prožitek osvícení završují tři „andělé“, kteří odvádějí Kopfrkingla do sanitky a odváží do blázince.
V závěru novely je protagonista konfrontován s průvodem vyhublých lidí, kteří se vracejí po skončení války z koncentračního tábora.
Ani tehdy se nevzdává přesvědčení, že je mesiášem.
Kniha končí K slovy: „Šťastné lidstvo. Spasil jsem je.“
\subsection*{Autor}
\begin{tabularx}{\linewidth}{l|l}
  \textbf{Jméno:} Ladislav Fuks          & \textbf{Období:} 1923 - 1994            \\
  \hline
  \textbf{Původ:} Praha, Česká republika & \textbf{Zaměstnání:} Prozaik            \\
  \hline
  \multicolumn{2}{l}{Měl problém při válce kvůli jeho homosexuální orientaci}      \\
  \multicolumn{2}{l}{Oženil se s bohatou italkou, od které krátce po svatbě utekl} \\
  \multicolumn{2}{l}{Byl hospitalizován na psychiatrii}                            \\
  \hline
  \multicolumn{2}{l}{Další díla:}                                                  \\
  \multicolumn{2}{l}{\textbf{Mí černovlastí bratři}}                               \\
  \multicolumn{2}{l}{\textbf{Pan Theodor Munstock}}                                \\
  \multicolumn{2}{l}{\textbf{Zámek Kynžvart}}                                      \\
  \hline
  \multicolumn{2}{l}{Podobní autoři:}                                              \\
  \multicolumn{2}{l}{\textbf{Období:} Ota Pavel, Bohumil Hrabal, Karel Čapek}      \\
\end{tabularx}
\fancyfootnotetext{1}{Novela je narozdíl od románu kratší a rozvíjí pouze jednu dějovou linii.}
