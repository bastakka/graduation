\section{Vražda v Orient expresu}
\label{sec:orientexpres}
\subsection*{Základní informace}
\begin{tabularx}{\linewidth}{l|l}
    \textbf{Literární forma:} Próza (Detektivní Román) & \textbf{Literární druh:} Epika               \\
    \hline
    \textbf{Slohový postup:} Vyprávěcí                 & \textbf{Typ vypravěče:} Autorský, er-forma   \\
    \hline
    \textbf{Způsob vypravování:} Chronologické         & \textbf{Žánr:} Žánr                          \\
    \hline
    \multicolumn{2}{l}{\textbf{Prostředí:} Vlak Orient Expres, Zima roku 1929, Jugoslávie}            \\
    \hline
    \multicolumn{2}{l}{\textbf{Jazykové prostředky:} Francoužština, přímá řeč, ironie, personifikace} \\
\end{tabularx}
\subsection*{Postavy}
\begin{tabularx}{\linewidth}{l|l}
    \multirow{3}{15em}{\textbf{Hercule Poirot}} & Belgický detektiv, velice inteligentní, přemýšlivý, bystrý a jízlivě vtipný \\
                                                & Trochu výstřední, samolibý a puntičkářský, ale genialní                     \\
                                                & Vyzná se v lidech a psychologii a při vyšetřování spoléhá jen na svůj mozek \\
    \hline
    \textbf{Samuel Edward Ratchett (Casetti)}   & Byla z něj cítit zloba, strach, krutost, byl zavražděn                      \\
    \hline
    \textbf{Hector MacQueen}                    & Američan, tajemník Ratchetta, nevěděl o tom, že Ratchett je Casetti         \\
    \hline
    \textbf{Monsieur Bouc}                      & Ředitel vlakové společnosti, přítel Poirota                                 \\
    \hline
    \textbf{Doktor Constantine}                 & Lékař, který se podílí na vyšetřování.                                      \\
    \hline
    \textbf{Armstrongovi}                       & Ostatní cestující, všichni se podíleli na vraždě                            \\ 
\end{tabularx}
\subsection*{Děj}
Poirot přijede vlakem z Aleppa do Istanbulu, kde se ubytuje v hotelu Tokatlian.
Zde dostane telegram vyzývající ho k návratu do Londýna.
Zakoupí si jízdenku na Orient expres a večer odjede.

Během první noci je jeden z cestujících zavražděn.
Nedaleko města Vinkovci vlak uvízne v závěji a musí čekat na vyproštění.
Poirot postupně vyšetřuje 13 podezřelých cestujících a ukáže se, že všichni jsou nějakým způsobem spojeni s rodinnou tragédií Johna Armstronga, kterému byla unesena a zabita dcera.
Únoscem byl zavražděný cestující.

\subsection*{Autor}
\begin{tabularx}{\linewidth}{l|l}
    \textbf{Jméno:} Agatha Christie & \textbf{Období:} 1890 - 1976 (2. pol. 20. století), neorealismus, existencionalismus \\
    \hline
    \textbf{Původ:} Anglie          & \textbf{Zaměstnání:} Lékárnice a sestra                                                 \\
    \hline
    \multicolumn{2}{l}{Druhá nejprodávanější spisovatelka všech dob (první je W. Shakespeare)}                                \\
    \multicolumn{2}{l}{Vyučována doma, poté na studiích v Paříži}                                                             \\
    \multicolumn{2}{l}{V první světové válce pracovala jako dobrovolná sestra a lékárnice}                                    \\
    \hline
    \multicolumn{2}{l}{Další díla:}                                                                                           \\
    \multicolumn{2}{l}{\textbf{Past na myši -} divadelní hra}                                                                 \\
    \multicolumn{2}{l}{\textbf{Svědek obžaloby -} divadelní hra}                                                              \\
    \multicolumn{2}{l}{\textbf{Smrt na Nilu -} detektivka s vyšetřovatelem Poirotem}                                          \\
    \hline
    \multicolumn{2}{l}{Podobní autoři:}                                                                                       \\
    \multicolumn{2}{l}{\textbf{Období :} J. R. R. Tolkien, J. K. Rowling, Vladimir Nabokov}                                   \\
    \multicolumn{2}{l}{\textbf{Umělecký směr :} Edgar Allan Poe, Robert Louis Stevenson, Oscar Wilde}                         \\
    \multicolumn{2}{l}{\textbf{Období :} P. D. Jamesová, Dick Francis, William Golding}                                                     \\
\end{tabularx}
