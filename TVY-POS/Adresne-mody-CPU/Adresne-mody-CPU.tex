\section{Adresné módy běhu CPU}
\label{sec:adresnemody}
Každý procesor využívá nějakou vnější paměť, v našem případě se většinou jedná o paměť RAM.
Tuto paměť musí procesor nějakým způsobem adresovat, aby s ní mohl pracovat.
Adresování paměťi může probíhat 4 způsoby.
\subsection{Reálný režim}
Paměť kterou každý x86 CPU začíná.
Z tohoto režimu ho přepíná OS většinou do chráněného režimu.
Reálný režim znamená že pracuje s fyzickou adresou paměti, tudíž je adresace omezena šířkou adresní sběrnice.
Pracoval v něm DOS, využívá se akorát pro zpětnou kompatibilitu.
Tento režim nezabraňuje programům používat paměť jiných programů.
Fyzická adresa se počítá pomocí segmentu a offsetu.
Bez himem.sys vidí DOS pouze 1MB paměti.
V tomto případě první třetina paměti bývá prostor pro aplikace a pro přerušení.
Zbylé dvě třetiny paměti se používá pro systém (BIOS, video, aj.)

\subsection{Virtualní reálný režim}
Dokáže emulovat více Reálných režimů vedle sebe ve chráněnném režimu.

\subsection{Chráněnný režim}
Režim virtuálních adres.
Umožňuje využití větší části paměti podle velikosti registru.
32bitové procesory mohou pracovat s paměti do až velikosti 4GB.

\subsubsection{Multitasking}
Umožňuje paralelní práci více programů.
Procesor řeší ochranu mezi jednotlivými spuštěnými programy pomocí virtuální paměti pro každý program.
Každý samostatný program se takto chová jako by měl celou paměť pro sebe a neřeší kam svá data zapisuje.
To přenechává aplikace součásti CPU zvanou MMU.

\subsubsection{MMU}
Správuje přístup do paměti.
Vytváří virtuální adresy ze selektoru, offsetu a tabulky deskriptorů a zařizuje překlad virtuální adresy na fyzickou.
Selektor obsahuje informace o oprávnění přístupu, zdali se jedná o globální nebo lokální prostor a poté samotný index.
Globální sektor je určený více uživatelům, který zastřešuje ochranná funkce, která odděluje systémový prostor or uživatelského.
Lokální prostor je určení pro jedinečná data. (Nenašel jsem co to znamená)

\subsubsection{Stránkování segmentace}
MMU je schopné rozdělit operační paměť na stejně velké části (stránky).
Tyto stránky se následně přidelí programu a zapíše se jaké stránky daný program má.
Tyto stránky mohou být rozmístěny různě po paměti, tudíž je možné že část programu bude na začátku a část na konci paměti.
Pro výpočet adresy je poté nutné společně se selektorem a offsetem využít zapsanou stránku.
Bez této technologie by nám mohli časem v paměti vzniknout nevyužité mezery v paměti.

\subsection{Dlouhý režim}
Režim pro použití 64bitových instrukcí a registrů.
Programy, které tyto instrukce nepoužívají jsou spuštěny v režimu kompatibilty.
