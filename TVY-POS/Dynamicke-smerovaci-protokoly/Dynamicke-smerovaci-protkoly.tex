\section{Dynamické směrovací protokoly}
\label{sec:smerovaci-protokoly}
\subsection{Úvod}
Pro přesun paketů ze zdroje na požadované zařízení na lokálních sítích se dnes využívá routererů, které si vytváří tzv. směrovací tabulku, pomocí které vybírají nejlepší cestu pro tento paket.
V této tabulce je možné najít:
\begin{itemize}
  \item Adresu sítě (Na které je požadované zařízení)
  \item Typ cesty (Přímá, statická či dle prokolu)
  \item Fyzické rozhraní, po kterém se paket přenese
\end{itemize}
Mezi typy cest patří:
\begin{enumerate}
  \item Přímo připojené (C)
        \begin{itemize}
          \item Není nutno konfiguovat
          \item Přidají se automaticky je-li rozhraní aktivní
          \item Administrativní vzdálenost: 0
        \end{itemize}
  \item Statické (S)
        \begin{itemize}
          \item Zadávána ručne
          \item Je nutno specifikovat IP adresu sítě s maskou a rozhraní routeru
          \item Případně lze specifikovat IP adresu protějšího routeru a směr
          \item Administrativní vzdálenost: 1
          \item Využívají se pro:
                \begin{itemize}
                  \item Připojení sítě do Internetu pomocí jednoho ISP
                  \item Větší sítě v konfigurace „Hub-and-spoke“
                \end{itemize}
        \end{itemize}
  \item Dynamické (Dle protokolu)
        \begin{itemize}
          \item Generují se automaticky a dynamicky
          \item Routery si navzájem sdílí informace o těchto tabulkách, podle kterých upravují vlastní
        \end{itemize}
\end{enumerate}
\subsection{Rozdělení protkolů}
Protkoly se v základu dělí na:
\begin{itemize}
  \item IGP - Používané v rámci jedné domény
  \item EGP - Používané v rámci ISP na WAN
\end{itemize}
Dělí se také ale podle dělí podle využití masky sítí na:
\begin{itemize}
  \item Classfull - Nevyuživá masky sítí, nemohou být použity ve VLSM
  \item Classless - Využívá a vyžaduje masky sítí
\end{itemize}
Samotné IGP protokoly se poté dělí na:
\begin{itemize}
  \item Protkoly měřící vzdálenost a směr (DVRP - Distance Vektor Routing Protocol)
  \item Protkoly měřící stav linek a topologickou mapu sítě (LSRP - Link State Routing Protocol)
\end{itemize}
\subsection{Distance vector protokoly (DVRP)}
Cesta je určena vektorem, ve kterém hrají role vzdálenost a směr.
Vzdálenost je určena \textbf{počtem hopů}.
Směr je určen výstupním rozhraním routeru nebo naopak adresou vstupu vedlejšího routeru.
K určení cesty se používá \textbf{Bellman-Ford} algoritmu.
Tento algoritmus neumožňuje routerům znát skutečnou mapu síťové topologie.
Některé protokoly pravidelně rozesílají kompletní směrovací tabulky všem sousedům, čímž způsobují velkou zátěž na lince.
\subsection{Link state prokoly (LSRP)}
Cesta je určena stavem linek.
Každý router zná tímto způsobem celou topologii sítě a pomocí toho generují nejlepší cestu.
Síť se zde zatěžuje velmi méně, jelikož místo rozesílání celých tabulek se zde rozesílají pouze aktualizace stavu jednotlivých linek.
Vhodné pro větší a rozsáhlejší sítě.
\subsection{Konvergence}
Jedná se o dobu, za kterou se na všech routerech vytvoří platné a nejlepší směrovací tabulky.
Mezi parametry důležité pro konvergenci patří rychlost šíření informací a rychlost výpočtu optimálních cest.
\subsection{Metrika protkolů}
\begin{tabularx}{\linewidth}{l|l}
  \textbf{Parametr} & \textbf{Popis}                     \\
  \hline
  Počet hopů        & Počet routerů mezi zdrojem a cílem \\
  \hline
  Šířka pásma       & Propustnost linky                  \\
  \hline
  Zátěž             & Zatížení linky                     \\
  \hline
  Zpoždění          & Doba průchodu paketu linkou        \\
  \hline
  Spolehlivost      & Pravděpodobnost chyby linky        \\
  \hline
  Cena              & Určeno politikou linky             \\
\end{tabularx}
\subsection{Tabulka protokolů}
\begin{tabularx}{\linewidth}{l|l|l|l|l}
  \textbf{Název (IPv6 verze)}           & \textbf{Zkratka}        & \textbf{Vlastnosti}               & \textbf{Metrika}       & \textbf{Admin. vzdálenost} \\
  \hline
  RIP (RIPng)                           & R                       & Interní DVRP                      & Počet hopů             & 120                        \\
  \hline
  \multirow{2}{8em}{{IGRP}}             & \multirow{2}{5em}{{I}}  & \multirow{2}{8em}{{Interní DVRP}} & Šířka pásma, zpoždění, & \multirow{2}{5em}{{90}}    \\
                                        &                         &                                   & spolehlivost a zátěž                                \\
  \hline
  \multirow{2}{8em}{{EIGRP (for IPv6)}} & \multirow{2}{5em}{{EX}} & \multirow{2}{8em}{{Interní DVRP}} & Šířka pásma, zpoždění, & \multirow{2}{5em}{{170}}   \\
                                        &                         &                                   & spolehlivost a zátěž                                \\
  \hline
  OSPFv2 (OSPFv3)                       & O                       & Interní LSRP                      & Cena                   & 110                        \\
  \hline
  IS-IS (for IPv6)                      & i                       & Interní LSRP                      & Cena                   & 115                        \\
  \hline
  EGP                                   & E                       & Externí Path Vector               & Různé                  & 140                        \\
  \hline
  BGPv4 (for IPv6)                      & B                       & Externí Path Vector               & Různé                  & 20                         \\
\end{tabularx}
