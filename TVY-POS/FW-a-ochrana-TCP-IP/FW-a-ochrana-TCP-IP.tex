\section{FW a ochrany jednotlivých vrstev TCP/IP}
\label{sec:fw-a-ochrana-tcp-ip}
Ochrana před útoky na síti je důležitá pro bezpečnost společnosti.
Zaručí si tím ochranu před krádeží dat.
Každá společnost si sama stanovuje jaké prvky bezpečnosti použije dle "Bezpečnostní politky".
\subsection{Bezpečnostní politika}
Pojem který popisuje "kdo může co udělat" na síti.
Z počátku každý důvěřoval na síti každému, tudíž byla síť zneužitelná.
Pak přišel rázný přechod na politiku "Nikomu nic nevěřím".
Přišlo řešení s názvem \textbf{Active directory}, které nabízelo uživatelům přidělit role, které určovaly přístup k daným službám či datům.
Bylo důležité uživatelům nabídlout balanci mezi přístupem k potřebným věcech a zapřít přístup k ostatním.
\subsection{Firewall}
Pro zabezpečení přístupu zvenční se používají pravidla, které se určují v zabezpečovací vrstvě zvané \textbf{Firewall}.
V této vrstvě je možné určit okruh IP adres, které mají přístup k dané službě nebo i čas po který je služba přístupná.
Může být realizované pomocí SW (Na počítači, routeru) ale i HW (Samostané zařízení nebo v routeru).
Dvě sítě ošetřené firewallem mají mezi sebou demlitarizovanou zónu.
Co všechno lze tímto zakázat:
\begin{itemize}
  \item Aplikace
  \item Porty
  \item Protokoly
\end{itemize}
\subsection{Ochrany jednotlivých vrstev TCP/IP}
\begin{enumerate}
  \item Fyzická - Klíč na počítače, správné chlazení, schované kabely
  \item Linková - Filtr fyzických adres, omezení adres na port, blokace nepoužitých portů
  \item Síťová - Filtr logických adres a provozu na síti (VPN)
  \item Transportní - Filtr protoků a portů (HTTPS)
  \item Aplikační - Antivir, proxy, uživatelská gramotnost
\end{enumerate}
Služby zapínám pouze pokud je potřebuji, jinak jsou vypnuté.
