\section{Fyzická vrstva}
\label{sec:fyzicka-vrstva}
Jedná se o nijižší vrstvu ISO-OSI modelu.
Zajišťuje fyzické spojení obou stran.
Patří do ní kabeláž, HW, konektory, vysílače, příjimače aj.
Definuje elektrické, mechanické, optické a elektromagentické vlny a jejich vlastnosti.
Informace přenáší ve formě bitů - kabely, vlny a světlo neznají ale bity - je nutné definovat jaký stav znamená log. 1 a co log. 0.
Komunikace probíhá oběžníkovým způsobem a je řízena Linkovou vrstvou.
\subsection{Zařízení pracující na fyzické vrstvě}
\begin{description}
  \item[Zesilovač]- Zesiluje signál i se šumem, je jednodušší, levnější a rychlejší než opakovač
  \item[Opakovač]- Regeneruje signál (Opraví, zesílí a načasuje), elektricky odděluje segmenty
  \item[HUB (Rozbočovač)]- Navyšuje konektivitu, regeneruje signál, rozvětvuje signál do více výstupů
  \item[Kabely a propojky] 
\end{description}
\subsection{Přenosová média}
Mezi vlastnosti přenosových médií patří:\\
\begin{tabularx}{\linewidth}{l|l|l}
  \textbf{Parametr}    & \textbf{Jednotka} & \textbf{Popis}                                      \\
  Přenosová rychlost   & Bajt za sekundu   & Množství dat, které lze přenést za sekundu          \\
  \hline
  Útlum signálu        & dB na vzdálenost  & Zeslabení signálu po průchodu určitou délkou vodiče \\
  \hline
  Odolnost vůči rušení & ---               & Zejména proti elektromagickému rušení               \\
  \hline
  Zkreslení signálu    & ---               & Jaká změna nastala na signálu po průchodu médiem    \\
  \hline
  Fyzická odolnost     & ---               & Odolnost proti ohybu, tahu, mechická pevnost        \\
\end{tabularx}
\subsubsection{Kroucená dvojlinka}
Pravidelně zkroucený pár vodičů, většinou měděných.
V dnešní době se používá několik takových navzájem zkroucených párů.
Zkroucením se minimalizuje vyzařování elektromagentických vln a minimalizují se přeslechy.
Dnes používané hojně jako Patch kabely (Ethernet).
\subsubsection{Koaxilní kabel}
Kabel s vnitřním vodičem a druhým válcovým vnějším vodičem.
Navzájem jsou vodiče odděleny dielektrikem.
Průmery kabelů jsou od milimetrů po desítky centimetrů.
Používají se pro televize.
\subsubsection{Optické kabely}
Přenášejí data pomocí světelného paprsku.
Jsou mnohem dražší a komplikovanější než metalické vodiče.
Zdrojem světla bývají LED či laserové diody.
Pro přenos ve využívá \textbf{totálního odrazu} (Fyzikální zákon, který umožňuje odrazit veškeré světlo pomocí přechodu světla z opticky hustšího prostředí do opticky řidšího)
Dělí se na:
\begin{description}
  \item[Jednovidová] - Přenáší pouze jeden světelný paprsek
  \item[Mnohovidová] - Přenáší více světelných paprsků
\end{description}
\subsubsection{Rádiové spoje}
Jedná se o přenosy elektromagentických vln v určitém rozsahu elektromagentického spektra.
Hlavní výhodou tohoto spoje je absence kabelů, tudíž bezdrátová komunikace.
Nevýhodou je možnost oblivnění přenosu pomocí vlivů, které by u drátové nebyly problém.
Pro rádiové spoje se využívá různých frekvečních délek například 87.6 Mhz (Rádio Impuls) a 5 Ghz (Wi-Fi).
Mezi další technologie hojně využívající rádiové spoje patří: Bluetooth, Mobilní signál, GPS aj.
\subsection{Druhy možností šíření signálu}
\subsubsection{Sériové a paralelní přenos}
\begin{description}
  \item[Sériové spoje]- Bity jdou za sebou postupně jedním vodičem. Bajty jsou tvořeny posloupností bitů.
  \item[Paralelní spoje]- Několik bitů se posílá současně pomocí více vodičů. Obvykle se přenáší celý bajt.
\end{description}
\subsubsection{Synchronní a asynchronní přenos}
\begin{description}
  \item[Synchronní spoje]- Přenos dat probíhá společně se signálem CLK, který generuje pouze jedna strana.
  \item[Asynchronní spoje]- Každá strana si CLK generuje sama. Komunikace začíná pomocí start a stop bitu.
\end{description}
\subsubsection{Symetričnost singálu}
\begin{description}
  \item[Symetrický signál]- Přenos probíhá dvou vodičím v opačné polaritě. Výstupní signál je rozdíl přenosů.
  \item[Asymetrický signál]- Přenos probíhá po jednom vodiči. Náchylné na poruchy, nižší spolehlivita.
\end{description}