\section{Fyzická vrstva}
Zajišťuje fyzické spojení obou stran.
Patří do ní kabeláž, HW, konektory aj.
Komunikace probíhá oběžníkovým způsobem a je řízena Linkovou vrstvou.
Definuje fyzické, elektrické, mechanické a funkční parametry fyzického propojení.
Přenáší se samotné Bity převodem bitů na signál ať už elektrický či vlnový.\\
\begin{description}
  \item[Analogový okruh]- jednodušší, nižší rychlosti, menší odolnost proti rušení, asynchronní i synchronní, nižší cena
  \item[Digitální okruh]- komunikace synchronní, odolnost proti rušení, komfort vzdáleného řízení linky, využívá stejná média\\
\end{description}
\subsection{HW}
Spojení může poskytnout:
\begin{itemize}
  \item Optický kabel - možnost využití jiných barev pro více linek
  \item Metalický kabel - Koax, Kroucená dvojlinka
  \item Wi-Fi, Bluetooth, Rádio - Pomocí protokolu IEEE 802.11 aj.\\
\end{itemize}
Zařízení pracující na fyzické vrstvě:
\begin{description}
  \item[Opakovač]- Regeneruje signál a elektricky odděluje segmenty
  \item[HUB]- Navyšuje konektivitu, regeneruje signál, rozvětvuje signál do více výstupů\\
\end{description}
\subsection{Synchronní přenos}
Používán pro přenosy, kde je vyžadována garance rychlosti přenosu (zvuk, video..) tk. zajištěna požadovaná šíře pásma.
Při komunikaci se musí přenášet synchronizační signál, tzv. hodiny CLK ze zdroje hodin, udává ho jenom jedna ze stran komunikace.
Signál CLK může jít po vlastním vodiči, nebo jedním společným fyzickým kanálem (vkomponován s daty).
\subsection{Asynchronní přenos}
Používán pro přenosy, kde je vyžadována jednoduchost komunikace a její široká přizpůsobitelnost.
Způsob synchronizace je prováděn v datovém toku tzv. služebními signály.
Signál CLK není obsažen ve vlastním datovém toku, jelikož si ho každá strana vytváří sama.
Vzorkovací signál CLK je asi 10x vyšší než přenosová rychlost.
Přenos dat i propustnost je nižší než u synchronního.
Data jsou dávkována mezi START bitem a STOP bitem.
\subsection{Asymterický signál}
Signál je na jedné svorce zpravidla na společný vodič (GND).
Rušící signál vytváří zdroj poruch a snižování spolehlivosti přenosu.
Přenosové médium vyžaduje proto odstínění proti rušivým signálů (koax\dots)
\subsection{Symetrický signál}
Signál je šířen symetricky po obou vodičích v opačné polaritě.
Rušící signál vytváří na obou vodičích rušení stejné polarity.
Výstupní signál je dán rozdílem signálu v obou vodičích.
Lze použít levnější přenosové médium a překlenout větší vzdálenosti.