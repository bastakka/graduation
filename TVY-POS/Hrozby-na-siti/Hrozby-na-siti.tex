\section{Hrozby na síti}
\label{sec:hrozby-na-siti}
Hrozby můžou být software či hardware sestrojené za účelem odcizení dat, peněz, informací nebo k zablokování přístupu.
Je nutné se proti tumto bránit na všech vrstvách ISO/OSI.
Mezi takové hrozby může patřit:
\begin{itemize}
  \item SW
        \begin{itemize}
          \item Viry - bez vědomí uživatele spouští v PC operace
          \item Malware - zajišťuje tajný přístup do zařízení
        \end{itemize}
  \item HW
        \begin{itemize}
          \item Sniffer - zařízení pro sběr dat
          \item Backdoor - brána pro připojení do sítě
        \end{itemize}
\end{itemize}
\subsection{Dělení malware}
\begin{description}
  \item[Spyware]- Odesílá uživatelovi data bez jeho vědomí.
  \item[Adware]- Odstranění agresivní reklamy za poplatek.
  \item[Phishing]- Stránka vydávající se za jinou. Po zadání přihlašovacích údajů se odešlou hackerovi.
  \item[Trojský kůň]- Funkční program, který obsahuje skrytý vir.
  \item[Červ]- Samo šířící se malware. Má v sobě zabudovanou šířící se funkci.
  \item[Keylogger]- Každý stisk klávesy se zaznamenává a odesílá zkrze internet hackerovi.
  \item[Ransomware]- Zašifruje disk a pro dešifraci je nutné zaplatit poplatek.
\end{description}
\subsection{Šíření malware}
\begin{itemize}
  \item Mailem - jako příloha, například pod falešnou identitou úřadu, policie\dots
  \item Trojský kůň
  \item Červ
\end{itemize}
\subsection{Odstranění malware}
\begin{enumerate}
  \item Odinstalace programu
  \item Odstavit vir pomocí antiviru
  \item Reinstalace OS
\end{enumerate}
\subsection{Druhy útoků}
\begin{description}
  \item[Brute force]- Snaha o prolomění ochrany hrubou silou. Zkoušení všech možných kobinací hesel. 
  \item[Social engineering]- Snaha o získání přístupu pomocí samotné interakce s uživatelem.
  \item[DNS spoofing]- DNS špatně přeloží adresu na adresu hackera, který poté získá údaje. (Phishing)
  \item[DDOS]- Vyřazení služby z provozu. Zahlcení infrastruktury provozem co není schopen zvládnout.
  \item[BOTnet]- Pomocí jednotlivě nakažených PC hacker provádí útoky ve vetším množství.
\end{description}