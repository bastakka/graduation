\section{IPv4 adresace}
\label{sec:ipv4-adresace}
Slouží k logické adresaci síťových prvků v internetu, ale i na lokální síti.
Komunikace se poté, co zařízení obdrží svou IPv4 adresu, řídí pomocí směrování.
IPv4 adresa se skládá z 4B, které zapisuje decimálně (Od 0 do 255).
\[\mathlarger{\mathlarger{\mathlarger{\mathlarger{192.168.0.1}}}}\]
Adressy rozdělujeme podle stanovené masky sítě:
\subsection{Classful (1. Epocha)}
První oktet adresy určuje třídu, neboli i masku
Rozdělují se na třídy:
\begin{description}
  \item[A] 1.0.0.0 - 126.0.0.0 mají masku /8
  \item[B] 128.0.0.0 - 191.255.0.0 mají masku /16
  \item[C] 192.0.0.0 - 233.255.255.0 mají masku /24
  \item[D] 224.0.0.0 - 239.255.255.255 mají masku /32 
\end{description}
\subsection{Classless (2. Epocha)}
Proměnná maska.
Zavedeno pro šetření adres.
Umožňuje nám vytvářem podsítě za pomoci FLSM nebo VLSM.
Masku sítě uvádíme suffixem /číslo, které nám udává počet binárních 1, které se budou v masce vyskytovat.
Proto $/8 = 1111 1111.0.0.0$.

Pro určení podsítě nebo nadsítě je nutné zjisit jestli je maska vyšší nebo nižší než maska pro adresu z první epochy.
Proto $192.100.100.0/28 =$ podsíť a $192.100.100.0/22 =$ nadsíť.

Tato maska umožňuje IP adresu rozdělit na dvě části.
Na část síťovou a část hostitele.
Z adresy $192.100.100.0/16$ jsem tedy schopen zjistit že se jedná o síť $192.100.0.0$ a hostitele $192.100.100.0$.
\subsection{Lokální adresy}
Pro šetření adres je možné použít adresy určené pouze pro přenos po lokální síti.
Nejsou routovatelné do internetu.
Dvě zařízení ve dvou různých lokálních sítích mohou mít stejnou adresu.
Mezi tyto adresy patří
\begin{itemize}
  \item 10.0.0.0 - 10.255.255.255
  \item 172.16.0.0 - 172.31.255.255
  \item 192.168.0.0 - 192.168.255.255
\end{itemize}
\subsection{Speciální adresy}
\begin{itemize}
  \item 0.0.0.0/32 - Využíváno hostem pro žádost o logickou adresu od DHCP serveru.
  \item 127.0.0.0/8 - Využíváno pro připojení se samo na sebe. (Loopback)
  \item 255.255.255.255/32 - Lokální broadcast
  \item 224.0.0.0 - 239.255.255.255 - Multicast směrovací protokoly (Třída D)
\end{itemize}
Všechny adresy co nespadají do rozsahu speciálních nebo lokálních adres jsou globální.
