\section{Linková vrstva}
\label{sec:linkova-vrstva}
Jedná se o druhou nejnižší vrstvu ISO-OSI modelu.
Zajišťuje integritu toku dat, synchronizuje bloky bitů a přepíná komunikace vzhledem k fyzickým adresám přijímačů.
Její datovou jednotkou je rámec, který se posílá posílá na většinou určenou MAC adresu.
\subsection{MAC adresa}
Každé zařízení má od výroby přiřazenou MAC adresu používanou pro komunikaci na linkové vrstvě.
Staří verze MAC adres má formát šesti oktetů (FF:FF:FF:FF:FF:FF), nově se používá i formát osmi oktetů (FF:FF:FF:FF:FF:FF:FF).
Jestliže se posílá informace na adresy ze samých FF, tak se jedná o broadcast.
První tři oktety adresy určují výrobce zbytek pro samotné zařízení.
Dále se MAC adresy dělí na:
\begin{description}
  \item[Unikátní]- Určené od výrobce
  \item[Lokální]- Přepsaná administrátorem či softwarem
\end{description}
Toto je určeno pomocí předposledního bitu prvního oktetu.
Poslí byt prvního oktetu určuje jestli se jedná o unicat či multicast.
\subsection{Zařízení pracující na linkové vrstvě}
\subsubsection{Bridge}
Rozděluje síť na dvě části.
Podle MAC adresy buďto rámec pošle na druhou stranu nebo ho dál nepošle.
Nepotřebuje konfigurovat, jelikož si sám ukládá kde daná MAC adresa leží po prvním zjištění pomocí broadcastu.
\subsubsection{Switch}
\label{subsubsec:switch}
Nahrazuje HUBy.
Dal by se nazvat jako bridge s více porty.
Propojuje jednotlivé prvky do hvězdicové topologie.
Má tabulku se seznamem MAC adres připadajících na port a podle toho rozesílá data.
Takzvaně \textbf{PŘEPÍNÁ} komunikaci.
Existují i L3 switche, které dokáží rozesílat data i podle IP adres.
K přenosu rámců switch používá tři způsoby:
\begin{description}
  \item[Flooding]- Rámec z jednoho portu je přenesen na všechny ostatní porty.
  \item[Forwarding]- Rámec předá pouze portu podle určené MAC adresy.
  \item[Filtering]- Switch přijme rámec na kterém je cílová MAC adresa na stejném portu jako port, ze kterého rámec přišel a tak rámec zahodí.   
\end{description}
