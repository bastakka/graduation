\section{Logické členění paměti podle určení}
\label{sec:logicke-cleneni-pameti}
Paměť je médium, které umožňuje uchovávat informace.
Pro každé prostředí je potřeba paměť s jinou specifikací.
Jinde jsou potřeba paměti s velmi vysokou rychlost, někde s velmi krátkou přístupovou dobou nebo s velmi velkou kapacitou.
V počítači používáme následující paměti:
\subsection{Konvenční paměť}
Bývalé označení paměti v DOS, která měla 640 kB.
Sloužila pro aplikace a pracovalo se v ní v Realném režimu.
Zbytek paměti byl určet pro rozšiřující karty.
\subsection{Cache}
Jedná se o velmi rychlou paměť malé kapacity.
Slouží ke zrychlení toku dat mezi CPU a RAM.
Na procesoru se jich nachází více a každá z nich má svůj řadič.
Od Bufferu se liší tím, že může uložená data poskytovat opakovaně.
V procesoru se mohou nacházet až tři vrstvy L + číslo cache pamětí (L1, L2, L3).
Tyto paměti slouží k ukládání právě zpracovávaných instrukcí a dat.
Stupňují se s rychlostí a kapacitou.
L1 je nejrychlejší ale nejmenší.
L3 je naopak nepomalý, ale největší. 
I přesto ale je její rychlost neporovnatelná s rychlostí konvenční paměti.
\subsubsection{Asociativnost cache}
\begin{description}
  \item[Asociativní]- Vyhládáví v paměti pomocí klíče
  \item[Nesociativní]-  Vyhledávání v paměti pomocí adresy
\end{description}
\subsubsection{Adresace cache}
\begin{description}
  \item[Virtuální]- Vyhledávání podle logické adresy
  \item[Fyzická]- Výpočet fyzické adresy a až poté vyhledávání v paměti
\end{description}
\subsubsection{Synchronnost cache}
\begin{description}
  \item[Synchronní]- CPU po zadání adresy nečeká na data. Požádá o ně v dalším taktu. Odpadá čekání.
  \item[Asynchronní]-  Sběrnice je obsazena celou dobu přenosu. Nepouživá se.
\end{description}
\subsection{Buffer}
Buffer neboli vyrovnávací paměť má za úkol vyrovnat rychlosti, jesltiže vstupní data přicházejí moc rychle nebo sporadicky.
Pro práci s daty potřebujeme konstatní datovou rychlost a tu nám v takovém případě poskytuje Buffer.
Paměť je typu FIFO, tudíž první data, která do paměti přijdou, z ní také první odejdou.
Tyto data už poté nelze z této paměti získat znova.
\subsection{Segment}
Segment je část operační paměti, která je určena pro jeden konkrétní účel.
Délka tohoto segmentu je určena podle takového účelu.
Při výpočtu adresy paměti je k této informaci ještě potřeba zjisit offset, aby bylo možné z takového segmentu vybrat pouze potřebnou část dat.
\subsection{Page}
Stránkování paměti je proces, který rozdělí operační pamět na stejně dlouhé úseky o malé velikosti.
Tyto úseky poté přiděluje jednotlivým procesům v množství, ve kterém potřebují.
Tyto úseky je možné narozdíl od segmentu rozdělit po celé operační paměti a není nutné, aby byl přidělován celý blok v jednotném tvaru.
Pro aplikace se jeví paměť poté jakože má k dispozici paměť od 0 až po maximum -1.
Adresa se musí překládat pomocí MMU.
Velmi efektivní způsob rozdělení paměti.

