\section{Moderní CPU s integrovanými podpůrnými technologiemi}
\label{sec:moderni-cpu}
Integrované podpůrné technologie CPU urychlují komunikaci procesoru s obovody základní desky počítače.
Mezi další výhody integrovaných podpůrných technologií je odlehčení procesoru nebo dokonce jeho zastoupení.
\subsection{MMU - Memory Management Unit}
Slouží jako vrstva mezi CPU a operační pamětí.
Hlavní funkce jsou:
\begin{itemize}
  \item Překlad virtuální adresy na fyzickou
  \item Správa virtuální paměti
  \item Rozdělení virtuálního prostoru na stránky/segmenty
\end{itemize}
Překlad adresy musí být bezpečný, rychly a umožňovat rychlé přepínání kontextu.
\subsubsection{TLB - Translation Lookaside Buffer}
Součást MMU, která ukládá v sobě nedávné překlady z virtuální paměti na fyzickou.
Když bude znova překlad potřeba, použije se rychlejši přístupný překlad v TLB.
\subsection{MMX - Multi Medial eXtension}
Aplikace využívající zvuk a video, 3D grafiku a animace.
Práce s těmito prvky se provádí ve velkých blocích dat, kde se vykonávají jednoduché repetetivní operace.
Proto se pro zrychlení práce s multimédií využívá \textbf{SIMD (Single Instruction, Multiple Data)} technologie.
Tato technologie popsiuje schopnost několika prvků provádět stejnou operaci na různých datech paralelními výpočty.
Kvůli lepší kompatibilitě jsou MMX registry aliasy pro instrukční sady x87 pro opreace s floaty.
\subsection{Cache}
Jedná se o velice rychlou SRAM paměť zabudouvanou do CPU.
Paměť cache umožňuje vyhnout se opakovanému pomalému procesu, na který bychom museli čekat.
Místo toho provedeme víc takových procesů najednou a poté tyto data nahrajeme do pamětí cache.
Mimo nahrávání dat lze do cache nahrát i instrukce pro omezení potřeby je překládat.
Jinak také cachujeme data z DRAM, které dostáváme z MMU.
\subsection{Pipelining}
Snaží se uspořádat instrukce tak aby byl procesor nejefektivněji využit.
Rozdělí příchozí instrukce na po sobě jdoucí kroky, které pak přiřadí správným částem procesoru k vykonání což umožňuje paralelní zpracování různých instrukcí.
Častou praktikou je aby se liché operace prováděli na jedné hraně hodin a sudé na druhé.
Správně nakonfigurovaný pipeline dokončí jednu instrukci každý cyklus.
Tento problém nemusíme řešit při programování ve vyšších jazycích jak assembler.
\subsection{Branch prediction}
Jakmile v rámci procesu dojde k rozhodování mezi více možnostmi, vytváří se různé větve, kterými se proces můžet vydat.
CPU neví, po které větvi se vydá, kvůli tomu musí čekat než dostane výsledek.
Pokud se snažíme předpovědět větev:
\begin{itemize}
  \item Uhodneme správně - navýšení rychlosti
  \item Uhodneme špatně - ještě větší zpomalení kvůli zbytečné práci
\end{itemize}
Rozdělení na dvě větve je většinou řešeno tak, že se druhá větev načte na jiné místo v paměti programu, a pokud je splněna podmínka pro přepnutí větvě, tak provedem skok na toto jiné místo.
Jestli ne, pokračujeme ve staré větvi.
Dynamický branch predictor si uchovává kolikrát se na daném skoku skočilo na jinou větev a kolikrát ne, podle těchto nasbíraných dat se snaží uhodnout zdali skoční neb one.