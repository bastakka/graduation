\section{Motherboard}
  Deska plošných spojů, která elektricky a fyzicky propojuje jednotlivé komponenty počítače.
  Funkce základní desky:
  \begin{enumerate}
    \item Napájet komponenty
    \item Mechanicky udržet komponenty u sebe
    \item Umožnit rychlý a spolehlivý přenos dat z jedné periférie do druhé
  \end{enumerate}
  Možnosti zapojení komponent k základní desce:
  \begin{itemize}
    \item Interně - Porty a vstupy uvnitř case na základní desce
    \item Externě - Porty a vstupny z vnější strany case, taktéž na zákldní desce
  \end{itemize}
  Základní desky nejsou jenom v počítačích, ale i noteboocích, mobilech aj.\\
  Velikosti základních desek: E-ATX, ATX, mATX, ITX\dots
  \subsection{Blokové schéma a jednotlivé komponenty}
    \includegraphics[width=1\linewidth,height=0.58\linewidth]{TVY-POS/Motherboard/MB.png}
    Popis fyzického schéma:
    \begin{enumerate}
      \item[1)] Patice CPU - Rozdíl mezi AMD (AM4, AM3) a Intel (12th gen LGA1700)
      \item[2)] CPU FAN Header - Pro chlazení CPU, jiné usazení chladiče pro AMD a Intel
      \item[3)] ATX napájení - Napájení základní desky a jejich komponent
      \item[4)] DDR sloty - Rozdíl mezi DDR3, DDR4, DDR5 (Umístění mezery)
      \item[8)] Porty pro case - HD Audio, USB, PWR, Reset, HDDLED, PWRLED 
      \item[9)] RTC - Hodiny reálného času, baterie + kondenzátor, 32.768 kHz, 1.7 sekundy chyba denně
      \item[10)] Southbridge - Propojení NB a Serial \& Paralel port, PS2 klávesnice, myš
      \item[11)] PCI - Univerzální rozšiřující sloty. x4, x8, x16
      \item[12)] BIOS - Flash pamět se zaváděcím systémem
      \item[13)] PCI-E - Vysokorychlostní rožšiřující slot, v dnešní době hlavně pro grafické karty.
      \item[14)] Northbridge - Rozhraní pro propojení CPU s opereační pamětí a PCI-E
      \item[15)] FAN Header - Napájení a ovládání
      \item[] CPU napájení - Chybí na obrázku
    \end{enumerate}