\section{Optická zařízení}
\label{sec:opticka-zarizeni}
Optické zařízení je médium uchovávající informace při jehož čtení se využívá zdroje světla.
V dnešní době se jedná o 12cm disky s dírou 1.5cm uprostřed a tlouštku 1.2mm.
Skládají se většinou ze tří částí:\\
\begin{enumerate}
  \item Směs polykarbonátu a polymethylmethakrylátu s vytlačenou spirálou pro data
  \item Reflexní vrstva pro odrážení světla laseru
  \item Ochranná vrstva - lak, potisk\dots
\end{enumerate}
\subsection{Druhy optických zařízení}
\subsubsection{CD}
Nejstarší optické médium.
Využívá vlnovou délku 870nm (Červená).
Má maximální velikost 650 MB, rychlost v násobcích 150kB/s a zapisuje se pouze na jedné straně.\\
\begin{description}
  \item[CR-ROM]- Pouze pro čtení, rychlost otáčení buďto konstatní nebo u okraje rychleji, lisování podle matice
  \item[CD-R]- Umožňuje jednorázoví zápis, poté funguje jako CD-ROM, speciální 4x rychlejší a silnější laser.
  \item[CD-RW]- Dovoluje opakovaně zapisovat data. Datová vrstva buďto krystalizuje nebo se mění na amorfní.
  Obodoba prohlubní a mezery pro zápis. Zápis zahřátím, čímž se převede disk do amorfního stavu.
\end{description}
\subsubsection{DVD}
Nástupce CD. Má větší kapacitu pomocí větší hustoty zápisu.
Využívá přesnější laser 660nm (Stále červená).
Má možnost oboustranného zápisu či dvouvrstvého provedení.
Spirála pro data je stejně dlouhá a široká jak u CD.
Pro zápis na pouze jednu z vrstev laser zaostří pouze na jednu, spodní vrstva je polo transparentní.
Má stejnou tloušťku pomocí poloviční tloušťky nosných vrstev.
Přenosová rychlost je v násobních 1350 kB/s.
Mechanika pro DVD měla dvakrát přesnější čtecí hlavy, po změně vlnové délky bylo nutné ovšem do mechanik přidávat laser pro čtení CD zvlášť.
\subsubsection{DRM}
DVD přineslo možnost DRM (Digital Rights Management) pro ochranu dat.
Na disku lze uložil copyright, který se při kopirování nepřenáší s číslem regionu.
Čtení bylo poté možné pouze jestliže na disku je nacházel tento copyright a pouze ve správném regionu.
\subsubsection{Blu-ray}
Nejmodernější optické médium.
Využívá nejmenší vlnovou délku 405nm (Modrá).
Tudíž má i vetší hustotu zápisu a větší kapacitu (Jedna datový vrstva 25GB).
Datová vrstva je blíže pod povrchem, aby nedocházelo k rozptylu laseru, ovšem to znamená menší odolnost vůči škrábancům.
Existují přepisovatelné a i menší verze disku.
Rychlosti v násobku 4.5MB/s.
\subsection{Čtení}
Mechanika obsahuje laser, která za pomocí čoček zaostřuje na disk.
Tento laser poté veden podel spirály v první vrstvě disku.
Odrázené světlo se vrací do senzoru za polopropustným zrcadlem.
Záznam proveden pomocí různě dlouhých prohlubní oddělený mezerami.
Přechod mezi prohlubní a mezerou je log. 1.
Delší prohlubeň či mezera je log. 0.
