\section{Činnost CPU při zpracování přerušení, DMA a podprogramu}
\label{sec:cpu-preruseni}
Por svou činnost CPU potřebuje vstupy a výstupy.
Jelikož ale není možná aby CPU konstantně kontroloval zda-li nějaké zařízení předává vstup, tak se takové nové požadavky řeší pomocí přerušení.
Defakto se jedná o předání informace, že se něco něco stalo a vyžaduje pozornost.
Existují různé druhy přerušení.
\subsection{Přerušení}
Přerušení je \textbf{asynchronní}, což znamená, že když nastane, tak se vykoná ihned.
Využití pro myš, klávesnice, USB, ale i při dělení nulou.
Přerušení můžou mít prioritu pro určení pořadí jejich zpracování.
Je možné taky zakázat přerušení na CPU, ale některé nemohou být takto zakázána, jelikož například souvisí s právě spuštěným programem.
V operační paměti je tabulka přerušení a pro jeho vykonání si CPU uloží svůj předchozí stav do zásobníku.
Rozdělení přerušení:\\
\begin{description}
  \item[Vnitřní HW]- Chyba při provádění instrukce (Dělení nulou)
  \item[Vnější HW]- Periferie potřebuje zpracování dat (myš\dots)
  \item[SW]- Volání podprogramu
\end{description}
\subsection{Podprogram}
Podprogram je \textbf{synchronní}, což znamená, že když to kdy nastane nemá vliv na jeho výkon.
Vykoná se až CPU určí.
Jinak se s ním pracuje stejně jako s přerušením.
\subsection{DMA (Direct Memory Access)}
DMA je speciální druh přerušení, u kterého CPU nemusí přestat pracovat.
CPU v tomto případě akorát zahájí přenos dat nebo z paměti a pak dále pokračuje v jeho činosti mezitím co DMA přenáší data.
DMA se většinou nachází na jižním můstku.
Jestliže CPU v průběhu DMA potřebuje přístup do datové sběrnice, tak ovšem musí počkat na její uvolnění.
Využíváno pro: Paměťová média, grafické, zvukové, síťové karty.
DMA musí podporovat obě strany přenosu.
