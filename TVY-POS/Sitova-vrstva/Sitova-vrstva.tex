\section{Síťová vrstva}
Zajišťuje přenos dat mezi vzdálenými počítači (WAN).
Klíčovým prvkem v této komunikaci je směrovač (router).
Každý takový směrovač má svoji jednoznačnou identifikaci v rámci WAN (IP Adresu).
Přenáší se IP datagramy neboli pakety. Nezajímá se o protokoly linkové a fyzické vrstvy.\\
Funkce síťové vrstvy:
\begin{itemize}
  \item Hledání cesty pro pakety mezi libovolnými dvěma uzly v síti
  \item Nestará o spolehlivost, ale o co nejrychlejší přenos dat
  \item Zajistění postupného přenosu paketů přes mezilehlé uzly v cestě
  \item Přenosový protokol IP se snaží zakrývat rozdíly v technologiích nižší vrstvy
\end{itemize}
Cesta paketu:
\begin{enumerate}
  \item Zabalení přenášeného paketu do rámce
  \item Prostřednictvím vrstvy síťového rozhraní předání rámce přímému sousedovi
  \item V sousedním uzlu přijmutí a rozbalení rámce vrstvou síťového rozhraní
  \item Předání získaného paketu své síťové vrstvě, která najde nejvhodnější cestu k cíli
  \item Prostřednictvím vrstvy síťového rozhraní zaslání rámce k dalšímu uzlu
\end{enumerate}
\subsection{Směrování}
Technika k vnitřnímu rozčlenění rozsáhlých sítí LAN i MAN.\\
Běžně se používá jako proces šíření globální síťové komunikace v Internetu.\\
Účely směrování:
\begin{itemize}
  \item Usměrnění komunikace
  \item Optimalizace zátěže sítě
  \item Implementace bezpečnosti
  \item Zvýšení spolehlivosti na síťové vrstvě
\end{itemize}
\begin{description}
  \item[Statické]je nastaveno pevně administrativně - DHCP
  \item[Dynamické]je pravidelně aktuailzováno speciálními protokoly - Mnoho
\end{description}
\subsection{IP Adresace}
Každá síťová stanice musí mít pevně stanovenou identifikaci - IP adresu.
Ta je buďto napevno přidělena nebo dynamicky přidělována.
V současné době se využívá IPv4 adresa o velikosti 32bitů.
Tato IP adresa má v sobě zahrnuté jak číslo sítě, tak číslo stanice.
Tvar adresy je \[XXX.XXX.XXX.XXX\]
V každé síti musí být jedinečný router, který síť propojuje pomocí směrování. IP adresa hostitele a routeru musí být odlišná.\\
Jiné protkoly dostupné pro síťovou vrstvu: ICMP - Internet Control Message Protocol (Řešení problémů IP, ping\dots), IGMP, ARP, RARP.
