\section{Síťová vrstva}
\label{sec:sitova-vrstva}
Jedná se o třetí vrstvu ISO-OSI modelu.
Každé zařízení dostává na síti jednoznačnou identifikaci v podobě logické adresy (většinou IPv4 nebo IPv6).
Zajišťuje komunikaci mezi zařízeními pomocí těchto adres.
Její datovou jednotkou je paket.
V rámci ISO-OSI a TCP/IP je tato vrstva stejná.
Nerozlišuje fyzickou ani linkovou vrstvu.
\subsection{Zařízení fungující na síťové vrstvě}
\subsubsection{NIC}
Zařízení připojující PC do sítě.
Může připojit i více počítačů, ale většinou se jedná o karty připojované do základové desky nebo o integrovaný čip na ní.
Její rozhraní se dá přímo adresovat, posílat a přijímat jí data.
\subsubsection{Router}
Zařízení které jednotlivé připojené zařízení připojuje do sítě LAN, pomocí které se pak zařízení připojují do sítě WAN.
Zajišťujě směrování paketů (přenos na správnou logickou adresu).
Směrování zajiťuje pomocí směrovací tabulky, ve které je možné určit buďto cestu statickou nebo dynamickou za pomoci \hyperref[sec:smerovaci-protokoly]{dynamických směrovacích protkolů.}
V dnešní době toto zařízení v sobě obsahuje i DNS a DHCP server.
DNS server překládá URL adresy na logické adresy (google.com => 142.251.36.142).
DHCP server přiděluje logické adresy jednotlivých klientům na síti.
Doporučené nastavení DHCP serveru je přidělení nejvyšší možnou adresu routeru, nejnižší serverům a klientům zbytek.
\subsubsection{L3 Switch}
Oproti klasickému \hyperref[subsubsec:switch]{switchy} umí komunikaci oddělovat a \textbf{SMĚROVAT} i na třetí vrstvě pomocí logických adres.
Lze ho takto i dobře použít na komunikaci v rámci \hyperref[sec:vlan]{VLAN}
\subsubsection{Gateway}
Zařízení které spojuje sítě s jinými komunikačními protokoly.
Například počítač který pomocí webové stránky odešle zprávu do GSM pomocí SMS zprávy.
Může akorát změnit datagramy transportní vrstvy místo kódování celé zprávy.
\subsection{Protokoly síťové vrsty}
\subsubsection{IP}
Stará se adresaci zařízení.
Má také za úkol dostat pakety z jedné sítě do druhé.
Tato komunikace je nespolehlivá a nespojová, proto se o integritu dat stará transportní vrstva (například TCP).
Jestliže se využívá například UDP protokolu na transportní vrstvě, tak se o integritu starají protkoly vyšších vrstev.
\subsubsection{ICMP}
Využívám pro odesílání chybových hlášení.
Například pro oznání o nedostupné službě či nedosažitelného cíle.
\subsubsection{ARP}
Pro naplnění směrovacích tabulek routerů a switchů je nutné použít ARP protokol pro získání fyzické adresy a logické adresy od zařízení na síti.